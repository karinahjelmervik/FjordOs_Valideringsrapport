
\section{Introduction}
Provided is an evaluation of the FjordOs model \citep{roed:etal:2016}. The aim of this study is to reveal any weaknesses of the FjordOs model and clarify the extent to which the model can be be trusted. The study is a part of the FjordOs project. FjordOs is a cooperation between MET Norway, University College of Southeast Norway (HSN), The Norwegian Institute for Water Research (NIVA), The Norwegian Coastal Administration (Kystverket), Exxonmobil, Norwegian Defence Research Establishment (FFI), Vestfold, Buskerud, and \O stfold county, and AGNES AB Milj\o konsulent.

The evaluation is based on existing observations in the area of interst and results from the FjordOs model \citep{roed:etal:2016}. The observations are gathered from different sources and not carried out as a part of the FjordOs project. The observations includes measurements of water level, current measurements, water temperature, and CTD measurements. A short scientific cruise on board the research vessel (R/V) Trygve Braarud in September 2015 provided additional observations of hydrography and trajectories of drifters. These observations are compared with simulated results in \cite{hjelm:etal:2016}. In addition some observations was performed close to Svelvik in 2015 in order to evaluate the model in the Drammensfjord, the western branch of the Oslofjord \citep{staalstrom:2017}. 



\newpage
