
\section{Introduction}

We assess the performance of a new regional circulation model developed specifically for the Oslofjord, Norway. The model is named FjordOs and is a version of the Regional Ocean Model System (ROMS) adapted for the fjord utilizing its curvilinear option. For details on the FjordOs model the reader is referred to \cite{roed:etal:2016}. 

The motivation is to construct a model with high enough resolution to properly resolve the fjord's many small islands, narrow straits and sounds, and to resolve its highly irregular coastline and topography \ref{fig:kart}. The hypothesis is that such a resolution is necessary to avoid effluents to be stranded artificially when simulating their pathways from the source. As is well known currents, besides wind and waves, is one of the dominant sources when predicting pathways of effluents like oil and/or discharges of other contaminants. Thus the FjordOs model is designed to deliver simulation and/or forecasts of water level, current, temperature and salinity accurate enough to be a useful input to drift models.
The study is a part of the FjordOs project. FjordOs is a cooperation between MET Norway, University College of Southeast Norway (HSN), The Norwegian Institute for Water Research (NIVA), The Norwegian Coastal Administration (Kystverket), Exxonmobil, Norwegian Defence Research Establishment (FFI), Vestfold, Buskerud, and {\O}stfold county, and AGNES AB Milj{\o}konsulent.

The evaluation is based on available observations for the two year period 2014 and 2015 for which simulations with the FjordOs model is performed. Most of the observations are gathered from different sources independent of the FjordOs project, and include measurements of water level, currents, and water temperature at fixed station in addition to CTD measurements scattered in time and space. In addition a short scientific cruise on board the research vessel (R/V) Trygve Braarud was conducted as part of the FjordOs project in September 2015 to provide additional observations of hydrography and not least trajectories of drifters \citep{hjelm:etal:2016}. Finally some observations was performed close to Svelvik in 2015 in the Drammensfjord, a western branch of the main Oslofjord \citep{staalstrom:2017}.

While Section 2 and 3 give a brief introduction to the Oslofjord and the model, more details on the observations are offered in Section 4. The assessment of the model's performance is presented in Section 5, while a summary including conclusions and some final remarks are proffered in Section 6. 



%Provided is an evaluation of the FjordOs model \citep{roed:etal:2016}. The aim of this study is to reveal any weaknesses of the FjordOs model and clarify the extent to which the model can be be trusted. The study is a part of the FjordOs project. FjordOs is a cooperation between MET Norway, University College of Southeast Norway (HSN), The Norwegian Institute for Water Research (NIVA), The Norwegian Coastal Administration (Kystverket), Exxonmobil, Norwegian Defence Research Establishment (FFI), Vestfold, Buskerud, and \O stfold county, and AGNES AB Milj\o konsulent.

%The evaluation is based on existing observations in the area of interst and results from the FjordOs model \citep{roed:etal:2016}. The observations are gathered from different sources and not carried out as a part of the FjordOs project. The observations includes measurements of water level, current measurements, water temperature, and CTD measurements. A short scientific cruise on board the research vessel (R/V) Trygve Braarud in September 2015 provided additional observations of hydrography and trajectories of drifters. These observations are compared with simulated results in \cite{hjelm:etal:2016}. In addition some observations was performed close to Svelvik in 2015 in order to evaluate the model in the Drammensfjord, the western branch of the Oslofjord \citep{staalstrom:2017} . 



\newpage
