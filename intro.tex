
\section{Introduction}

We assess the performance of a new regional circulation model for the Oslofjord, Norway. The model is named FjordOs, and was recently developed specifically for the Oslofjord as detailed in \cite{roed:etal:2016}. The model is a version of the Regional Ocean Model System (ROMS) adapted for the fjord utilizing its curvilinear option. 

The rationale behind the development of the the FjordOs model was to construct a model with high enough resolution to properly resolve the Oslofjord's highly irregular coastline and topography, and thereby resolve the fjord's many small islands, narrow straits and sounds as shown by Figure \ref{fig:kart}. The intended use of the model is to provide currents as input to a drift model to be able to forecast any drift of oil or other effluents to the fjord. Hence the model resolution must be high enough to avoid effluents stranding artificially when simulating their pathways from the source. As is well known currents, besides wind and waves, is one of the dominant sources when predicting pathways of effluents like oil and/or discharges of other contaminants. Thus, the FjordOs model is designed to deliver simulation and/or forecasts of water level, current, temperature and salinity accurate enough to be a useful input to drift models.

The evaluation is based on available observations for the two-year period 2014 and 2015 for which simulations with the FjordOs model is performed. Most of the observations are gathered from different sources independent of the FjordOs project, but also include measurements gathered during two short scientific cruises conducted by use of the research vessel (R/V) Trygve Braarud. The observations consist of measurements of water level, currents, temperature and salinity (CTD and water temperature at fixed stations), and trajectories of drifters \citep{hjelm:etal:2016}. Most of the data are scattered in time and space. Finally, some observations were performed close to Svelvik in 2015 in the Drammensfjord, a western branch of the main Oslofjord \citep{staalstrom:2017}.

Section \ref{sec:area} gives a brief introduction to the Oslofjord and the model, while Section \ref{sec:obser} offers details on the observations. The assessment of the model's performance is presented in Section \ref{sec:evalu}, while a summary including conclusions and some final remarks are proffered in Section \ref{sec:summa}. Some calculations on drifting lanes from the Slagen refinery are added in the Appendix.

