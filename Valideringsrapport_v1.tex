\documentclass[12pt,a4paper,english]{article}
\usepackage[utf8]{inputenc}
\usepackage[english]{babel}
\usepackage{graphicx}
\usepackage{epsfig}            % To allow figures
\usepackage{pstricks}          % To draw color pictures directly
\usepackage{fancyhdr}
\usepackage[verbose,a4paper,tmargin=30mm,bmargin=37.5mm,lmargin=30mm,rmargin=30mm]{geometry}
\usepackage{helvet}
\usepackage{mathptmx}
\usepackage[T1]{fontenc}
\usepackage{rotating}
\usepackage{amssymb}       % More special characters          
\usepackage{amsmath}       % More mathematical characters
\usepackage{natbib} % Needed e.g. for \citep
\usepackage{multirow}
\usepackage{verbatim}          % Use the verbatim styles
\definecolor{METblue}{cmyk}{0.85,0,0.2,0.2}
\usepackage{titlesec}
\titleformat{\section}
  [block]
  { \normalfont\sffamily\Large\bfseries\color{METblue} }
  {\makebox[2em][r]{\thesection}}
  {5mm}
  {\vspace{5mm}}

\titleformat{\subsection}[block]%
  {\normalfont\sffamily\bfseries}%
  {\makebox[2em][r]{\thesubsection}}%
  {5mm}
  {\vspace{3mm}}[]
\titleformat{\subsubsection}[block]%
  {\normalfont\sffamily\bfseries}%
  {\makebox[3em][r]{\thesubsubsection}}%
  {5mm}
  {\vspace{3mm}}[]

% Titlespacing syntax: 
\titlespacing*{\section}{-17mm}{5mm}{0mm}
\titlespacing*{\subsection}{-13.5mm}{5mm}{0mm}
\titlespacing*{\subsubsection}{-17.5mm}{5mm}{0mm}

%%% Formatting the table of contents
\usepackage{tocloft}
\renewcommand{\cfttoctitlefont}{\sffamily\bfseries\color{METblue}\Large}
\providecommand{\cftchapfont}{\sffamily\bfseries }
\renewcommand{\cftsecfont}{\sffamily\bfseries }
\renewcommand{\cftsubsecfont}{\sffamily }
\renewcommand{\cftsubsubsecfont}{\sffamily }
\providecommand{\cftsubsubsubsecfont}{\sffamily }
\renewcommand{\cftfigfont}{Figure }
\renewcommand{\cfttabfont}{Table }
\providecommand{\cftchappagefont}{\sffamily}
\renewcommand{\cftsecpagefont}{\sffamily\bfseries}
\renewcommand{\cftsubsecpagefont}{\sffamily}
\renewcommand{\cftsubsubsecpagefont}{\sffamily}
\providecommand{\cftsubsubsubsecpagefont}{\sffamily}

\renewcommand{\baselinestretch}{1.33}

% --------------------------------------

\begin{document}

\bibliographystyle{agufull04}
%\bibliographystyle{ams}

\pagenumbering{roman} % Use roman numerals for page numbering before main content. OL: The new template begins with page 1 on the cover.
\thispagestyle{empty}  % Hide page numbers

\noindent
\begin{tabular}{@{} p{63mm} p{50mm} r}
\includegraphics*[]{met_rapport_logo_eng} % Automatically uses PDF or EPS in same directory depending on latex or pdflatex.
&
\fontsize{27.5pt}{33pt} \selectfont \bf \sffamily MET{\color{gray} report}
&
 \begin{minipage}[b]{28mm}
  \begin{flushright}
   \footnotesize \sffamily No. X/2016 \\ ISSN 2387-4201 \\ Oceanography              % Report number and Category
  \end{flushright}
 \end{minipage}
\end{tabular}

\vfill

\begin{flushright}
{\fontsize{30pt}{36pt}\selectfont \bf \sffamily Evaluation of the FjordOs-model}          % Title
 
\vspace{5mm}
{\fontsize{12.5pt}{15pt}\selectfont \sffamily July 2016                                          % Subtitle
\\
\sffamily Karina Hjelmervik$^1$, Andr\'{e} Staalstr\o m$^2$, Nils M. Kristensen$^3$, Lars P. R\o ed$^{3,4}$% Author name(s)
}
\end{flushright}

%\vspace{25mm}
\vspace{2mm}


\begin{figure}[!h]
%\includegraphics*[height=5.5cm,angle=7]{Figurer/Driftere_ombord} \\
\vspace{3.5cm}
\begin{center}
\rput[bl](-7.5,-0.2){\includegraphics[height=1.7cm]{Figurer/logo_hsn}} 
\rput[b](0,0.1){\includegraphics[height=1.2cm]{Figurer/logo_met}} 
\rput[br](7.5,0){\includegraphics[height=1.2cm]{Figurer/logo_niva}} 
\end{center}
\end{figure}
\vspace{-1cm}
\noindent$^1$University College of Southeast Norway, 
$^2$Norwegian Institute for Water Research, \\ 
$^3$Norwegian Meteorological Institute,
$^4$Department of Geosciences, University of Oslo
\\




%\newpage

%\thispagestyle{empty}  % Hide page numbers

\clearpage

\setlength{\unitlength}{1mm}  %Needed for picture environment

\begin{table}[!ht]

\begin{tabular}[c]{lr}
\vspace{5mm}
\includegraphics*{met_rapport_logo_eng} & \hspace{43mm}
{\fontsize{27.5pt}{33pt}\selectfont \bf \sffamily MET{\color{gray} report}}\\
\end{tabular}

\sffamily{
\begin{tabular}[t]{|p{110mm}|p{40mm}|} \hline
{\bf \sffamily Title}                  & {\bf \sffamily Date}               \\ 
Evaluation of the FjordOs-model
                             & \today                   \\ \hline
{\bf \sffamily Section}                & {\bf \sffamily Report no.}         \\ 
 Ocean and Ice                        &  X/2015                  \\ \hline
{\bf \sffamily Author(s)}                 & {\bf \sffamily Classification}     \\ 
Karina Hjelmervik, Nils Melsom Kristensen, Andr\'{e} Staalstr\o m, Lars Petter R\o ed                 
                             & \begin{picture}(20,4)(-2,-1.0)
                               \put (0,0){\circle*{4}}
                               \put (7,0){\makebox(0,0){Free}}
                               \put (15,0){\circle{4}}
                               \put (27,0){\makebox(0,0){Restricted}}
                               \end{picture}
                               \\ \hline
{\bf \sffamily Client(s)}              & {\bf \sffamily Client's reference} \\ 
Client name                  &               \\ \hline
\end{tabular}

\begin{tabular}[t]{|p{154.3mm}|}
{\bf \sffamily Abstract}                                          \\
Abstract text...
\\
\\
\\
%\\[50mm] % Add whitespace if necessary
\\ \hline
{\bf \sffamily Keywords}                                          \\ 
  relevant, keywords, here    \\ 
\hline
\end{tabular}
}

\begin{tabular}[t]{cc}
                             &                            \\
                             &                            \\
                             &                            \\
\line(1,0){70}               & \line(1,0){70}             \\ 
Disciplinary signature       & Responsible signature      \\
%Jan Erik Haugen             & \O{}ystein Hov             \\       % Add names if needed
\hspace{75mm}                & \hspace{75mm}              \\

\end{tabular}
\end{table}

\clearpage

\thispagestyle{fancy} % footer from fancyhdr package
\headheight=15pt
\renewcommand{\headrulewidth}{0pt}

%\clearpage
\section*{\hspace{17mm}Abstract}
Abstract text...

%\clearpage

\vfill

\fancyfoot{
% If abstract on separate page is not needed, move the following table to the page before
\begin{tabular}[b]{p{40mm}p{25mm}p{25mm}p{25mm}p{25mm}}
 \begin{minipage}[l]{40mm} \tiny \color{METblue} {\bf Norwegian Meteorological Institute}\\ Org.no 971274042\\ post@met.no\\ www.met.no / www.yr.no
 \end{minipage} & 
 \begin{minipage}[l]{25mm} \tiny \color{METblue} {\bf Oslo}\\ P.O. Box 43, Blindern\\ 0313 Oslo, Norway\\ T. +47 22 96 30 00
 \end{minipage} &
 \begin{minipage}[l]{25mm} \tiny \color{METblue} {\bf Bergen}\\ All\'egaten 70\\ 5007 Bergen, Norway\\ T. +47 55 23 66 00
 \end{minipage} & 
 \begin{minipage}[l]{25mm} \tiny \color{METblue} {\bf Troms\o}\\ P.O. Box 6314, Langnes\\ 9293 Troms\o, Norway\\ T. +47 77 62 13 00
 \end{minipage} & 
 \begin{minipage}[l]{25mm} \tiny \color{METblue} 
 \end{minipage}
\end{tabular}
}

\clearpage
\tableofcontents

\clearpage
\pagestyle{plain}
\pagenumbering{arabic}

\section{Introduction - Andr\'{e}/Karina}
Provided is an evaluation of the FjordOs model \citep{roed:etal:2016}. The aim of this study is to reveal any weaknesses of the FjordOs model and clarify the extent to which the model can be be trusted. The study is a part of the FjordOs project. FjordOs is a cooperation between MET Norway, University College of Southeast Norway (HSN), The Norwegian Institute for Water Research (NIVA), The Norwegian Coastal Administration (Kystverket), Exxonmobil, Norwegian Defence Research Establishment (FFI), Vestfold, Buskerud, and \O stfold county, and AGNES AB Milj\o konsulent.

The evaluation is based on existing observations in the area of interst and results from the FjordOs model \citep{roed:etal:2016}. The observations are gathered from different sources and not carried out as a part of the FjordOs project. The observations includes measurements of water level, current measurements, water temperature, and CTD measurements. A short scientific cruise on board the research vessel (R/V) Trygve Braarud in September 2015 provided additional observations of hydrography and trajectories of drifters. These observations are compared with simulated results in \cite{hjelm:etal:2016}.



\newpage
\section{Area of interest}

The area of interest is the Oslofjord including the Drammensfjord and the Inner Oslofjord (Fig.~\ref{fig:kart}). The Oslofjord is located in Southern Norway with the capital of Norway, Oslo, in the innermost part of the fjord. 
The fjord is about 100 km long. Around 59.56$^o$N the fjord splits into two branches, the Inner Oslofjord (eastern branch) and the Drammensfjord (western branch). The width varies from around 25 km at the entrance to about 1-2 km at Dr\o bak Sound in the Inner Oslofjord and 180 meters at Svelvik in the Drammensfjord.

Numerous smaller and larger islands combined with deeper and shallower basins (Fig.~\ref{fig:kart}) contributes to a complex circulation pattern. In addition several river discharge fresh water into the fjord, including two of Norway's largest rivers, namely Glomma (near Fredrikstad) and Drammenselva (near Drammen).

\begin{figure}[htb]
\centerline{
\includegraphics*[trim=0cm 0.8cm 0cm 0cm,clip=true,width=0.7\textwidth]{Figurer/kart}
}
\caption{\small
Area of interest. 
}
\label{fig:kart}
\end{figure}

\newpage
\section{Model}

The FjordOs model is a curvilinear, free-surface, and terrain-following model based on the Rutgers Regional Ocean Modeling System (ROMS) \citep{haidv:etal:2008,shche:mcwil:2003,shche:mcwil:2005,shche:mcwil:2009} adapted to the Oslofjord \citep{roed:etal:2016}. 

The model applies several external inputs, such as atmospheric input, river input, tides, and input of sea level, currents and hydrography at the model's open lateral boundaries, in addition to bathymetry (Fig.~\ref{fig:oppbygging}). Mean values of sea level, currents, and hydrography from the NorKyst800 model \citep{albre:etal:2011} is applied on the open boundary towards Skagerak. The necessary atmospheric input is extracted from the AROME-MetCoOp model that runs operationally at MET Norway \citep{mulle:etal:2015}. The tidal input is based on the TPXO Atlantic database \citep{egber:erofe:2002} and modified using the measurements at Viker close to the southern boundary. The freshwater discharges from the rivers are based on the discharge data from a database constructed by use of the hydrological model HBV \citep{beldr:etal:2003}. For more details on the FjordOs model, see \cite{roed:etal:2016}.

The FjordOs model covers the area of interest (Fig.~\ref{fig:kart}). The model period in this study is April 2014 to December 2015.

\begin{figure}[htb]
\centerline{
\includegraphics*[width=0.9\textwidth]{Figurer/oppbygging}
}
\caption{\small
Illustration of external inputs to the FjordOs model.
}
\label{fig:oppbygging}
\end{figure}

\newpage
\section{Observed data}

The relevant observed data in the area of interest are scattered in both time and space (Fig.~\ref{fig:kart_obs}). Observations over longer time period includes three stations measuring sea level (Viker, Oscarsborg, and Oslo), one bottom-mounted doppler measuring currents in two depths level (Slagentangen), and one device measuring sea temperature (Scanmar AS). More occasional observations includes CTD measurements, current measurements, ferrybox, some drifter experiments, and sea temperature at beaches during the summer.  

\begin{figure}[htb]
\centerline{
\includegraphics*[trim=0cm 0.8cm 0cm 0cm,clip=true,width=0.8\textwidth]{Figurer/kart_obs}
}
\caption{\small
Positions for observations are marked.
}
\label{fig:kart_obs}
\end{figure}

\newpage
\subsection{Water level}
The Norwegian Mapping Authority has three permanent stations measuring sea level in the area of interest (Fig.~\ref{fig:kart_obs}). The station at Viker is placed close to the open boundary of the model area. The station at Oscarsborg is placed halfway in the Inner Oslofjord, and the station at Oslo is placed in the innermost part of the fjord.

\clearpage

\subsection{Currents}

\subsubsection{Currents in two cross sections}
In the period from Monday 15 to Thursday 18 of September 2014 altogether seven rigs with instruments was deployed at seven positions in Oslofjorden (Tab.~\ref{tab:Statnett}). All but one rig had profiling current meters, while the last rig had TinyTag temperature loggers deployed at seven different depths between 20 and 120 m. At one station a single point current meter was deployed just below the profiling current meter. 

The rigs were deployed as a part of a project conducted by Statnett, NIVA, Akvaplan NIVA, and the University of Oslo. The research vessel F/F Trygve Braarud was used during deployment and recovery of the rigs. Details on the applied rigs and corresponding instruments can be found in \cite{staalstrom:2015}.

\begin{table}[ht] 
\caption{Target positions (WGS84) of the instrument rigs. Depths at the stations are from the Statnett terrain model.} 
\label{tab:Statnett} 
\centering 
\begin{tabular}{|llcccl|} 
\hline  
{\bf Station} & {\bf Name} & {\bf Latitude} & {\bf Longitude} & {\bf Depth} & {\bf Instruments} \\ 
&& {\bf [$^o$N]} & {\bf [$^o$E]} & {\bf [m]} & \\ \hline
Kp11.2 & Sm\aa skj\ae r & 59.350124 & 10.497661 & 20 & Aquadopp600 AQP1531 \\
(Ri1) &&&&& Transducer LRT2 \\ \hline
Kp5.7 & Laksetrappa & 59.343452 & 10.581023 & 75 & Aquadopp400 AQP4689 \\
(Rl1) &&&&& Transducer LRT3 \\
      &&&&& Aanderaa Seaguard \\  \hline
Kp2.6 & Botnegrunnen & 59.352375 & 10.626822 & 96 & Continental WAV6117 \\
(Rm1) &&&&& Transducer LRT4 \\ \hline
Kp0.7 & Evje & 59.363182 & 10.653576 & 64 & Aquadopp400 AQP2931 \\
(Rn1) &&&&& Transducer LRT5 \\ \hline
Kn2 & Brenntangen & 59.581803 & 10.646087 & 54 & Aquadopp400 AQP5608 \\
      &&&&& Transducer LRT6 \\ \hline
Km1 & Filtvedt & 59.582064 & 10.627372 & 153 & Continental CNL6037 \\
      & (current) &&&& Transducer 207-2 \\ \hline
Km2 & Filtvedt & 59.580778 & 10.626239 & 125 & 7 TinyTags UIO1-7 \\
      & (temperature) &&&& Transducer 203-2 \\
\hline
\end{tabular}
\end{table}

\clearpage

\subsubsection{Currents at Slagentangen}
Using a bottom-mounted doppler Exxonmobil has measured the currents in two depths since 1997. The device is placed 50-80 meters northwest of Turning Dolphin at the Slagen Refinery (Fig.~\ref{fig:Slagen-kart}). Note that Bliksekilen nature reserve is located west of Slagen Refinery. It is a shallow water area with rare flora and fauna. 

\begin{figure}[htb]
\centerline{
\includegraphics*[trim=0cm 0cm 1cm 0cm,clip=true,width=0.8\textwidth]{Figurer/Slagen_kart}}
\caption{\small
Map of Slagen Refinery. The red dot marks the position corresponding to the extracted simulated data. Souce: the Norwegian Coastal Administration}
\label{fig:Slagen-kart}
\end{figure}

\subsubsection{Currents by Svelvik - Andr\'{e}}

\clearpage
\subsection{Hydrography}

\subsubsection{CTD measurements - Andr\'{e}}

\subsubsection{Water temperature near \AA sg\aa rdstrand}

Hourly temperature measurements at one meter depth over the last 10 years have been measured by Scanmar AS located three kilometres south of \AA sg\aa rdstrand. The device has an accuracy of $\pm 0.15^o$C in the range from -5 to +30$^o$C. Note that 2014 had a warmer summer followed by a warm winter resulting in higher maximum and lower minimum than 2015 (Fig.~\ref{fig:Scan_temp}). 
% B�r kanskje skrive noe om fluktuasjonene her...

\begin{figure}[htb]
\centerline{
\includegraphics*[trim=2cm 0cm 2cm 0cm,clip=true,width=\textwidth]{Figurer/temp_Scanmar}}
\caption{\small
Observed temperature measured by Scanmar AS}
\label{fig:Scan_temp}
\end{figure}

%\begin{table}[ht] 
%\caption{Observed water temperature near \AA sg\aa rdstrand} 
%\label{tab:Scan_temp} 
%\centering 
%\begin{tabular}{|cccccc|} 
%\hline  
%{\bf Year} & {\bf Minimum} & {\bf 5 percentile} & {\bf Mean} & {\bf 95 percentile} & {\bf Maximum} \\
%\hline
%\small 2005 & 1.0 & 4.0 & 13.2 & 20.7 & 24.9 \\
%\small 2006 & -6.0 & -0.5 & 10.1 & 20.6 & 23.9 \\
%\small 2007 & -1.2 & 1.1 & 9.7 & 18.3 & 21.4 \\
%\small 2008 & 0.1 & 2.6 & 10.4 & 19.3 & 24.6 \\
%\small 2009 & -2.3 & -0.3 & 9.4 & 19.8 & 24.9 \\
%\small 2010 & -1.7 & -0.2 & 8.7 & 18.7 & 20.5 \\
%\small 2011 & -1.4 & -1.0 & 9.6 & 19.0 & 22.6 \\
%\small 2012 & -1.0 & 0.4 & 9.2 & 18.5 & 21.4 \\
%\small 2013 & -1.3 & -0.4 & 9.3 & 19.4 & 22.2 \\
%\small 2014 & -1.1 & 1.2 & 10.6 & 21.7 & 26.4 \\
%\small 2015 & 0.7 & 3.5 & 10.5 & 18.7 & 20.8 \\
%\hline
%\end{tabular}
%\end{table}

\newpage
\subsubsection{Water temperature in the Inner Oslofjord}

Temperature measurements at three beaches in the Inner Oslofjord (Fig.~\ref{fig:kart_strand}) are performed in a cooperation between Asker and B\ae rum kommune, and Finnerud Elektronikk. The digital thermometers (Maxim Integrated DS18B20) have an accuracy of $\pm 0.5^o$C and are placed 40 cm beneath the water surface in positions where the water depths are several meters. Temperatures are measured every three hours from 09:00 to 18:00 during the summer months. 

The trends of the observed temperatures at the three beaches are similar, but the temperature is generally lower at the southern beach, Sj\o strand, than at the northern beach, Stor\o yodden (Fig.~\ref{fig:temp_strand}). In 2014 there were two local maximums during July, and the maximum observed temperatures in 2014 were higher than in both 2013 and 2015. The temperature increases 1-3 degrees during the day and decreases during the night.

\begin{figure}[ht]
\centerline{
\begin{minipage}[l]{0.59\textwidth}
\includegraphics*[trim=0 0 0 1cm,clip=true,width=\textwidth]{Figurer/badestrand_kart}
\end{minipage}
\begin{minipage}[r]{0.4\textwidth}
\fbox{\includegraphics*[trim=1 0 0 3cm,clip=true,width=\textwidth]{Figurer/kart_Storoyodden.png}} \\
\fbox{\includegraphics*[trim=0 0 0 3cm,clip=true,width=\textwidth]{Figurer/kart_Hvalstrand.png}} \\
\fbox{\includegraphics*[trim=1 1cm 0 1.5cm,clip=true,width=\textwidth]{Figurer/kart_Sjostrand.png}} \\
\end{minipage}
}
\caption{\small
The positions at three beaches in the Inner Oslofjord where the temperature measurements are performed}
\label{fig:kart_strand}
\end{figure}

\begin{figure}[ht]
\centerline{
\includegraphics*[trim=2cm 0 2cm 0cm,clip=true,width=\textwidth]{Figurer/badetemp}
}
\caption{\small
Observed temperature at three beaches in the Inner Oslofjord}
\label{fig:temp_strand}
\end{figure}

%\begin{table}[ht]
%\caption{Mean observed temperatures at three beaches in the Inner Oslofjord. Only time periods with more than 8 days of observations during the given time period are included.}
%\label{tab:temp_strand}
%\begin{center}
%\begin{tabular}{|l|ccc|ccc|ccc|} \hline
%     & \multicolumn{3}{c|}{Stor\o yodden} & \multicolumn{3}{c|}{Hvalstrand} & \multicolumn{3}{c|}{Sj\o strand} \\ \hline
%Time period & 2013 & 2014 & 2015 & 2013 & 2014 & 2015 & 2013 & 2014 & 2015 \\ \hline
%%01 - 15 May & 0 & 0 & 0 & 0 & 0 & 0 & 0 & 0 & 0 \\ 
%16 - 31 May & 13.7 &  -   & 11.6 &  -   &  -   & 12.2 &  -   &  -   & 12.8 \\ 
%01 - 15 Jun & 15.0 & 19.0 & 13.3 & 15.9 & 19.4 & 13.7 &  -   & 19.9 & 14.6 \\ 
%16 - 30 Jun & 16.2 & 17.3 & 17.1 & 16.7 & 17.7 & 17.5 &  -   & 17.9 & 18.0 \\ 
%01 - 15 Jul & 17.8 & 17.6 & 18.2 & 18.3 & 18.3 & 19.8 & 19.4 & 19.3 &  -   \\ 
%16 - 31 Jul & 20.1 & 22.2 & 18.1 & 19.7 &  -   & 18.9 & 21.2 & 23.9 & 18.4 \\ 
%01 - 15 Aug & 19.5 & 21.3 & 18.0 & 19.8 & 21.3 &  -   & 20.6 & 21.7 & 18.8 \\ 
%16 - 31 Aug & 18.9 & 19.6 & 19.1 & 18.9 & 19.6 & 19.3 & 19.5 & 19.4 & 19.0 \\ 
%01 - 15 Sep & 17.9 & 18.3 & 16.1 & 17.7 & 18.6 & 16.5 & 18.1 & 18.9 & 16.2 \\ 
%16 - 30 Sep &  -   & 15.9 & 14.1 &  -   & 16.2 & 14.1 & 15.1 & 16.3 & 13.7 \\ 
%01 - 15 Oct &  -   &  -   & 12.0 &  -   &  -   & 11.9 &  -   &  -   & 11.5 \\ 
%%16 - 31 Oct & 0 & 0 & 0 & 0 & 0 & 0 & 0 & 0 & 0 \\ 
%\hline
%\end{tabular}
%\end{center}
%\end{table}

\newpage
\subsubsection{Ferrybox - Andr\'{e}}

\subsection{Drifting lanes - Nils/Karina}
Godafoss ...
De to drifterne i f\o rste tokt ... og kanskje de siste i Svelvik.


\clearpage
\section{Evaluation}
\subsection{Water level and tide}

Time series from the three permanent stations measuring water level have been analysed and compared with simulated time series of water level extracted from locations near the three permanent stations. Both simulated and observed time series of water level are analysed using t\_tide \citep{pavlo:etal:2002} in order to extract the tidal components. The same period in time are applied for both the simulations and the observations (April 2014 to December 2015). 

\begin{figure}[hb] 
\centerline{ 
\includegraphics*[trim=3cm 0cm 2.5cm 0cm,clip=true,width=\textwidth]{Figurer/Oscarsborg_Tide_selected_jan15}  
} 
\centerline{ 
\includegraphics*[trim=3cm 0cm 2.5cm 0cm,clip=true,width=\textwidth]{Figurer/Oscarsborg_WL_rest_jan15} 
} 
\caption{\small 
Simulated (black) and observed (red) time series of tides (upper) and residual (lower) at Oscarsborg. Here the tidal elevation includes only the eleven components included in the tidal forcing. The residual includes the total water level minus the tidal elevation.} 
\label{fig:Waterlevel_jan15} 
\end{figure} 

Eleven tidal components are included in the model at the southern open boundary using their corresponding amplitudes and phases for both depth integrated currents and water level \citep{roed:etal:2016}. The time series of tidal components included in the tidal forcing are in fairly good agreement (Fig.~\ref{fig:Waterlevel_jan15},upper, and Tab.\-\ref{tab:Tide}). In addition to the tidal components included in the tidal forcing, more tidal components are present in the time series (Fig.~\ref{fig:Waterlevel_tide}). Tidal components with periods of approximately one year (SA) and half a year (SSA) respectively are present in both the observations and the simulations (Tab.\-\ref{tab:Tide}). In addition, the observations have more components with shorter periods which are not included in the tidal forcing and thereby not present in the simulations. This is consistent with the frequency series of the Fourier transformed water level (Fig.~\ref{fig:Waterlevel_freq}).

\begin{figure}[tbh] 
\centerline{ 
\includegraphics*[trim=3cm 0cm 2.5cm 0cm,clip=true,width=\textwidth]{Figurer/Oscarsborg_Tide_not_included}  
} 
\caption{\small 
Time series at Oscarsborg of the tidal components not included in the tidal tidal forcing.} 
\label{fig:Waterlevel_tide} 
\end{figure} 

\begin{figure}[tbh] 
\centerline{ 
\includegraphics*[trim=3cm 0cm 2.5cm 0cm,clip=true,width=\textwidth]{Figurer/Oscarsborg_Tide_Frequency_obs.png}  
} 
\centerline{ 
\includegraphics*[trim=3cm 0cm 2.5cm 0cm,clip=true,width=\textwidth]{Figurer/Oscarsborg_Tide_Frequency_sim.png} 
} 
\caption{\small 
Frequency series of Fourier transformed observed (upper) and simulated (lower) water levels at Oscarsborg} 
\label{fig:Waterlevel_freq} 
\end{figure} 

\newpage 
The amplitude of M$_2$ increases from south to north in the Inner Oslofjord both in the simulations and the observations (Fig.~\ref{fig:M2field} and Tab.~\ref{tab:Tide}). The lowest M$_2$ amplitude in the area of interest, is in the Drammensfjord north of the threshold in Svelvik. The M$_2$ phase has a sudden increase at the thresholds of Svelvik and Dr\o bak (Fig.~\ref{fig:M2field}). The same yields for the majority of the other relevant tidal components.

\begin{figure}[hb] 
\centerline{ 
\includegraphics*[trim=1cm 0cm 0cm 0cm,clip=true,width=0.49\textwidth]{Figurer/M2amp_felt}  
\includegraphics*[trim=0.8cm 0cm 0cm 0cm,clip=true,width=0.49\textwidth]{Figurer/M2fase_felt} 
} 
\caption{\small 
Simulated field of M$_2$ amplitude (left) and phase (right). The corresponding observed values for M$_2$ amplitude and phase are marked with circles at Viker, Oscarsborg, and Oslo.} 
\label{fig:M2field} 
\end{figure} 

\begin{table}[ht] 
%\vspace{-1.5cm} 
\caption{Simulated and observed tidal amplitude and phase for selected tidal components.} 
\label{tab:Tide} 
\centering 
\begin{tabular}{|c|c|l|cc|cc|cc|c|} 
\hline  
&&& \multicolumn{2}{|c|}{\bf Viker} & \multicolumn{2}{|c|}{\bf Oscarsborg} & \multicolumn{2}{|c|}{\bf Oslo} & {\bf Included} \\  
{\bf Comp.} & {\bf Period} &  {\bf sim/} & {\bf amp.} & {\bf phase.} & {\bf amp.} & {\bf phase.} & {\bf amp.} & {\bf phase.} & {\bf in tidal} \\ 
& {\bf [h]} & {\bf obs} & {\bf [cm]} & {\bf [deg]} & {\bf [cm]} & {\bf [deg]} & {\bf [cm]} & {\bf [deg]} & {\bf forcing} \\ \hline 
\small SA   & 8764	 	& sim & 15.5 & 284 & 15.6 & 286 & 15.4 & 286 & no   \\
\small      &        	& obs & 10.0 & 319 & 11 & 322 & 11.4 & 324 &    \\
\small SSA  & 4382 		& sim & 8.8 & 197 & 9.2 & 200 & 9.4 & 200 & no   \\
\small      &        	& obs & 7.5 & 188 & 8.0 & 189 & 8.2 & 190 &    \\
\small K2   & 11.9672 	& sim & 1.6 & 10 & 2.0 & 13 & 2.1 & 15 & yes  \\
\small      &        	& obs & 0.7 & 45 & 0.8 & 66 & 0.9 & 66 &    \\
\small S2   & 12.0000 	& sim & 3.3 & 64 & 3.9 & 69 & 4.2 & 70 & yes  \\
\small      &        	& obs & 2.9 & 46 & 3.3 & 65 & 3.5 & 69 &    \\
\small M2   & 12.4206 	& sim & 11.5 & 105 & 13.2 & 112 & 13.9 & 114 & yes  \\
\small      &        	& obs & 11.9 & 105 & 13.8 & 121 & 14.4 & 125 &    \\
\small N2   & 12.6584 	& sim & 3.0 & 69 & 3.5 & 75 & 3.7 & 76 & yes  \\
\small      &        	& obs & 3.0 & 60 & 3.4 & 76 & 3.6 & 80 &    \\
\small K1   & 23.9345 	& sim & 0.2 & 187 & 0.1 & 175 & 0.2 & 157 & yes  \\
\small      &        	& obs & 0.4 & 127 & 0.7 & 130 & 0.8 & 130 &    \\
\small P1   & 24.0659 	& sim & 0.6 & 322 & 0.6 & 334 & 0.7 & 342 & yes  \\
\small      &        	& obs & 0.2 & 129 & 0.3 & 102 & 0.4 & 97 &    \\
\small O1   & 25.8193 	& sim & 3.5 & 337 & 3.8 & 339 & 3.8 & 339 & yes  \\
\small      &        	& obs & 2.2 & 277 & 2.3 & 281 & 2.4 & 282 &    \\
\small Q1   & 26.8684 	& sim & 0.0 & 231 & 0.0 & 216 & 0.1 & 215 & no   \\
\small      &        	& obs & 1.1 & 190 & 1.2 & 198 & 1.3 & 200 &    \\
\small MN4  & 6.2692 	& sim & 0.2 & 5 & 0.5 & 32 & 0.6 & 35 & yes  \\
\small      &        	& obs & 0.4 & 249 & 0.6 & 289 & 0.7 & 297 &    \\
\small M4   & 6.2103 	& sim & 1.0 & 355 & 1.9 & 18 & 2.5 & 23 & yes  \\
\small      &        	& obs & 1.2 & 281 & 1.8 & 324 & 2.3 & 332 &    \\
\small MS4  & 6.1033 	& sim & 0.6 & 80 & 1.2 & 107 & 1.6 & 111 & yes  \\
\small      &        	& obs & 0.3 & 360 & 0.5 & 44 & 0.7 & 56 &    \\
\hline
\end{tabular}
\end{table}



\clearpage 
\subsection{Currents}

\subsubsection{Currents in two cross sections - Andr\'{e}/Karina}

At Filtvedt the observered and simulated currents are of the same magnitude in strength (Fig.~\ref{fig:Filtvedt-cur}). The observations are dominated by noise in the upper 35-40 meters. Only depths larger than 40 meters are therefore inlcuded. 





To be continued... Andr\'{e}?


\begin{figure}[ht]
\centerline{
\includegraphics*[trim=0 0 0 0,clip=true,width=\textwidth]{Figurer/Filtvedt_obs_cur}}
\centerline{
\includegraphics*[trim=0 0 0 0,clip=true,width=\textwidth]{Figurer/Filtvedt_sim_cur}}
\caption{\small
Observed (upper) and simulated (lower) currents at Filtvedt. Since the observations near the surface where dominated by noise, only depths larger than 40 meters are included in the upper plot. $z$ = 40 meters is marked with a black line in the lower plot. Note that the model depth is only 155 meters at the position the observations where performed.}
\label{fig:Filtvedt-cur}
\end{figure}

\begin{figure}[ht]
\centerline{
\includegraphics*[trim=0 0 0 0,clip=true,width=\textwidth]{Figurer/Filtvedt_obs_tide}}
\centerline{
\includegraphics*[trim=0 0 0 0,clip=true,width=\textwidth]{Figurer/Filtvedt_sim_tide}}
\caption{\small
Observed (upper) and simulated (lower) tidal currents at Filtvedt.}
\label{fig:Filtvedt-tide}
\end{figure}


\clearpage 
\subsubsection{Current at Slagentangen}
The observed currents at Slagentangen are compared with simulated data from 1st October 2014 until 30th November 2015 at approximately the same location and depth (Fig.~\ref{fig:Slagen-kart}).

%\begin{figure}[ht]
%\centerline{
%\includegraphics*[trim=1cm 0cm 1cm 0cm,clip=true,width=0.5\textwidth]{Figurer/Slagen_kart}}
%\caption{\small
%Map retrived from the Norwegian Coastal Administration. The red dot marks the position corresponding to the extracted simulated data.}
%\label{fig:Slagen-kart}
%\end{figure}

\begin{figure}[ht]
\centerline{
\includegraphics*[trim=3cm 0cm 3cm 0cm,clip=true,width=\textwidth]{Figurer/Slagen_tid}}
\caption{\small
Timeseries of observed and simulated velocity magnitudes at Slagen.}
\label{fig:Slagen-tid}
\end{figure}

\begin{figure}[ht]
\centerline{
\includegraphics*[trim=2cm 1cm 1cm 0cm,clip=true,height=4cm]{Figurer/Slagen_Rose_obs} 
\includegraphics*[trim=2cm 1cm 3cm 0cm,clip=true,height=4cm]{Figurer/Slagen_Rose_sim}
}
\caption{\small
Current roses for observed (left) and simulated (right) velocity magnitude at the two depths from 1st of October 2014 to 1st of October 2015.}
\label{fig:Slagen-rose}
\end{figure}

\begin{figure}[t]
\centerline{
\includegraphics*[trim=2cm 0cm 2cm 0cm,clip=true,width=\textwidth]{Figurer/Slagen_pdf} 
}
\caption{\small
Probability density functions of velocities and directions at Slagen for 1st of October 2014 to 1st of October 2015. The bin width is 0.01 knots for velocity and 3 degrees for direction.}
\label{fig:Slagen-pdf}
\end{figure}

\begin{table}[ht]
%\vspace{-1.5cm}
\caption{Yearly maximum observed velocity at Slagen.}
\label{tab:Slagen_max}
\centering
\begin{tabular}{|l|lll|lll|}
\hline 
& \multicolumn{3}{|l|}{\bf Max. velocity at 10m depth} & \multicolumn{3}{|l|}{\bf Max. velocity at 2.5m depth} \\
{\bf Year} & {\bf Date} & {\bf [m/s} & {\bf [deg]} & {\bf Date} & {\bf [m/s]} & {\bf [deg]} \\ \hline 
\small 2006 & 21 Jan 2006 & 0.42 & 139 & 31 Oct 2006 & 0.57 & 140 \\
\small 2007 & 14 Jan 2007 & 0.42 & 172 & 21 Aug 2007 & 1.03 & 359 \\
\small 2008 & 22 Mar 2008 & 0.36 & 149 & 19 Dec 2008 & 0.57 & 160 \\
\small 2009 & 17 Dec 2009 & 0.45 & 142 & 24 Mar 2009 & 0.56 & 139 \\
\small 2010 & 09 Nov 2010 & 0.41 & 138 & 09 Nov 2010 & 0.54 & 138 \\
\small 2011 & 01 Jan 2011 & 0.39 & 146 & 30 Mar 2011 & 0.62 & 185 \\
\small 2012 & 05 Dec 2012 & 0.39 & 138 & 29 May 2012 & 0.57 & 140 \\
\small 2013 & 10 Oct 2013 & 0.42 & 143 & 10 Oct 2013 & 0.49 & 144 \\
\small 2014 & 18 Apr 2014 & 0.44 & 147 & 26 Mar 2014 & 0.55 & 143 \\
\small 2015 & 24 Jan 2015 & 0.33 & 128 & 21 Mar 2015 & 0.55 & 141 \\
\hline
\end{tabular}
\end{table}

\begin{figure}[ht]
\centerline{
\includegraphics*[trim=0cm 0cm 0cm 0cm,clip=true,width=0.8\textwidth]{Figurer/Slagen_QQ}}
\caption{\small
Combined QQ- and scatter plot of observed and simulated current at Slagen from 1st of October 2014 to 1st of October 2015.}
\label{fig:Slagen_QQ}
\end{figure}

Time series reveal that the observed velocities varies and follows no striking pattern (Fig.~\ref{fig:Slagen-tid}).
Current roses show that both the observed and the simulated velocities are stronger in the upper layer (Fig.~\ref{fig:Slagen-rose}). The simulated velocities are stronger than the observed velocities. This is in accordance with the probability density functions (Fig.~\ref{fig:Slagen-pdf}). The yearly maximum observed velocities are approximately 0.4 and 0.6 m/s at 10 and 2.5 meters depth respectively (Tab.~\ref{tab:Slagen_max}). During 2014 and 2015 maximum observed velocity at 2.5 meters depth was 0.55 m/s in southeast direction (143$^o$N) the 26th of March 2014. The velocity at 10 meters depth was 0.08 m/s (153$^o$N) at the time of maximum velocity at 2.5 meters depth indicating that the velocities are different in the two layers.

The mean directions are to the south east. At approximately 2.5 meters depth the mean directions are 146$^o$N and 139$^o$N for observed and simulated directions respectively which is in fairly good agreement. At approximately 10 meters depth the observed mean direction shifts to 170$^o$N while the simulated mean direction is 148$^o$N. 
Testing with popcorn indicate that the preferred direction of the surface currents are towards Bliksekilen located west of the Slagen Refinery. This is not the case neither in the observations nor the simulations.
The probability density functions reveals that the model captures the distribution of directions in the upper layer, but does not capture the change in direction between the two depths (Fig.~\ref{fig:Slagen-pdf}). The standard deviations at 2.5 and 10 meters are 55 and 66 degrees respectively for the observed directions, and 56 and 61 for the simulated directions.

The time series scatter plots reveal that the correlation in time is not satisfying (Fig.~\ref{fig:Slagen-tid} and \ref{fig:Slagen_QQ}). The model seem to have difficulties with capturing the right phenomena influencing the currents to the right time. This is a well known problem when it comes to forecasting currents. The QQ-plots also confirms that the simulated currents are stronger than the observed currents. 

\subsubsection{Currents by Svelvik - Andr\'{e}}

\clearpage
\subsection{Hydrography}

\subsubsection{CTD-measurements - Andr\'{e}}
To be continued... Andr\'{e}

\clearpage 

\subsubsection{Water temperature near \AA sg\aa rdstrand}

The temperature observations from Scanmar AS are compared with simulated data extracted from 1.15 meters depth at approximately the same location as the observations.

Time series reveal that the simulated and observed temperature are in fairly good agreement (Fig.~\ref{fig:temp_2015}). During winter and spring, but the model underestimate the temperature in the summer and fall with a few degrees  and \ref{fig:temp-QQ_scatter}). The model captures the timing of the daily variations in temperature, but seems to overestimate heating and cooling causing too large daily variations (Fig.~\ref{fig:temp_jun2015}). 

2014 had a warmer summer than 2015. This is evident in both the observations and the simulations (Tab.~\ref{tab:temp}). The summer months of 2014 also had the largest variance in both the observed and the simulated temperature during 2014 and 2015. Generally, the simulated monthly temperature had a larger variance than the observed monthly temperature. The mean of the observed and simulated temperatures are 10.0$^o$C and 8.6$^o$C respectively, while the variances are 27.2$^o$C and 21.1$^o$C respectively.  

\begin{figure}[ht]
%\centerline{
%\includegraphics*[trim=0cm 0cm 0cm 0cm,clip=true,width=\textwidth]{Figurer/Temperatur_2014}}
\centerline{
\includegraphics*[trim=0cm 0cm 0cm 0cm,clip=true,width=\textwidth]{Figurer/Temperatur_2015}}
\caption{\small
Time series of observed and simulated temperature at \AA sg\aa dstrand. The difference is smoothed over 10 days.}
\label{fig:temp_2015}
\end{figure}

\begin{figure}[htb]
\centerline{
\includegraphics*[trim=1cm 0cm 1cm 0cm,clip=true,width=0.7\textwidth]{Figurer/Temperatur_QQ_scatter}}
\caption{\small
Combined QQ- and scatter plot of observed and simulated temperature at \AA sg\aa dstrand.}
\label{fig:temp-QQ_scatter}
\end{figure}

\begin{figure}[htb]
\centerline{
\includegraphics*[trim=0cm 0cm 0cm 0cm,clip=true,width=\textwidth]{Figurer/Temperatur_jun2015}}
\caption{\small
Time series of observed and simulated temperature at \AA sg\aa dstrand in June 2015.}
\label{fig:temp_jun2015}
\end{figure}

\newpage 

\begin{table}
\caption{Monthly statistics for observed and simulated temperature at \AA sg\aa rdstrand.}
\label{tab:temp}
\centering
\begin{tabular}{|ll|rrr|rrr|}
\hline 
&& \multicolumn{3}{|c|}{\bf 2014} & \multicolumn{3}{|c|}{\bf 2015} \\
&& {\bf quantity} & {\bf mean} & {\bf variance}  
& {\bf quantity} & {\bf mean} & {\bf variance} \\ \hline 
\small Jan & obs & 558 & 2.5 & 6.2 & 558 & 4.9 & 2.5 \\
\small     & sim & 745 & - & - & 745 & 4 & 0.7 \\
\small Feb & obs & 504 & 1.8 & 0.5 & 504 & 4 & 1.8 \\
\small     & sim & 673 & - & - & 673 & 3.6 & 0.6 \\
\small Mar & obs & 496 & 3.7 & 0.6 & 496 & 4.2 & 0.4 \\
\small     & sim & 745 & - & - & 745 & 4.7 & 0.3 \\
\small Apr & obs & 515 & 7.5 & 4.1 & 515 & 7 & 2.1 \\
\small     & sim & 721 & 7.6 & 7.1 & 721 & 7.4 & 2.8 \\
\small May & obs & 544 & 12.1 & 9.3 & 544 & 10.3 & 1.3 \\
\small     & sim & 745 & 11.3 & 10.4 & 745 & 9.9 & 3.5 \\
\small Jun & obs & 531 & 16.2 & 4.2 & 531 & 14 & 3.9 \\
\small     & sim & 721 & 15.5 & 8.3 & 721 & 12.8 & 6 \\
\small Jul & obs & 558 & 20.2 & 10.5 & 558 & 17.1 & 2 \\
\small     & sim & 745 & 17 & 24.3 & 745 & 14.6 & 5.1 \\
\small Aug & obs & 512 & 20 & 3.2 & 512 & 18.2 & 1.3 \\
\small     & sim & 745 & 16.4 & 5.6 & 745 & 15.5 & 8.7 \\
\small Sep & obs & 536 & 16.3 & 1.9 & 536 & 15.1 & 1.2 \\
\small     & sim & 721 & 12.9 & 8.9 & 721 & 12.7 & 2.5 \\
\small Oct & obs & 557 & 12.3 & 1.7 & 557 & 10.6 & 0.8 \\
\small     & sim & 745 & 9 & 1.5 & 745 & 8.6 & 2.7 \\
\small Nov & obs & 540 & 8.3 & 2.9 & 540 & 8.8 & 2.4 \\
\small     & sim & 721 & 6.3 & 2 & 721 & 6.1 & 1.3 \\
\small Dec & obs & 490 & 4.4 & 3.6 & 490 & 7.8 & 0.4 \\
\small     & sim & 733 & 3.4 & 1.5 & 733 & 4.8 & 0.8 \\
\hline
\end{tabular}
\end{table}

\clearpage

\subsubsection{Water temperature in the Inner Oslofjord}

The observed and simulated temperature at three beaches in the Inner Oslofjord are in relatively good agreement (Fig.~\ref{fig:badetemp_2014} - \ref{fig:badetemp_2015}). 

Close to the shoreline and only 40 cm under the surface, the temperature is heavily influenced by the weather situation and local circulation patterns. 

The temperature differences during the day are larger in the model than in the observations. The observed temperature increases 1-3 degrees from 09:00 to 18:00 and is not measured during the night, while the modelled temperature increases up to six degrees from 06:00 to 23:00. The fact that temperature is not measured during the night, but only from 09:00 to 18:00, might explain differences i temperature rise during the day, but the difference might indicate too much heating in the model.

%Since the temperature is observed close to the shoreline, some near river outlets, and only 40 cm under the surface, the temperature ise heavily influenced by the weather situation and local circulation patterns. The model is not expected to capture such detailed effects. Still there are similarities between the modelled and the observed temperatures both in temperature level and in fluctuations. 

During the summer 2014 the model predicts higher temperatures at Sj\o strand than was observed. The observations in Hvalstrand have some of the same trends as the modelled temperature with temperatures up to 25 degrees. The air temperatures in 2014 was higher than in 2015 and resulted in higher water temperatures, especially in shallow areas. 

\begin{figure}[ht]
\centerline{
\includegraphics*[trim=0 0 0 0,clip=true,width=\textwidth]{Figurer/badetemp_2014}
}
\caption{\small
The observed and modelled temperature at three beaches in the Inner Oslofjord during the summer 2014}
\label{fig:badetemp_2014}
\end{figure}

\begin{figure}[ht]
\centerline{
\includegraphics*[trim=0 0 0 0,clip=true,width=0.8\textwidth]{Figurer/badetemp_2015}
}
\caption{\small
The observed and modelled temperature at three beaches in the Inner Oslofjord during the summer 2015}
\label{fig:badetemp_2015}
\end{figure}


\subsubsection{Ferrybox - Andr\'{e}?}
To be continued... Andr\'{e}?

\begin{figure}[ht]
\centerline{
\includegraphics*[trim=1cm 0cm 1cm 0cm,clip=true,height=10cm]{Figurer/FjordOs_with_FA_track}}
\caption{\small
The track of Color Fantasy.}
\label{fig:Ferrybox_track}
\end{figure}

\begin{figure}[ht]
\centerline{
\includegraphics*[trim=1cm 0cm 1cm 0cm,clip=true,width=.5\textwidth]{Figurer/FjordOs_vs_Ferrybox_TEMP}
\includegraphics*[trim=1cm 0cm 1cm 0cm,clip=true,width=.5\textwidth]{Figurer/FjordOs_vs_Ferrybox_SALT}}
\caption{\small
Simulated daily mean compared with observations from ferryboxes for temperature (left) and salinity (right).}
\label{fig:Ferrybox_temp_salt}
\end{figure}


\clearpage 

\subsection{Drifting lanes - Karina/Nils}

De to drifterne i f\o rste tokt

Oppsummer v\ae rforhold og oljedriften etter Godefoss. Sammenlign med lignende tilfelle i 2014/15, samt alternative drivbaner under andre forhold.

Utslipp fra Slangentangen havner alltid i Bliksekilen if\o lge erfaring (Exxonmobil). Vi skal gj\o re noen beregninger om hvor utslipp fra Slagentangen havner if\o lge modellen. 

\clearpage 
\paragraph{Godafoss}

\begin{figure}[ht]
\centerline{
\includegraphics*[width=.5\textwidth]{Figurer/opendrift/opendrift_godafoss_shortest_time_crop}
\includegraphics*[width=.5\textwidth]{Figurer/opendrift/opendrift_godafoss_consentration_crop}
}
\centerline{
\includegraphics*[width=.5\textwidth]{Figurer/opendrift/opendrift_godafoss_shortest_time_zoom_crop}
\includegraphics*[width=.5\textwidth]{Figurer/opendrift/opendrift_godafoss_consentration_zoom_crop}
}
\caption{\small
Number of hours from particle release, to particle in given area (left panels), and the number of particles that has been inside a given 140x140m area (right panel). Godafoss. Based on one year (April 1st 2015 - April 1st 2016) of simulations, with a maximum lifetime of 15 days og the released particles. This amounts to a total number of 8760 released particles. Please note the different scales of each figure.}
\label{fig:opendrift_godafoss1}
\end{figure}

\begin{figure}[ht]
\centerline{
\includegraphics*[width=.5\textwidth]{Figurer/opendrift/opendrift_godafoss_shortest_time_zoom_endpos_crop}
\includegraphics*[width=.5\textwidth]{Figurer/opendrift/opendrift_godafoss_consentration_zoom_endpos_crop}
}
\caption{\small
For endposition of each trajectory: Number of hours from particle release, to particle in given area (left panels), and the number of particles that has been inside a given 140x140m area (right panel). Godafoss. Based on one year (April 1st 2015 - April 1st 2016) of simulations, with a maximum lifetime of 15 days og the released particles. This amounts to a total number of 8760 released particles. Please note the different scales of each figure.}
\label{fig:opendrift_godafoss2}
\end{figure}

\clearpage 
\paragraph{Slagen}

\begin{figure}[ht]
\centerline{
\includegraphics*[width=.5\textwidth]{Figurer/opendrift/opendrift_slagen_shortest_time_crop}
\includegraphics*[width=.5\textwidth]{Figurer/opendrift/opendrift_slagen_consentration_crop}
}
\centerline{
\includegraphics*[width=.5\textwidth]{Figurer/opendrift/opendrift_slagen_shortest_time_zoom_crop}
\includegraphics*[width=.5\textwidth]{Figurer/opendrift/opendrift_slagen_consentration_zoom_crop}
}
\caption{\small
Number of hours from particle release, to particle in given area (left panels), and the number of particles that has been inside a given 140x140m area (right panel). Slagentangen. Based on one year (April 1st 2015 - April 1st 2016) of simulations, with a maximum lifetime of 15 days og the released particles. This amounts to a total number of 8760 released particles. Please note the different scales of each figure.}
\label{fig:opendrift_slagen1}
\end{figure}

\begin{figure}[ht]
\centerline{
\includegraphics*[width=.5\textwidth]{Figurer/opendrift/opendrift_slagen_shortest_time_zoom_endpos_crop}
\includegraphics*[width=.5\textwidth]{Figurer/opendrift/opendrift_slagen_consentration_zoom_endpos_crop}
}
\caption{\small
For endposition of each trajectory: Number of hours from particle release, to particle in given area (left panels), and the number of particles that has been inside a given 140x140m area (right panel). Slagentangen. Based on one year (April 1st 2015 - April 1st 2016) of simulations, with a maximum lifetime of 15 days og the released particles. This amounts to a total number of 8760 released particles. Please note the different scales of each figure.}
\label{fig:opendrift_slagen2}
\end{figure}


\clearpage 

\section{Summary and final remarks}



\clearpage
\section*{\hspace{17mm}Acknowledgements}
\addcontentsline{toc}{section}{Acknowledgements}

\clearpage
\section*{\hspace{17mm}Appendix}
\addcontentsline{toc}{section}{Appendix}




%\begin{table}[h]
%\vspace{-1.5cm}
%{\bf Table A1: ESD statistics.}\\
%\label{tab:1}
%\begin{tabular}{llll}
%\small OSLO - BLINDERN 18700 & 
%\small $T_m$ $R^2$= 71 -- 86 $T_x$ $R^2$= 69 -- 82 $T_n$ $R^2$= 49 -- 76 \\
%\small $\Delta T_m$ $q_{0.05}$= 3.21 $q_{0.50}$= 1.33 $q_{0.95}$= 1.75 \\  
%\small $\Delta T_x$ $q_{0.05}$= 3.09 $q_{0.50}$= 1.38 $q_{0.95}$= 2.16 \\  
%\small $\Delta T_n$ $q_{0.05}$= 3.15 $q_{0.50}$= 1.12 $q_{0.95}$= 1.01 \\
%\end{tabular}
%\end{table}

\clearpage
\pagebreak

\bibliography{referanse}
\addcontentsline{toc}{section}{References}
%\begin{thebibliography}{1}
%\end{thebibliography}

\clearpage
\pagebreak
 

\end{document}


