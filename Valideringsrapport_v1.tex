\documentclass[12pt,a4paper,english]{article}
\usepackage[utf8]{inputenc}
\usepackage[english]{babel}
\usepackage{graphicx}
\usepackage{epsfig}            % To allow figures
\usepackage{pstricks}          % To draw color pictures directly
\usepackage{fancyhdr}
\usepackage[verbose,a4paper,tmargin=30mm,bmargin=37.5mm,lmargin=30mm,rmargin=30mm]{geometry}
\usepackage{helvet}
\usepackage{mathptmx}
\usepackage[T1]{fontenc}
\usepackage{rotating}
\usepackage{amssymb}       % More special characters          
\usepackage{amsmath}       % More mathematical characters
\usepackage{natbib} % Needed e.g. for \citep
\usepackage{multirow}
\usepackage{verbatim}          % Use the verbatim styles
\usepackage{url}
\definecolor{METblue}{cmyk}{0.85,0,0.2,0.2}
\usepackage{titlesec}
\titleformat{\section}
  [block]
  { \normalfont\sffamily\Large\bfseries\color{METblue} }
  {\makebox[2em][r]{\thesection}}
  {5mm}
  {\vspace{5mm}}

\titleformat{\subsection}[block]%
  {\normalfont\sffamily\bfseries}%
  {\makebox[2em][r]{\thesubsection}}%
  {5mm}
  {\vspace{3mm}}[]
\titleformat{\subsubsection}[block]%
  {\normalfont\sffamily\bfseries}%
  {\makebox[3em][r]{\thesubsubsection}}%
  {5mm}
  {\vspace{3mm}}[]

% Titlespacing syntax: 
\titlespacing*{\section}{-17mm}{5mm}{0mm}
\titlespacing*{\subsection}{-13.5mm}{5mm}{0mm}
\titlespacing*{\subsubsection}{-17.5mm}{5mm}{0mm}

%%% Formatting the table of contents
\usepackage{tocloft}
\renewcommand{\cfttoctitlefont}{\sffamily\bfseries\color{METblue}\Large}
\providecommand{\cftchapfont}{\sffamily\bfseries }
\renewcommand{\cftsecfont}{\sffamily\bfseries }
\renewcommand{\cftsubsecfont}{\sffamily }
\renewcommand{\cftsubsubsecfont}{\sffamily }
\providecommand{\cftsubsubsubsecfont}{\sffamily }
\renewcommand{\cftfigfont}{Figure }
\renewcommand{\cfttabfont}{Table }
\providecommand{\cftchappagefont}{\sffamily}
\renewcommand{\cftsecpagefont}{\sffamily\bfseries}
\renewcommand{\cftsubsecpagefont}{\sffamily}
\renewcommand{\cftsubsubsecpagefont}{\sffamily}
\providecommand{\cftsubsubsubsecpagefont}{\sffamily}

\renewcommand{\baselinestretch}{1.33}

%------------------------- LPR definitions -------------------------
% Definitions:
% 1. Often repeated stuff
\def\PDE{partial differential equation}
\def\wrt{with respect to}
\def\hhv{respectively}
\def\fe{for instance}
\def\dvs{that is}
\def\metno{Norwegian Meteorological Institute}
\def\swe{shallow water equations}

% 2. Mathematical stuff: 
% 2.1 Equations:
\def\be{\begin{equation}}
\def\ee{\end{equation}}
\def\beq{\begin{eqnarray}}
\def\eeq{\end{eqnarray}}
\def\bfig{\begin{figure}}
\def\efig{\end{figure}}
\DeclareMathOperator{\sgn}{sgn}
% 2.2 Differentiation:
\def\pt{\partial_t}
\def\px{\partial_x}
\def\py{\partial_y}
\def\pz{\partial_z}
\def\ptt{\partial_t^2}
\def\pxx{\partial_x^2}
\def\pyy{\partial_y^2}
\def\pzz{\partial_z^2}
\def\pts{\partial_{t'}}
\def\pxs{\partial_{x'}}
\def\pys{\partial_{y'}}
\def\pzs{\partial_{s}}
\def\dels{\nabla_{s}}
% 2.3 Numerical time step and grid steps:
\def\Dx{\Delta x}
\def\Dy{\Delta y}
\def\Dz{\Delta z}
\def\Dt{\Delta t}
\def\Ds{\Delta s}
% 2.4 Vectors:
\def\vec{\mathbf}
\def\bu{\mathbf{u}}
\def\bv{\mathbf{v}}
\def\bi{\mathbf{i}}
\def\bj{\mathbf{j}}
\def\bk{\mathbf{k}}
\def\bU{\mathbf{U}}
\def\bV{\mathbf{V}}
\def\ud{\textrm{d}}
% 2.5 Convenient abbrivations
\def\th{\theta}
\def\bhu{\vec{\bar{u}}}
\def\hu{\bar{u}}
\def\hv{\bar{v}}
\def\hp{\bar{\phi}}
\def\ha{\hat{a}}
\def\hb{\hat{b}}
\def\hc{\hat{c}}
\def\hC{\bar{C}}
\def\hd{\hat{d}}
\def\he{\hat{e}}
\def\hf{\hat{f}}
\def\hg{\hat{g}}
% 8. Sigma coordinate derivatives
\def\ptg{\partial_{t'}}
\def\pxg{\partial_{x'}}
\def\pyg{\partial_{y'}}
\def\pzg{\partial_{\sigma}}
\def\delg{\nabla_{\sigma}}
% 2.6 Operators
\def\del{\nabla_H}
\def\dels{\nabla_s}
\def\eps{\epsilon}
\def\vphi{\varphi}
\def\matO{\mathcal{O}}
\def\matG{\mathcal{G}}
\def\matH{\mathcal{H}}
% 2.7 Tensors
\def\mat{\boldsymbol}
\def\matI{\boldsymbol{\mathcal{I}}}
\def\matA{\boldsymbol{\mathcal{A}}}
\def\matB{\boldsymbol{\mathcal{B}}}
\def\matC{\boldsymbol{\mathcal{C}}}
\def\matD{\boldsymbol{\mathcal{D}}}
\def\matM{\boldsymbol{\mathcal{M}}}
\def\matE{\boldsymbol{\mathcal{E}}}
\def\matF{\boldsymbol{\mathcal{F}}}
\def\matL{\boldsymbol{\mathcal{L}}}
\def\matU{\boldsymbol{\mathcal{U}}}
\def\matV{\boldsymbol{\mathcal{V}}}
\def\matK{\boldsymbol{\mathcal{K}}}
\def\matP{\boldsymbol{\mathcal{P}}}
\def\matR{\boldsymbol{\mathcal{R}}}

% End definitions
%%%%%%%%%%%%%%%%%%%%%%%%%%%%%%%%%%%%%%%%%%%%%%%%%%%%%%%%%%%%%%%%%%%%%%%
% --------------------------------------

\begin{document}

\bibliographystyle{agufull04}
%\bibliographystyle{ams}

\pagenumbering{roman} % Use roman numerals for page numbering before main content. OL: The new template begins with page 1 on the cover.
\thispagestyle{empty}  % Hide page numbers

\noindent
\begin{tabular}{@{} p{63mm} p{50mm} r}
\includegraphics*[]{met_rapport_logo_eng} % Automatically uses PDF or EPS in same directory depending on latex or pdflatex.
&
\fontsize{27.5pt}{33pt} \selectfont \bf \sffamily MET{\color{gray} report}
&
 \begin{minipage}[b]{28mm}
  \begin{flushright}
   \footnotesize \sffamily No. X/2016 \\ ISSN 2387-4201 \\ Oceanography              % Report number and Category
  \end{flushright}
 \end{minipage}
\end{tabular}

\vfill

\begin{flushright}
{\fontsize{30pt}{36pt}\selectfont \bf \sffamily Evaluation of the FjordOs-model}          % Title
 
\vspace{5mm}
{\fontsize{12.5pt}{15pt}\selectfont \sffamily July 2016                                          % Subtitle
\\
\sffamily Karina Hjelmervik$^1$, Nils M. Kristensen$^3$, Lars P. R{\o}ed$^{3,4}$, Andr{\'e} Staalstr{\o}m$^2$% Author name(s)
}
\end{flushright}

%\vspace{25mm}
\vspace{2mm}


\begin{figure}[!h]
%\includegraphics*[height=5.5cm,angle=7]{Figurer/Driftere_ombord} \\
\vspace{3.5cm}
\begin{center}
\rput[bl](-7.5,-0.2){\includegraphics[height=1.7cm]{Figurer/logo_hsn}} 
\rput[b](0,0.1){\includegraphics[height=1.2cm]{Figurer/logo_met}} 
\rput[br](7.5,0){\includegraphics[height=1.2cm]{Figurer/logo_niva}} 
\end{center}
\end{figure}
\vspace{-1cm}
\noindent$^1$University College of Southeast Norway, 
$^2$Norwegian Institute for Water Research, \\ 
$^3$Norwegian Meteorological Institute,
$^4$Department of Geosciences, University of Oslo
\\




%\newpage

%\thispagestyle{empty}  % Hide page numbers

\clearpage

\setlength{\unitlength}{1mm}  %Needed for picture environment

\begin{table}[!ht]

\begin{tabular}[c]{lr}
\vspace{5mm}
\includegraphics*{met_rapport_logo_eng} & \hspace{43mm}
{\fontsize{27.5pt}{33pt}\selectfont \bf \sffamily MET{\color{gray} report}}\\
\end{tabular}

\sffamily{
\begin{tabular}[t]{|p{110mm}|p{40mm}|} \hline
{\bf \sffamily Title}                  & {\bf \sffamily Date}               \\ 
Evaluation of the FjordOs-model
                             & \today                   \\ \hline
{\bf \sffamily Section}                & {\bf \sffamily Report no.}         \\ 
 Ocean and Ice                        &  X/2015                  \\ \hline
{\bf \sffamily Author(s)}                 & {\bf \sffamily Classification}     \\ 
Karina Hjelmervik, Nils Melsom Kristensen, Lars Petter R{\o}ed, Andr\'{e} Staalstr{\o}m                 
                             & \begin{picture}(20,4)(-2,-1.0)
                               \put (0,0){\circle*{4}}
                               \put (7,0){\makebox(0,0){Free}}
                               \put (15,0){\circle{4}}
                               \put (27,0){\makebox(0,0){Restricted}}
                               \end{picture}
                               \\ \hline
{\bf \sffamily Client(s)}              & {\bf \sffamily Client's reference} \\ 
Oslofjordfondet                  &               \\ \hline
\end{tabular}

\begin{tabular}[t]{|p{154.3mm}|}
{\bf \sffamily Abstract}                                          \\
%Provided is an evaluation of the capabilities of the newly developed ocean model for the Oslofjord. The model fields has been compared to observations of water level, salinity, temperature, currents and trajectories of drifters. Overall, the model provides a good agreement with the observations, but there are still room for improvements.
%\\
\small{Provided is an evaluation on the performance of the FjordOs model, a new circulation model covering the Oslofjord, Norway. The model is developed to improve the ocean input (e.g., currents) to emergency models used to predict pathways of oil and/or other effluents. The FjordOs model is a regional adaption of the Regional Ocean Modeling System (ROMS), and makes use of the model's curvilinear option to increase the resolution without inflating the computer demand significantly. To assess the model's rendition of the circulation we compare results from a simulation near two years long to available observations. The observations encompass water level, currents and temperature at various time periods at fixed stations, and observed trajectories of drifters. The evaluation reveals that the model is not perfect, but nevertheless we argue that its performance is adequate for its purpose. An important justification is that the higher resolution offers a decrease in the number of stranded trajectories compared to models of coarser resolution. Also of importance is that the model provides a realistic depth profile of the currents, and the tidal elevation.} 
%\\[50mm] % Add whitespace if necessary
\\ \hline
{\bf \sffamily Keywords}                                          \\ 
  Ocean model, Oslofjord, ROMS, Validation, Fjordos    \\ 
\hline
\end{tabular}
}

\begin{tabular}[t]{cc}
                             &                            \\
                             &                            \\
                             &                            \\
\line(1,0){70}               & \line(1,0){70}             \\ 
Disciplinary signature       & Responsible signature      \\
Kai H. Christensen           & Lars Anders Breivik        \\       % Add names if needed
\hspace{75mm}                & \hspace{75mm}              \\

\end{tabular}
\end{table}

\clearpage

\thispagestyle{fancy} % footer from fancyhdr package
\headheight=15pt
\renewcommand{\headrulewidth}{0pt}

%\clearpage
\section*{\hspace{17mm}Abstract}
Provided is an evaluation on the performance of the FjordOs model, a new circulation model covering the Oslofjord, Norway. The model is developed to improve the ocean input (e.g., currents) to emergency models used to predict pathways of oil and/or other effluents. The FjordOs model is a regional adaption of the Regional Ocean Modeling System (ROMS), and makes use of the model's curvilinear option to increase the resolution without inflating the computer demand significantly. To assess the model's rendition of the circulation we compare results from a simulation near two years long to available observations. The observations encompass water level, currents and temperature at various time periods at fixed stations, and observed trajectories of drifters. The evaluation reveals that the model is not perfect, but nevertheless we argue that its performance is adequate for its purpose. An important justification is that the higher resolution offers a decrease in the number of stranded trajectories compared to models of coarser resolution. Also of importance is that the model provides a realistic depth profile of the currents, and the tidal elevation. 

%\clearpage

\vfill

\fancyfoot{
% If abstract on separate page is not needed, move the following table to the page before
\begin{tabular}[b]{p{40mm}p{25mm}p{25mm}p{25mm}p{25mm}}
 \begin{minipage}[l]{40mm} \tiny \color{METblue} {\bf Norwegian Meteorological Institute}\\ Org.no 971274042\\ post@met.no\\ www.met.no / www.yr.no
 \end{minipage} & 
 \begin{minipage}[l]{25mm} \tiny \color{METblue} {\bf Oslo}\\ P.O. Box 43, Blindern\\ 0313 Oslo, Norway\\ T. +47 22 96 30 00
 \end{minipage} &
 \begin{minipage}[l]{25mm} \tiny \color{METblue} {\bf Bergen}\\ All\'egaten 70\\ 5007 Bergen, Norway\\ T. +47 55 23 66 00
 \end{minipage} & 
 \begin{minipage}[l]{25mm} \tiny \color{METblue} {\bf Troms\o}\\ P.O. Box 6314, Langnes\\ 9293 Troms\o, Norway\\ T. +47 77 62 13 00
 \end{minipage} & 
 \begin{minipage}[l]{25mm} \tiny \color{METblue} 
 \end{minipage}
\end{tabular}
}

\clearpage
\tableofcontents

\clearpage
\pagestyle{plain}
\pagenumbering{arabic}


\section{Introduction}

We assess the performance of a new regional circulation model developed specifically for the Oslofjord, Norway. The model is named FjordOs and is a version of the Regional Ocean Model System (ROMS) adapted for the fjord utilizing its curvilinear option. For details on the FjordOs model, the reader is referred to \cite{roed:etal:2016}. 

The motivation is to construct a model with high enough resolution to properly resolve the fjord's many small islands, narrow straits and sounds, and to resolve its highly irregular coastline and topography \ref{fig:kart}. The hypothesis is that such a resolution is necessary to avoid effluents to be stranded artificially when simulating their pathways from the source. As is well known currents, besides wind and waves, is one of the dominant sources when predicting pathways of effluents like oil and/or discharges of other contaminants. Thus, the FjordOs model is designed to deliver simulation and/or forecasts of water level, current, temperature and salinity accurate enough to be a useful input to drift models.

The study is a part of the FjordOs project. FjordOs is a cooperation between MET Norway, University College of Southeast Norway (HSN), The Norwegian Institute for Water Research (NIVA), The Norwegian Coastal Administration (Kystverket), Exxonmobil, Norwegian Defence Research Establishment (FFI), Vestfold, Buskerud, and {\O}stfold county, and AGNES AB Milj{\o}konsulent.

The evaluation is based on available observations for the two year period 2014 and 2015 for which simulations with the FjordOs model is performed. Most of the observations are gathered from different sources independent of the FjordOs project, and include measurements of water level, currents, and water temperature at fixed station in addition to CTD measurements scattered in time and space. In addition a short scientific cruise on board the research vessel (R/V) Trygve Braarud was conducted as part of the FjordOs project in September 2015 to provide additional observations of hydrography and not least trajectories of drifters \citep{hjelm:etal:2016}. Finally, some observations were performed close to Svelvik in 2015 in the Drammensfjord, a western branch of the main Oslofjord \citep{staalstrom:2017}.

While Section 2 and 3 give a brief introduction to the Oslofjord and the model, more details on the observations are offered in Section 4. The assessment of the model's performance is presented in Section 5, while a summary including conclusions and some final remarks are proffered in Section 6. Some calculations on driftling lanes from the Slagen refinery are added in the Appendix.



%Provided is an evaluation of the FjordOs model \citep{roed:etal:2016}. The aim of this study is to reveal any weaknesses of the FjordOs model and clarify the extent to which the model can be be trusted. The study is a part of the FjordOs project. FjordOs is a cooperation between MET Norway, University College of Southeast Norway (HSN), The Norwegian Institute for Water Research (NIVA), The Norwegian Coastal Administration (Kystverket), Exxonmobil, Norwegian Defence Research Establishment (FFI), Vestfold, Buskerud, and \O stfold county, and AGNES AB Milj\o konsulent.

%The evaluation is based on existing observations in the area of interst and results from the FjordOs model \citep{roed:etal:2016}. The observations are gathered from different sources and not carried out as a part of the FjordOs project. The observations includes measurements of water level, current measurements, water temperature, and CTD measurements. A short scientific cruise on board the research vessel (R/V) Trygve Braarud in September 2015 provided additional observations of hydrography and trajectories of drifters. These observations are compared with simulated results in \cite{hjelm:etal:2016}. In addition some observations was performed close to Svelvik in 2015 in order to evaluate the model in the Drammensfjord, the western branch of the Oslofjord \citep{staalstrom:2017} . 



\newpage

\section{The Oslofjord and the FjordOs model}

The area of interest, and the domain covered by the model FjordOs, is the Oslofjord including the Drammensfjord and the Inner Oslofjord (Fig.~\ref{fig:kart}). The fjord is located in southeastern Norway and is well described in the literature, e.g., \cite{baalsrud:2002}, \cite{roed:etal:2016}, and \cite{hjelm:etal:2017}. Here we only point out some salient facts to keep in mind when establishing a circulation model aimed at providing pathways of various effluents to the fjord.

As revealed by Figure \ref{fig:kart}, the fjord is rather long and narrow with occasional wider parts. At about 59.5$^o$N the fjord splits in two branches. A western branch tapers into a narrow strait at Svelvik before it opens up a bit to form the Drammensfjord. An eastern branch forms the long and narrow Dr{\o}bak Sound, before it also opens up to form the somewhat wider Inner Oslofjord with its characteristic "swan head". In the north south direction it is about 100 km long. At the entrance it is about 50 km wide, in the Dr{\o}bak Sound about 1-2 km, and as narrow as 180 meters at Svelvik.


\begin{figure}[htb]
\centerline{
\includegraphics*[trim=0cm 0.9cm 0cm 0cm,clip=true,width=0.85\textwidth]{Figurer/kart}
}
\caption{\small
Area of interest; The Oslofjord. The red dots show the location of some major and minor cities and villages along the coast. The blue diamnond indicated the position of the F{\ae}rder Lighthouse.
}
\label{fig:kart}
\end{figure}

Both branches have a sill. The sill in the eastern branch, the Dr{\o}bak Sill, is located close to the island Kaholmen which holds the citadel Oscarsborg. It consists partly of a man made underwater jetty only 1-2 meters deep extending halfway across the fjord from the western side. East of the jetty there is a natural sill of about 20 meters depth. Due its narrowness and shallowness the Dr{\o}bak Sill area is famous for its strong tidal currents, which easily exceeds 1 m/s even though the mean total tidal amplitude is less than 20 cm. 
The sill in the western branch is rather long and narrow, about 1 km long and 180 meters wide. The minimum depth is as shallow as 11 meters. This sill causes a strong tidal current called the Svelvikstraum.
North of the sills the maximum depth is more than 120 meters in both branches. 

In addition to the existence of many small and large islands giving rise to many narrow sounds and straits, the fjord also have several deep basins ranging from 190 to 400 meters depths. Moreover, the fjord also exhibit a rather irregular coastline, and several river discharge fresh water into the fjord. Among the latter are two of Norway's largest rivers, namely Glomma (near Fredrikstad) and Drammenselva (near Drammen). Another important contributer to the water level variations and circulation pattern in the fjord is the impact of events in the Skagerrak/North Sea through the Oslofjord's southern perimeter. For instance are storm surge events with amplitudes of one meter and higher observed in the fjord. These events are mostly associated with wind and pressure events in the Skagerrak/North

These complexities all contributes to a compounded circulation pattern, a pattern that is important to resolve when designing a circulation model for the fjord aimed at providing as realistic as possible pathways of effluents. To possibly account for all these complexities and at the same time not exceeding the available computer capacities we opted to adapt the Rutgers Regional Ocean Modeling System (ROMS) when constructing the model for the Oslofjord, and to exploit its curvilinear option. ROMS is a publicly available ocean model featuring a terrain-following vertical coordinate and a free-surface. It is well documented by \cite{haidv:etal:2008} and by \cite{shche:mcwil:2003,shche:mcwil:2005,shche:mcwil:2009}. The particular version adapted to the Oslofjord is called FjordOs. The FjordOs model and the simulations performed for the two years 2014 and 2015 is described in \cite{roed:etal:2016}. The latter also describes the setup including the applied external inputs, such as atmospheric input, river input, tidal input, and the input of sea level, currents and hydrography at the model's open lateral southern boundary.

%The FjordOs model covers the area of interest (Fig.~\ref{fig:kart}). The model period in this study is April 2014 to December 2015.

\clearpage

%\section{Area of interest}

The area of interest is the Oslofjord including the Drammensfjord and the Inner Oslofjord (Fig.~\ref{fig:kart}). The Oslofjord is located in Southern Norway with the capital of Norway, Oslo, located in the innermost part of the fjord. 
The fjord is about 100 km long, and at 59.56$^o$N the fjord splits into two branches; the Inner Oslofjord (eastern branch) and the Drammensfjord (western branch). The width of the fjord varies from around 50 km at the entrance to about 1-2 km at the Dr{\o}bak Sound in the Inner Oslofjord and 180 meters at Svelvik in the Drammensfjord.

Both branches have a sill. The sill in the eastern branch, the Dr{\o}bak Sill, is located close to the island Oscarsborg. The Dr{\o}ak Sill consists partly of a man made sill, an underwater jetty only 1-2 meters deep extending halfway across the fjord from the western side. Towards the eastern side there is a natural sill of about 20 meters depth. Due its narrowness and shallowness the Dr{\o}bak Sill area is famous for its strong tidal currents, which easily exceeds 1 m/s even though the mean total tidal amplitude is less than 20 cm. 
The western branch is almost cut in two by a narrow, shallow, and long sill which is only 11 meters deep, 180 meters wide, and more than 1 km long. This sill causes a relatively strong tidal current, called the Svelvikstraum.
North of these sills the maximum depth is more than 120 meters in both branches. 
The location of these sills, about two thirds into the fjords, makes the Oslofjord peculiar among Norwegian fjords in that most of them have their sill at the entrance.

Numerous smaller and larger islands combined with deeper and more shallow basins (Fig.~\ref{fig:kart}) contributes to a complex circulation pattern. In addition several river discharge fresh water into the fjord, including two of Norway's largest rivers, namely Glomma (near Fredrikstad) and Drammenselva (near Drammen).

The water level and motion in the Oslofjord is also impacted by events in the Skagerrak, which lies in the north-eastern part of the North Sea outside of the fjord's southern boundary. Storm surge events with amplitudes of one meter and higher are observed in the fjord and are mostly associated with wind and pressure events in the Skagerrak/North Sea area.

\begin{figure}[htb]
\centerline{
\includegraphics*[trim=0cm 0.9cm 0cm 0cm,clip=true,width=0.7\textwidth]{Figurer/kart}
}
\caption{\small
Area of interest; The Oslofjord. The red dots show the location of some major and minor cities and villages along the coast. The blue diamnond indicated the position of the F{\ae}rder Lighthouse.
}
\label{fig:kart}
\end{figure}

\clearpage

%\section{Model}

The FjordOs model is a curvilinear, free-surface, and terrain-following model based on the Rutgers Regional Ocean Modeling System (ROMS) \citep{haidv:etal:2008,shche:mcwil:2003,shche:mcwil:2005,shche:mcwil:2009} adapted to the Oslofjord \citep{roed:etal:2016}. 

The model applies several external inputs, such as atmospheric input, river input, tides, and input of sea level, currents and hydrography at the model's open lateral boundaries, in addition to bathymetry (Fig.~\ref{fig:oppbygging}). Mean values of sea level, currents, and hydrography from the NorKyst800 model \citep{albre:etal:2011} is applied on the open boundary towards Skagerak. The necessary atmospheric input is extracted from the AROME-MetCoOp model that runs operationally at MET Norway \citep{mulle:etal:2015}. The tidal input is based on the TPXO Atlantic database \citep{egber:erofe:2002} and modified using the measurements at Viker close to the southern boundary. The freshwater discharges from the rivers are based on the discharge data from a database constructed by use of the hydrological model HBV \citep{beldr:etal:2003}. For more details on the FjordOs model, see \cite{roed:etal:2016}.

The FjordOs model covers the area of interest (Fig.~\ref{fig:kart}). The model period in this study is April 2014 to December 2015.

\begin{figure}[htb]
\centerline{
\includegraphics*[width=0.9\textwidth]{Figurer/oppbygging}
}
\caption{\small
Illustration of external inputs to the FjordOs model.
}
\label{fig:oppbygging}
\end{figure}

\newpage

\clearpage
\section{Observations}
\label{sec:obser}

Model simulations are performed for the period April 2014 through December 2015. Locations of observations available for this period, are shown by Figure~\ref{fig:kart_obs}. They encompass water level as detailed in Section \ref{subsec:wlevelo}, profiles of currents (Section \ref{subsec:curreo}), temperature and salinity (Section \ref{subsec:CTDo}), and temperature at fixed positions (Section \ref{subsec:tempeo}). Also data from the Godafoss oil spill in February 2011 are available to us (Section \ref{sect:godafoss_obs}). These observations are all gathered independent of the FjordOs project, and most of them are scattered in time and space. The exceptions are water level at Viker, Oscarsborg and Oslo and currents at Slagen. They are gathered regularly in time and for a much longer period. Finally, through the FjordOs project, we have collected surface drifter trajectory data during two cruises, one in September 2014 and another in September 2015 as detailed in Section \ref{subsec:drifto}.
\begin{figure}[htb]
  \begin{center}
    \begin{tabular}{c}
      \includegraphics*[height=11cm]{Figurer/kart_obs} \\ 
    \end{tabular}
    \caption{\small Names and locations where observations independent of the FjordOs project are available for the simulation period. Dark purple solid circles correspond to fixed temperature stations, green triangles to hydrographic stations (CTD stations), blue squares to water level stations, and red diamonds to moorings equipped with an Acoustic Doppler Current Profiler (ADCP).}
    \label{fig:kart_obs}
  \end{center}
\end{figure}




% % % % % % % % % % % % % % % % % % % % %
\subsection{Water level}
\label{subsec:wlevelo}
The Norwegian Mapping Authority has three permanent stations measuring sea level in the Oslofjord (Figure~\ref{fig:kart_obs}), namely Viker, Oscarsborg and Oslo. As shown the station at Viker is located close to the open boundary of the model area, while Oscarsborg is located halfway into the Inner Oslofjord. The station Oslo is located in the Oslo Harbour. The station Viker was used to adjust the tidal input to the FjordOs model in accord with \cite{hjelm:etal:2017}.

% % % % % % % % % % % % % % % % % % % % % % % % % % % %
\subsection{Currents}
\label{subsec:curreo}
%  %  %  %  %  %  %  %  %  %  %  %  % 
\subsubsection{Statnett moorings}
During the period mid September through late November 2014 six moorings, each fitted with an upward looking Acoustic Doppler Current Progiler (ADCP), were deployed in the positions shown by Figure~\ref{fig:kart_obs}. Information about mooring reference, names and locations of these moorings (in terms of latitude and longitude) and instrument type are tabulated by Table~\ref{tab:Statnett}. Two of the moorings (Filtvedt and Brenntangen) were deployed at the entrance to the Dr{\o}bak Sound. The remaining four (Sm{\aa}skj{\ae}r, Laksetrappa, Botnegrunnen, and Evje) were deployed further south forming an east-west section across the fjord along a line with deep basins on either side (Figure~\ref{fig:kart_obs}). The moorings were all deployed as part of a project conducted by Statnett, NIVA, Akvaplan NIVA, and the University of Oslo (UiO). The R/V Trygve Braarud, UiO was used during deployment and recovery of the moorings. For further details about the moorings and corresponding instruments the reader is referred to \cite{staalstrom:2015}.
\begin{table}[htb] 
\caption{\small Target positions (WGS84) of the Statnett ADCP current instrument moorings. Depths at the stations are from the Statnett terrain model. Note that the model depth at the same location may differ due to the smoothing of the model topography.} 
\label{tab:Statnett} 
\centering 
\begin{tabular}{|llcccl|} 
\hline  
\small{{\bf Mooring}} & \small{{\bf Name}} & \small{{\bf Latitude}} & \small{{\bf Longitude}} & \small{{\bf Depth}} & \small{{\bf Instruments}}\\ 
\small{{\bf ref}}    &		           & \small{{\bf [$^o$N]}}  & \small{{\bf [$^o$E]}}   & \small{{\bf [m]}} &	\\ \hline
\small{Kp11.2} & \small{Sm{\aa}skj{\ae}r} & \small{59.350124} & \small{10.497661} & \small{20}	& \small{Aquadopp600 AQP1531}	\\
\small{(Ri1)}  &                          &		      &			  &		& \small{Transducer LRT2}	\\ \hline
\small{Kp5.7}  & \small{Laksetrappa}      & \small{59.343452} & \small{10.581023} & \small{75}  & \small{Aquadopp400 AQP4689}	\\
\small{(Rl1)}  &			  &		      &			  &     	& \small{Transducer LRT3}	\\
	       &			  &		      &			  &     	& \small{Aanderaa Seaguard}	\\ \hline
\small{Kp2.6}  & \small{Botnegrunnen}     & \small{59.352375} & \small{10.626822} & \small{96}  & \small{Continental WAV6117}	\\
\small{(Rm1)}  &			  &		      &			  &     	& \small{Transducer LRT4}	\\ \hline
\small{Kp0.7}  & \small{Evje}             & \small{59.363182} & \small{10.653576} & \small{64}  & \small{Aquadopp400 AQP2931}	\\
\small{(Rn1)}  &			  &	      	      &			  &		& \small{Transducer LRT5}	\\ \hline
\small{Kn2}    & \small{Brenntangen} 	  & \small{59.581803} & \small{10.646087} & \small{54}  & \small{Aquadopp400 AQP5608}	\\
	       &			  &		      &			  &		& \small{Transducer LRT6}	\\ \hline
\small{Km1}    & \small{Filtvedt}	  & \small{59.582064} & \small{10.627372} & \small{153} & \small{Continental CNL6037}	\\
	       & (current) 		  &		      &			  &		& \small{Transducer 207-2}	\\ \hline
\small{Km2}    & \small{Filtvedt}	  & \small{59.580778} & \small{10.626239} & \small{125} & \small{TinyTags UIO1-7}	\\
	       & (temperature) 		  &		      &			  &		& \small{Transducer 203-2}	\\
\hline
\end{tabular}
\end{table}




%  %  %  %  %  %  %  %  %  %  %  %  % 
\subsubsection{ExxonMobil mooring}
Data from an additional bottom-mounted ADCP, named Slagen (Figure~\ref{fig:kart_obs}), and made available to us by ExxonMobil, Slagentangen, is also used for comparison with model results. It has measured currents regularly at two depths since 1997, and is located northwest of Turning Dolphin at the Slagen Refinery as shown by Figure~\ref{fig:Slagen-kart}. Note the Bliksekilen nature reserve, which is a shallow water area with rare flora and fauna, which is located west of the Slagen Refinery. 
\begin{figure}[htb]
  \begin{center}
    \begin{tabular}{c}
      \includegraphics*[height=9cm]{Figurer/Slagen_kart} \\ 
    \end{tabular}
    \caption{\small Zoom in of the location of the bottom-mounted ADCP close to the Slagen Refinery (red dot). Source: Norwegian Coastal Administration.}
    \label{fig:Slagen-kart}
  \end{center}
\end{figure}




% % % % % % % % % % % % % % % % % % % % % % % % % % % %
\subsection{CTD measurements}
\label{subsec:CTDo}
On behalf of Fagr{\aa}det for Ytre Olsofjord NIVA collects CTD measurements at a number of selected positions in the Oslofjord. The work is part of a program monitoring the eutrophication state of the Outer Oslofjord and the Drammensfjord. The data collected are available through a web portal\footnote{http://www.aquamonitor.no/ytreoslofjord/}. We use data from ten of these as listed by Table ~\ref{tab:CTD_pos}. Their resepctive locations are shown in Fig.~\ref{fig:kart_obs} as green triangles. The measurements include profiles of temperature and salinity as well as water quality parameters. During the simulation period April 2014 through December 2015 12 CTD profiles was available covering the months January, February, June, July, August, September, and November. No data are unfortunately available in spring and early summer. 
\begin{table}
	\caption{\small Positions (latitude, longitude) and number of profiles taken at each of the ten CTD measurement sites used.} 
	\label{tab:CTD_pos} 
	\centering 
	\begin{tabular}{|llcccc@{}c|} 
	\hline  
{\bf \small{Tag}} & {\bf \small{Station}} & {\bf \small{Latitude}} & {\bf \small{Longitude}} & \multicolumn{2}{c}{\bf \small{Number of measurements}} &\\ 
 	&	& {\bf \small{[$^o$N]}} & {\bf \small{[$^o$E]}} & \small{2014}  & \small{2015} &\\ \hline
\small{D-2}	& \small{Inner Drammensfjord} & \small{59.6280}	& \small{10.4210} & \small{5} & \small{7}  &	\\ 
\small{D-3}	& \small{Solumstrand} 	      & \small{59.7060} & \small{10.3140} & \small{5} & \small{6}  &	\\ \hline
\small{LA-1}	& \small{Larviksfjord}	      & \small{59.0190}	& \small{10.0520} & \small{5} & \small{7}  &	\\ 
\small{MO-2}	& \small{Kippenes}	      & \small{59.4840}	& \small{10.6780} & \small{5} & \small{7}  &	\\ \hline
\small{OF-1}	& \small{Torbj{\o}rnskj{\ae}r}& \small{59.0410}	& \small{10.7540} & \small{5} & \small{7}  &	\\ 
%\small{OF-2}	& \small{Missingene}	      & \small{59.1870}	& \small{10.6920} & \small{0} & \small{14} &	\\ 
\small{OF-5}	& \small{Breiangen}	      & \small{59.4870}	& \small{10.4580} & \small{5} & \small{7}  &	\\ \hline
\small{S-9}	& \small{Haslau, Singlefjord} & \small{59.1140}	& \small{11.1620} & \small{7} & \small{10} &	\\ 
\small{SF-1}	& \small{Sandefjord}	      & \small{59.0770}	& \small{10.2460} & \small{5} & \small{7}  &	\\ \hline
\small{T{\O}-1}	& \small{Vestfjord}	      & \small{59.2030}	& \small{10.3550} & \small{5} & \small{7}  &	\\ 
\small{{\O}-1}	& \small{Leira. Vesterelva}   & \small{59.1370}	& \small{10.8340} & \small{7} & \small{10} &	\\ \hline
	\end{tabular}
\end{table}




We note that only a few of the CTD stations are located in the open part of the fjord. In fact seven of the ten stations are located inside narrow straits and sounds, or inside lesser subfjords or inlets. Of the remaining three stations only two are in the open parts of the fjord, namely Torbj{\o}rnskj{\ae}r (OF-1) and Breiangen (OF-5), while the last ({\O}-1) is positioned in a semi-open location west of the Hvaler Archipelago. The latter is interesting in that it is influenced by the fresh water discharged by the western branch of the river Glomma. In this respect also the stations D-2 and D-3 are interesting being located inside the sill in the Drammensfjord in which the river Drammenselva discharges its fresh water.   

% % % % % % % % % % % % % % % % % % % % % % % % % % % %
\subsection{Temperature measurements}
\label{subsec:tempeo}
Temperature measurements at four fixed position are available to us. One of them is oerated by Scanmar AS and is located three kilometres south of {\AA}sg{\aa}rdstrand (Figure~\ref{fig:kart_obs}). The remaining three are located in the Inner Oslofjord (Figure~\ref{fig:kart_strand}).
\begin{figure}[htb]
    \centerline{
	\begin{minipage}[l]{0.59\textwidth}
		\includegraphics*[trim=0 0 0 1cm,clip=true,width=\textwidth]{Figurer/badestrand_kart}
	\end{minipage}
	\begin{minipage}[r]{0.4\textwidth}
		\fbox{\includegraphics*[trim=1 0 0 3cm,clip=true,width=\textwidth]{Figurer/kart_Storoyodden.png}} \\
		\fbox{\includegraphics*[trim=0 0 0 3cm,clip=true,width=\textwidth]{Figurer/kart_Hvalstrand.png}} \\
		\fbox{\includegraphics*[trim=1 1cm 0 1.5cm,clip=true,width=\textwidth]{Figurer/kart_Sjostrand.png}} \\
	\end{minipage}
	}
    \caption{\small The positions at three beaches in the Inner Oslofjord where the temperature measurements are performed.}
    \label{fig:kart_strand}
\end{figure}




%  %  %  %  %  %  %  %  %  %  %  %  % 
\subsubsection{The Scanmar mooring}
The Scanmar mooring measures temperature hourly at one meter depth, and has done so over the last ten years. The device has an accuracy of $\pm 0.15^{\textrm{o}}$C in the range from -5 to +30$^{\textrm{o}}$C. 

\begin{comment}

\begin{table}[ht] 
\caption{Observed water temperature near \AA sg\aa rdstrand} 
\label{tab:Scan_temp} 
\centering 
\begin{tabular}{|cccccc|} 
\hline  
{\bf Year} & {\bf Minimum} & {\bf 5 percentile} & {\bf Mean} & {\bf 95 percentile} & {\bf Maximum} \\
\hline
\small 2005 & 1.0 & 4.0 & 13.2 & 20.7 & 24.9 \\
\small 2006 & -6.0 & -0.5 & 10.1 & 20.6 & 23.9 \\
\small 2007 & -1.2 & 1.1 & 9.7 & 18.3 & 21.4 \\
\small 2008 & 0.1 & 2.6 & 10.4 & 19.3 & 24.6 \\
\small 2009 & -2.3 & -0.3 & 9.4 & 19.8 & 24.9 \\
\small 2010 & -1.7 & -0.2 & 8.7 & 18.7 & 20.5 \\
\small 2011 & -1.4 & -1.0 & 9.6 & 19.0 & 22.6 \\
\small 2012 & -1.0 & 0.4 & 9.2 & 18.5 & 21.4 \\
\small 2013 & -1.3 & -0.4 & 9.3 & 19.4 & 22.2 \\
\small 2014 & -1.1 & 1.2 & 10.6 & 21.7 & 26.4 \\
\small 2015 & 0.7 & 3.5 & 10.5 & 18.7 & 20.8 \\
\hline
\end{tabular}
\end{table}

\end{comment}
%\newpage

%  %  %  %  %  %  %  %  %  %  %  %  % 
\subsubsection{Temperatures in the Inner Oslofjord}
In addition to the Scanmar mooring we have access to temperature measurements at three beaches in the Inner Oslofjord, namely Sj{\o}strand, Hvalstrand and Stor{\o}yodden (Figure~\ref{fig:kart_strand}). These measurements are the result of a collaboration between Asker and B{\ae}rum kommune and Finnerud Elektronikk. The measurement device is a digital thermometer (Maxim Integrated DS18B20) with an accuracy of $\pm 0.5^{\textrm{o}}$C. They measure water temperature at 40 cm beneath the surface in water depths of several meters. Temperatures are measured every three hours from 09:00 to 18:00 during the summer months. We note that the site Sj{\o}strand is the only one located in a semi-open position, while the two other beaches are well within archipelagoes that are somewhat sheltered from the rest of the fjord.

%Finally one mooring at Filtvedt had TinyTag temperature loggers deployed at seven different depths between 20 and 120 m (rig Km2, Tab.~\ref{tab:Statnett})

\begin{comment}The trends of the observed temperatures at the three beaches are similar, but the temperature is generally lower at the southern beach, Sj{\o}strand, than at the northern beach, Stor{\o}yodden (Fig.~\ref{fig:temp_strand}). In 2014 there were two local maximuma during July, and the maximum observed temperatures in 2014 were higher than in both 2013 and 2015. The temperature increases 1-3 degrees during the day and decreases during the night.

\begin{figure}[ht]
\centerline{
\includegraphics*[trim=2cm 0cm 2cm 0cm,clip=true,width=\textwidth]{Figurer/temp_Scanmar}}
\caption{\small
Observed temperature measured by Scanmar AS}
\label{fig:Scan_temp}
\end{figure}

\begin{figure}[ht]
\centerline{
\includegraphics*[trim=2cm 0 2cm 0cm,clip=true,width=\textwidth]{Figurer/badetemp}
}
\caption{\small
Observed temperature at three beaches in the Inner Oslofjord}
\label{fig:temp_strand}
\end{figure}

\begin{table}[ht]
\caption{Mean observed temperatures at three beaches in the Inner Oslofjord. Only time periods with more than 8 days of observations during the given time period are included.}
\label{tab:temp_strand}
\begin{center}
\begin{tabular}{|l|ccc|ccc|ccc|} \hline
     & \multicolumn{3}{c|}{Stor\o yodden} & \multicolumn{3}{c|}{Hvalstrand} & \multicolumn{3}{c|}{Sj\o strand} \\ \hline
Time period & 2013 & 2014 & 2015 & 2013 & 2014 & 2015 & 2013 & 2014 & 2015 \\ \hline
%01 - 15 May & 0 & 0 & 0 & 0 & 0 & 0 & 0 & 0 & 0 \\ 
16 - 31 May & 13.7 &  -   & 11.6 &  -   &  -   & 12.2 &  -   &  -   & 12.8 \\ 
01 - 15 Jun & 15.0 & 19.0 & 13.3 & 15.9 & 19.4 & 13.7 &  -   & 19.9 & 14.6 \\ 
16 - 30 Jun & 16.2 & 17.3 & 17.1 & 16.7 & 17.7 & 17.5 &  -   & 17.9 & 18.0 \\ 
01 - 15 Jul & 17.8 & 17.6 & 18.2 & 18.3 & 18.3 & 19.8 & 19.4 & 19.3 &  -   \\ 
16 - 31 Jul & 20.1 & 22.2 & 18.1 & 19.7 &  -   & 18.9 & 21.2 & 23.9 & 18.4 \\ 
01 - 15 Aug & 19.5 & 21.3 & 18.0 & 19.8 & 21.3 &  -   & 20.6 & 21.7 & 18.8 \\ 
16 - 31 Aug & 18.9 & 19.6 & 19.1 & 18.9 & 19.6 & 19.3 & 19.5 & 19.4 & 19.0 \\ 
01 - 15 Sep & 17.9 & 18.3 & 16.1 & 17.7 & 18.6 & 16.5 & 18.1 & 18.9 & 16.2 \\ 
16 - 30 Sep &  -   & 15.9 & 14.1 &  -   & 16.2 & 14.1 & 15.1 & 16.3 & 13.7 \\ 
01 - 15 Oct &  -   &  -   & 12.0 &  -   &  -   & 11.9 &  -   &  -   & 11.5 \\ 
%16 - 31 Oct & 0 & 0 & 0 & 0 & 0 & 0 & 0 & 0 & 0 \\ 
\hline
\end{tabular}
\end{center}
\end{table}

\end{comment}

% % % % % % % % % % % % % % % % % % 
\subsection{Godafoss oil spill}
\label{sect:godafoss_obs}
On Thursday the 17th of February 2011 at 19:52 local time, the containership Godafoss ran aground at the Kv{\ae}rnskj{\ae}rgrunnen rock in L{\o}peren. It us located between the islands of Asmal{\o}y and Kirk{\o}y in the Hvaler municipality in southeastern Norway (Figure \ref{fig:godafoss_oil}). One of the effects of this grounding was an acute release of oil from the ship, which drifted westward from the accident site. The oil slick has been observed from aeroplane by the Norwegian Coastal Administration (Kystverket), and the sites where stranded oil has been observed, are also registered (Figure \ref{fig:godafoss_oil}). This accident was on of the motivation factors for the project FjordOs, and hence, although no simulations are performed for this particular event, we will, nevertheless simulate trajectories from this location using the simulations performed for the period April 2015 through December 2015 (Section \ref{sec:evalu}). 

At the time of the grounding there were clear skies and temperatures around -3$^o$C. Observations of wind from Str{\o}mtangen lighthouse (15 km away from the grounding site) indicate 6-7 m/s winds from the north-east.
\begin{figure}[htb]
	\centerline{ \includegraphics*[width=1.0\textwidth]{Figurer/Godafoss} }
	\caption{\small The observed oil spill from the Godafoss accident 17th of February 2011. The red arrow on the right-hand side indicates the grounding position (Kv{\ae}rnskj{\ae}rgrunnen). The grey areas are oil slicks observed from aircraft, while green areas indicate stranded oil. The name of the locations where oil was observed and at which time are included as text. Source: The Norwegian Coastal Administration}
	\label{fig:godafoss_oil}
\end{figure}




% % % % % % % % % % % % % % % % % % 
%\newpage
\subsection{Surface drifters}
\label{subsec:drifto}
As part of the FjordOs project two cruises in the Oslofjord were performed. During these cruises a total number of 15 surface drifters were released (Figure \ref{fig:drifters_design}). Two were released during a cruise in September 2014, and the remaining 13 was released in a cruise in September 2015 (Figure \ref{fig:drifters_tracks}). The second cruise is documented in \cite{hjelm:etal:2016}, which also describes the drifters used in detail. The focus area of the drifter campaigns is the Breiangen area, and the area between Horten and Moss.
\begin{figure}[htb]
	\centerline{
		\includegraphics*[width=0.3\textwidth]{Figurer/Driftere_ombord}\hspace{2cm}
		\includegraphics*[width=0.3\textwidth]{Figurer/Driftere_vann}
		}
	\caption{\small The "home-made" drifters on the deck of the R/V Trygve Braarud (left-hand panel) and in the water after deployment (right-hand panel). In all 15 of these were dropped and tracked during the FjordOs project, two in 2014 and 13 in 2015.}
	\label{fig:drifters_design}
\end{figure}



\begin{figure}[htb]
  \begin{center}
    \begin{tabular}{cc}
      \includegraphics*[height=9cm]{Figurer/drifters_sept2014} & 
      \includegraphics*[height=8cm]{Figurer/drifters_low_crop} \\ 
    \end{tabular}
    \caption{\small Trajectories of the two drifters released in September 2014 (left-hand panel) and the 13 drifters released in September 2015 (right-hand panel).}
    \label{fig:drifters_tracks}
  \end{center}
\end{figure}

\begin{comment}

\begin{figure}[ht]
	\centerline{
		\includegraphics*[width=0.4\textwidth]{Figurer/drifters_sept2014}
		\includegraphics*[width=0.4\textwidth]{Figurer/drifters_low_crop}
		}
	\caption{\small Trajectories of drifters released in September 2014 (left-hand panel, two drifters) and September 2015 (right-hand panel, 13 drifteres).}
	\label{fig:drifters_tracks-2}
\end{figure}
\end{comment}




\clearpage


\newpage
\section{Evaluation}
\subsection{Water level and tide}

Time series from the three permanent stations measuring water level have been analysed and compared with simulated time series of water level extracted from locations near the three permanent stations. Both simulated and observed time series of water level are analysed using t\_tide \citep{pavlo:etal:2002} in order to extract the tidal components. The same period in time is applied for both the simulations and the observations (April 2014 to December 2015). 

\begin{figure}[hb] 
\centerline{ 
\includegraphics*[trim=3cm 0cm 2.5cm 0cm,clip=true,width=\textwidth]{Figurer/Oscarsborg_Tide_selected_jan15}  
} 
\centerline{ 
\includegraphics*[trim=3cm 0cm 2.5cm 0cm,clip=true,width=\textwidth]{Figurer/Oscarsborg_WL_rest_jan15} 
} 
\caption{\small 
Simulated (black) and observed (red) time series of tides (upper) and residual (lower) at Oscarsborg. Here the tidal elevation includes only the eleven components included in the tidal forcing. The residual includes the total water level minus the tidal elevation.} 
\label{fig:Waterlevel_jan15} 
\end{figure} 

Eleven tidal components are included in the model at the southern open boundary using their corresponding amplitudes and phases for both depth integrated currents and water level \citep{roed:etal:2016}. The time series of tidal components included in the tidal forcing are in fairly good agreement (Fig.~\ref{fig:Waterlevel_jan15},upper, and Tab.\-\ref{tab:Tide}). In addition to the tidal components included in the tidal forcing, more tidal components are present in the time series (Fig.~\ref{fig:Waterlevel_tide}). Tidal components with periods of approximately one year (SA) and half a year (SSA) respectively are present in both the observations and the simulations (Tab.\-\ref{tab:Tide}). In addition, the observations have more components with shorter periods which are not included in the tidal forcing and thereby not present in the simulations. This is consistent with the frequency series of the Fourier transformed water level (Fig.~\ref{fig:Waterlevel_freq}).

\begin{figure}[tbh] 
\centerline{ 
\includegraphics*[trim=3cm 0cm 2.5cm 0cm,clip=true,width=\textwidth]{Figurer/Oscarsborg_Tide_not_included}  
} 
\caption{\small 
Time series at Oscarsborg of the tidal components not included in the tidal tidal forcing.} 
\label{fig:Waterlevel_tide} 
\end{figure} 

\begin{figure}[tbh] 
\centerline{ 
\includegraphics*[trim=3cm 0cm 2.5cm 0cm,clip=true,width=\textwidth]{Figurer/Oscarsborg_Tide_Frequency_obs.png}  
} 
\centerline{ 
\includegraphics*[trim=3cm 0cm 2.5cm 0cm,clip=true,width=\textwidth]{Figurer/Oscarsborg_Tide_Frequency_sim.png} 
} 
\caption{\small 
Frequency series of Fourier transformed observed (upper) and simulated (lower) water levels at Oscarsborg} 
\label{fig:Waterlevel_freq} 
\end{figure} 

\newpage 
The amplitude of M$_2$ increases from south to north in the Inner Oslofjord both in the simulations and the observations (Fig.~\ref{fig:M2field} and Tab.~\ref{tab:Tide}). The lowest M$_2$ amplitude in the area of interest, is in the Drammensfjord north of the threshold in Svelvik. The M$_2$ phase has a sudden increase at the thresholds of Svelvik and Dr{\o}bak (Fig.~\ref{fig:M2field}). The same yields for the majority of the other relevant tidal components.

\begin{figure}[hb] 
\centerline{ 
\includegraphics*[trim=1cm 0cm 0cm 0cm,clip=true,width=0.49\textwidth]{Figurer/M2amp_felt}  
\includegraphics*[trim=0.8cm 0cm 0cm 0cm,clip=true,width=0.49\textwidth]{Figurer/M2fase_felt} 
} 
\caption{\small 
Simulated field of M$_2$ amplitude (left) and phase (right). The corresponding observed values for M$_2$ amplitude and phase are marked with circles at Viker, Oscarsborg, and Oslo.} 
\label{fig:M2field} 
\end{figure} 

\begin{table}[ht] 
%\vspace{-1.5cm} 
\caption{Simulated and observed tidal amplitude and phase for selected tidal components.} 
\label{tab:Tide} 
\centering 
\begin{tabular}{|c|c|l|cc|cc|cc|c|} 
\hline  
&&& \multicolumn{2}{|c|}{\bf Viker} & \multicolumn{2}{|c|}{\bf Oscarsborg} & \multicolumn{2}{|c|}{\bf Oslo} & {\bf Included} \\  
{\bf Comp.} & {\bf Period} &  {\bf sim/} & {\bf amp.} & {\bf phase.} & {\bf amp.} & {\bf phase.} & {\bf amp.} & {\bf phase.} & {\bf in tidal} \\ 
& {\bf [h]} & {\bf obs} & {\bf [cm]} & {\bf [deg]} & {\bf [cm]} & {\bf [deg]} & {\bf [cm]} & {\bf [deg]} & {\bf forcing} \\ \hline 
\small SA   & 8764	 	& sim & 15.5 & 284 & 15.6 & 286 & 15.4 & 286 & no   \\
\small      &        	& obs & 10.0 & 319 & 11 & 322 & 11.4 & 324 &    \\
\small SSA  & 4382 		& sim & 8.8 & 197 & 9.2 & 200 & 9.4 & 200 & no   \\
\small      &        	& obs & 7.5 & 188 & 8.0 & 189 & 8.2 & 190 &    \\
\small K2   & 11.9672 	& sim & 1.6 & 10 & 2.0 & 13 & 2.1 & 15 & yes  \\
\small      &        	& obs & 0.7 & 45 & 0.8 & 66 & 0.9 & 66 &    \\
\small S2   & 12.0000 	& sim & 3.3 & 64 & 3.9 & 69 & 4.2 & 70 & yes  \\
\small      &        	& obs & 2.9 & 46 & 3.3 & 65 & 3.5 & 69 &    \\
\small M2   & 12.4206 	& sim & 11.5 & 105 & 13.2 & 112 & 13.9 & 114 & yes  \\
\small      &        	& obs & 11.9 & 105 & 13.8 & 121 & 14.4 & 125 &    \\
\small N2   & 12.6584 	& sim & 3.0 & 69 & 3.5 & 75 & 3.7 & 76 & yes  \\
\small      &        	& obs & 3.0 & 60 & 3.4 & 76 & 3.6 & 80 &    \\
\small K1   & 23.9345 	& sim & 0.2 & 187 & 0.1 & 175 & 0.2 & 157 & yes  \\
\small      &        	& obs & 0.4 & 127 & 0.7 & 130 & 0.8 & 130 &    \\
\small P1   & 24.0659 	& sim & 0.6 & 322 & 0.6 & 334 & 0.7 & 342 & yes  \\
\small      &        	& obs & 0.2 & 129 & 0.3 & 102 & 0.4 & 97 &    \\
\small O1   & 25.8193 	& sim & 3.5 & 337 & 3.8 & 339 & 3.8 & 339 & yes  \\
\small      &        	& obs & 2.2 & 277 & 2.3 & 281 & 2.4 & 282 &    \\
\small Q1   & 26.8684 	& sim & 0.0 & 231 & 0.0 & 216 & 0.1 & 215 & no   \\
\small      &        	& obs & 1.1 & 190 & 1.2 & 198 & 1.3 & 200 &    \\
\small MN4  & 6.2692 	& sim & 0.2 & 5 & 0.5 & 32 & 0.6 & 35 & yes  \\
\small      &        	& obs & 0.4 & 249 & 0.6 & 289 & 0.7 & 297 &    \\
\small M4   & 6.2103 	& sim & 1.0 & 355 & 1.9 & 18 & 2.5 & 23 & yes  \\
\small      &        	& obs & 1.2 & 281 & 1.8 & 324 & 2.3 & 332 &    \\
\small MS4  & 6.1033 	& sim & 0.6 & 80 & 1.2 & 107 & 1.6 & 111 & yes  \\
\small      &        	& obs & 0.3 & 360 & 0.5 & 44 & 0.7 & 56 &    \\
\hline
\end{tabular}
\end{table}



\clearpage 
\subsection{Currents}

\subsubsection{Currents in two cross sections}

Current measurements were performed in two cross sections. Here the model results are evaluated using the measurements at the northern cross section, Brenntangen-Filtvedt. The results are similar at the southern cross section.

The observed and simulated currents at Filtvedt and Brenntangen are of the same magnitude in strength, but the flow pattern differs. Figure~\ref{fig:Filtvedt-cur} shows the observed and simulated currents at Filtvedt as an example. The observed currents at Brenntagen is similar in strength, but the peaks in current strength do not occur at the same time. The observations are dominated by noise in the upper 35-40 meters. Only depths larger than 40 meters are therefore included in the comparison.

Note that the 3D current field is complex, the currents vary horizontally, vertically and with time. Figure~\ref{fig:Filtvedt-simcur} shows the simulated currents at three different depths. In general, the currents in the upper layer are stronger than further down. In the upper layers the currents are towards north, at 40 meters depth the currents towards south, and at 100 meters depth towards north. Because of the complex flow pattern, the currents at a given coordinates cannot be taken at representative for the whole area. At 100 meters depth the currents at Filtvedt are weak and towards south even though the currents at 100 meters depth is generally stronger and towards north. At 40 meters depth the currents at Brenntangen is weaker than in the rest of the cross section. Note that the depth at Brenntangen is only 58 meters in the observations while in the depth is 46 meters at the corresponding point in the simulations. 

\begin{figure}[ht]
\centerline{
\includegraphics*[trim=0 0 0 0,clip=true,width=\textwidth]{Figurer/Filtvedt_obs_cur}}
\centerline{
\includegraphics*[trim=0 0 0 0,clip=true,width=\textwidth]{Figurer/Filtvedt_sim_cur}}
\caption{\small
Observed (upper) and simulated (lower) currents at Filtvedt. Since the observations near the surface where dominated by noise, only depths larger than 40 meters are included in the upper plot. $z$ = 40 meters is marked with a black line in the lower plot. Note that the model depth is only 155 meters at the position the observations where performed.}
\label{fig:Filtvedt-cur}
\end{figure}

\begin{figure}[ht]
\centerline{
\includegraphics*[trim=2cm 3cm 1cm 3.3cm,clip=true,height=5cm]{Figurer/Filtvedt_t4611_z2_current}
\includegraphics*[trim=3.8cm 3cm 6cm 3.3cm,clip=true,height=5cm]{Figurer/Filtvedt_t4611_z40_current}
\includegraphics*[trim=3.8cm 3cm 1cm 3.3cm,clip=true,height=5cm]{Figurer/Filtvedt_t4611_z100_current}}
\caption{\small
Simulated currents at 2 (left), 40, and 100 (right) meters depth at 10 October 2014 12:00. Note that the two plots to the right have the same colorbar.}
\label{fig:Filtvedt-simcur}
\end{figure}

The tides are evident in the observed and simulated currents at all observation points. Both simulated and observed time series of the currents at the seven observation points are analyzed using t\_tide \cite{pavlo:etal:2002} in order to extract the tidal components for each depth. The same period in time is applied for both simulations and observations (mid-September to the end of November 2014). Figure~\ref{fig:Filtvedt-tide} shows the tidal currents at Filtvedt. At Filtvedt the tidal impact in the observations occurs earlier at larger depths, but in the simulations the tidal impact occurs earlier at more shallow depths. This might be due to the flow pattern at the different depths and the fact that the flow pattern depends strongly on the bottom topography which is smoothed in the simulations. Note also that the calculations of the tides are based on only six weeks of data.


\begin{figure}[ht]
\centerline{
\includegraphics*[trim=0 0 0 0,clip=true,width=\textwidth]{Figurer/Filtvedt_obs_tide}}
\centerline{
\includegraphics*[trim=0 0 0 0,clip=true,width=\textwidth]{Figurer/Filtvedt_sim_tide}}
\caption{\small
Observed (upper) and simulated (lower) tidal currents at Filtvedt.}
\label{fig:Filtvedt-tide}
\end{figure}


\clearpage 
\subsubsection{Current at Slagentangen}
The observed currents at Slagentangen are compared with simulated data from 1st October 2014 until 30th November 2015 at approximately the same location and depth (Fig.~\ref{fig:Slagen-kart}).

%\begin{figure}[ht]
%\centerline{
%\includegraphics*[trim=1cm 0cm 1cm 0cm,clip=true,width=0.5\textwidth]{Figurer/Slagen_kart}}
%\caption{\small
%Map retrived from the Norwegian Coastal Administration. The red dot marks the position corresponding to the extracted simulated data.}
%\label{fig:Slagen-kart}
%\end{figure}

\begin{figure}[ht]
\centerline{
\includegraphics*[trim=3cm 0cm 3cm 0cm,clip=true,width=\textwidth]{Figurer/Slagen_tid}}
\caption{\small
Timeseries of observed and simulated velocity magnitudes at Slagen.}
\label{fig:Slagen-tid}
\end{figure}

\begin{figure}[ht]
\centerline{
\includegraphics*[trim=2cm 1cm 1cm 0cm,clip=true,height=4cm]{Figurer/Slagen_Rose_obs} 
\includegraphics*[trim=2cm 1cm 3cm 0cm,clip=true,height=4cm]{Figurer/Slagen_Rose_sim}
}
\caption{\small
Current roses for observed (left) and simulated (right) velocity magnitude at the two depths from 1st of October 2014 to 1st of October 2015.}
\label{fig:Slagen-rose}
\end{figure}

\begin{figure}[t]
\centerline{
\includegraphics*[trim=2cm 0cm 2cm 0cm,clip=true,width=\textwidth]{Figurer/Slagen_pdf} 
}
\caption{\small
Probability density functions of velocities and directions at Slagen for 1st of October 2014 to 1st of October 2015. The bin width is 0.01 knots for velocity and 3 degrees for direction.}
\label{fig:Slagen-pdf}
\end{figure}

\begin{table}[ht]
%\vspace{-1.5cm}
\caption{Yearly maximum observed velocity at Slagen.}
\label{tab:Slagen_max}
\centering
\begin{tabular}{|l|lll|lll|}
\hline 
& \multicolumn{3}{|l|}{\bf Max. velocity at 10m depth} & \multicolumn{3}{|l|}{\bf Max. velocity at 2.5m depth} \\
{\bf Year} & {\bf Date} & {\bf [m/s} & {\bf [deg]} & {\bf Date} & {\bf [m/s]} & {\bf [deg]} \\ \hline 
\small 2006 & 21 Jan 2006 & 0.42 & 139 & 31 Oct 2006 & 0.57 & 140 \\
\small 2007 & 14 Jan 2007 & 0.42 & 172 & 21 Aug 2007 & 1.03 & 359 \\
\small 2008 & 22 Mar 2008 & 0.36 & 149 & 19 Dec 2008 & 0.57 & 160 \\
\small 2009 & 17 Dec 2009 & 0.45 & 142 & 24 Mar 2009 & 0.56 & 139 \\
\small 2010 & 09 Nov 2010 & 0.41 & 138 & 09 Nov 2010 & 0.54 & 138 \\
\small 2011 & 01 Jan 2011 & 0.39 & 146 & 30 Mar 2011 & 0.62 & 185 \\
\small 2012 & 05 Dec 2012 & 0.39 & 138 & 29 May 2012 & 0.57 & 140 \\
\small 2013 & 10 Oct 2013 & 0.42 & 143 & 10 Oct 2013 & 0.49 & 144 \\
\small 2014 & 18 Apr 2014 & 0.44 & 147 & 26 Mar 2014 & 0.55 & 143 \\
\small 2015 & 24 Jan 2015 & 0.33 & 128 & 21 Mar 2015 & 0.55 & 141 \\
\hline
\end{tabular}
\end{table}

\begin{figure}[ht]
\centerline{
\includegraphics*[trim=0cm 0cm 0cm 0cm,clip=true,width=0.8\textwidth]{Figurer/Slagen_QQ}}
\caption{\small
Combined QQ- and scatter plot of observed and simulated current at Slagen from 1st of October 2014 to 1st of October 2015.}
\label{fig:Slagen_QQ}
\end{figure}

Time series reveal that the observed velocities varies and follows no striking pattern (Fig.~\ref{fig:Slagen-tid}).
Current roses show that both the observed and the simulated velocities are stronger in the upper layer (Fig.~\ref{fig:Slagen-rose}). The simulated velocities are stronger than the observed velocities. This is in accordance with the probability density functions (Fig.~\ref{fig:Slagen-pdf}). The yearly maximum observed velocities are approximately 0.4 and 0.6 m/s at 10 and 2.5 meters depth respectively (Tab.~\ref{tab:Slagen_max}). During 2014 and 2015 maximum observed velocity at 2.5 meters depth was 0.55 m/s in southeast direction (143$^o$N) the 26th of March 2014. The velocity at 10 meters depth was 0.08 m/s (153$^o$N) at the time of maximum velocity at 2.5 meters depth indicating that the velocities are different in the two layers.

The mean directions are to the south east. At approximately 2.5 meters depth the mean directions are 146$^o$N and 139$^o$N for observed and simulated directions respectively which is in fairly good agreement. At approximately 10 meters depth the observed mean direction shifts to 170$^o$N while the simulated mean direction is 148$^o$N. 
Testing with popcorn indicate that the preferred direction of the surface currents are towards Bliksekilen located west of the Slagen Refinery. This is not the case neither in the observations nor the simulations.
The probability density functions reveals that the model captures the distribution of directions in the upper layer, but does not capture the change in direction between the two depths (Fig.~\ref{fig:Slagen-pdf}). The standard deviations at 2.5 and 10 meters are 55 and 66 degrees respectively for the observed directions, and 56 and 61 for the simulated directions.

The time series scatter plots reveal that the correlation in time is not satisfying (Fig.~\ref{fig:Slagen-tid} and \ref{fig:Slagen_QQ}). The model seem to have difficulties with capturing the right phenomena influencing the currents to the right time. This is a well known problem when it comes to forecasting currents. The QQ-plots also confirms that the simulated currents are stronger than the observed currents. 


\clearpage

\subsection{Hydrography}

\subsubsection{CTD-measurements}
Simulated temperature and salinity profiles at the CTD positions indicated in Fig.~\ref{fig:kart_obs} are extracted for the points in time the observations were performed. 

During the summer, the water in the upper layers are heated. The maximum surface temperature is observed towards the end of the summer. The profiles in the outer parts of the fjord indicate that the upper layer is too thin in the simulated data (Fig.~\ref{fig:CTD_OF-1}). At larger depths, the water is too cold and the salinity is too high. This might indicate that the open boundary input and the representation of vertical mixing should be modified. 

Some of the observations are taken from smaller fjord branches, such as in the Larviksfjord close to the open boundary in the outer part of the Oslofjord. In such shallow waters observations reveal that the whole water column is heated during summer and cooled during winter, but the simulated temperature varies only in the upper 20 meters (Fig.~\ref{fig:CTD_OF-1}). 

The threshold at Svelvik in the Drammensfjord (western branch of the Oslofjord) is only 12 meters deep and 180 meters wide. The threshold causes formation of salinity water below the depth of the threshold north of the threshold. The observed and simulated profiles reveal that the simulated salinity in this area is too low. This might indicate a problem with the initialisation of the model. The simulated temperature profiles are more in accordance with the observed profiles, but the temperatures at the lower layers are 1-2 degrees too high.

\begin{figure}[tbh]
\centerline{
\includegraphics*[trim=0cm 0cm 0cm 0cm,clip=true,width=\textwidth]{Figurer/CTD_OF-1}}
\caption{\small
Observed (solid) and simulated (dashed) salinity and temperature profiles at station OF-1 Torbj{\o}rnskj{\ae}r in the outer part of the Oslofjord. All profiles are from 2015.}
\label{fig:CTD_OF-1}
\end{figure}

\begin{figure}[tbh]
\centerline{
\includegraphics*[trim=0cm 0cm 0cm 0cm,clip=true,width=\textwidth]{Figurer/CTD_LA-1}}
\caption{\small
Observed (solid) and simulated (dashed) salinity and temperature profiles at station LA-1 Larviksfjord in a fjord branch in the outer part of the Oslofjord. All profiles are from 2015.}
\label{fig:CTD_LA-1}
\end{figure}

\begin{figure}[tbh]
\centerline{
\includegraphics*[trim=0cm 0cm 0cm 0cm,clip=true,width=\textwidth]{Figurer/CTD_D-2}}
\caption{\small
Observed (solid) and simulated (dashed) salinity and temperature profiles at station D-2 Inner Drammensfjord. All profiles are from 2015.}
\label{fig:CTD_D-2}
\end{figure}

\clearpage 

\subsubsection{Water temperature near \AA sg\aa rdstrand}

The temperature observations from Scanmar AS are compared with simulated data extracted from 1.15 meters depth at approximately the same location as the observations.

Time series reveal that the simulated and observed temperature are in fairly good agreement (Fig.~\ref{fig:temp_2015}). During winter and spring, but the model underestimate the temperature in the summer and fall with a few degrees  and \ref{fig:temp-QQ_scatter}). The model captures the timing of the daily variations in temperature, but seems to overestimate heating and cooling causing too large daily variations (Fig.~\ref{fig:temp_jun2015}). 

2014 had a warmer summer than 2015. This is evident in both the observations and the simulations (Tab.~\ref{tab:temp}). The summer months of 2014 also had the largest variance in both the observed and the simulated temperature during 2014 and 2015. Generally, the simulated monthly temperature had a larger variance than the observed monthly temperature. The mean of the observed and simulated temperatures are 10.0$^o$C and 8.6$^o$C respectively, while the variances are 27.2$^o$C and 21.1$^o$C respectively.  

\begin{figure}[ht]
%\centerline{
%\includegraphics*[trim=0cm 0cm 0cm 0cm,clip=true,width=\textwidth]{Figurer/Temperatur_2014}}
\centerline{
\includegraphics*[trim=0cm 0cm 0cm 0cm,clip=true,width=\textwidth]{Figurer/Temperatur_2015}}
\caption{\small
Time series of observed and simulated temperature at \AA sg\aa dstrand. The difference is smoothed over 10 days.}
\label{fig:temp_2015}
\end{figure}

\begin{figure}[htb]
\centerline{
\includegraphics*[trim=1cm 0cm 1cm 0cm,clip=true,width=0.7\textwidth]{Figurer/Temperatur_QQ_scatter}}
\caption{\small
Combined QQ- and scatter plot of observed and simulated temperature at \AA sg\aa dstrand.}
\label{fig:temp-QQ_scatter}
\end{figure}

\begin{figure}[htb]
\centerline{
\includegraphics*[trim=0cm 0cm 0cm 0cm,clip=true,width=\textwidth]{Figurer/Temperatur_jun2015}}
\caption{\small
Time series of observed and simulated temperature at \AA sg\aa dstrand in June 2015.}
\label{fig:temp_jun2015}
\end{figure}

\newpage 

\begin{table}
\caption{Monthly statistics for observed and simulated temperature at \AA sg\aa rdstrand.}
\label{tab:temp}
\centering
\begin{tabular}{|ll|rrr|rrr|}
\hline 
&& \multicolumn{3}{|c|}{\bf 2014} & \multicolumn{3}{|c|}{\bf 2015} \\
&& {\bf quantity} & {\bf mean} & {\bf variance}  
& {\bf quantity} & {\bf mean} & {\bf variance} \\ \hline 
\small Jan & obs & 558 & 2.5 & 6.2 & 558 & 4.9 & 2.5 \\
\small     & sim & 745 & - & - & 745 & 4 & 0.7 \\
\small Feb & obs & 504 & 1.8 & 0.5 & 504 & 4 & 1.8 \\
\small     & sim & 673 & - & - & 673 & 3.6 & 0.6 \\
\small Mar & obs & 496 & 3.7 & 0.6 & 496 & 4.2 & 0.4 \\
\small     & sim & 745 & - & - & 745 & 4.7 & 0.3 \\
\small Apr & obs & 515 & 7.5 & 4.1 & 515 & 7 & 2.1 \\
\small     & sim & 721 & 7.6 & 7.1 & 721 & 7.4 & 2.8 \\
\small May & obs & 544 & 12.1 & 9.3 & 544 & 10.3 & 1.3 \\
\small     & sim & 745 & 11.3 & 10.4 & 745 & 9.9 & 3.5 \\
\small Jun & obs & 531 & 16.2 & 4.2 & 531 & 14 & 3.9 \\
\small     & sim & 721 & 15.5 & 8.3 & 721 & 12.8 & 6 \\
\small Jul & obs & 558 & 20.2 & 10.5 & 558 & 17.1 & 2 \\
\small     & sim & 745 & 17 & 24.3 & 745 & 14.6 & 5.1 \\
\small Aug & obs & 512 & 20 & 3.2 & 512 & 18.2 & 1.3 \\
\small     & sim & 745 & 16.4 & 5.6 & 745 & 15.5 & 8.7 \\
\small Sep & obs & 536 & 16.3 & 1.9 & 536 & 15.1 & 1.2 \\
\small     & sim & 721 & 12.9 & 8.9 & 721 & 12.7 & 2.5 \\
\small Oct & obs & 557 & 12.3 & 1.7 & 557 & 10.6 & 0.8 \\
\small     & sim & 745 & 9 & 1.5 & 745 & 8.6 & 2.7 \\
\small Nov & obs & 540 & 8.3 & 2.9 & 540 & 8.8 & 2.4 \\
\small     & sim & 721 & 6.3 & 2 & 721 & 6.1 & 1.3 \\
\small Dec & obs & 490 & 4.4 & 3.6 & 490 & 7.8 & 0.4 \\
\small     & sim & 733 & 3.4 & 1.5 & 733 & 4.8 & 0.8 \\
\hline
\end{tabular}
\end{table}

\clearpage

\subsubsection{Water temperature in the Inner Oslofjord}

The observed and simulated temperature at three beaches in the Inner Oslofjord are in relatively good agreement (Fig.~\ref{fig:badetemp_2014} - \ref{fig:badetemp_2015}). 

Close to the shoreline and only 40 cm under the surface, the temperature is heavily influenced by the weather situation and local circulation patterns. 

The temperature differences during the day are larger in the model than in the observations. The observed temperature increases 1-3 degrees from 09:00 to 18:00 and is not measured during the night, while the modelled temperature increases up to six degrees from 06:00 to 23:00. The fact that temperature is not measured during the night, but only from 09:00 to 18:00, might explain differences i temperature rise during the day, but the difference might indicate too much heating in the model.

%Since the temperature is observed close to the shoreline, some near river outlets, and only 40 cm under the surface, the temperature ise heavily influenced by the weather situation and local circulation patterns. The model is not expected to capture such detailed effects. Still there are similarities between the modelled and the observed temperatures both in temperature level and in fluctuations. 

During the summer 2014 the model predicts higher temperatures at Sj\o strand than was observed. The observations in Hvalstrand have some of the same trends as the modelled temperature with temperatures up to 25 degrees. The air temperatures in 2014 was higher than in 2015 and resulted in higher water temperatures, especially in shallow areas. 

\begin{figure}[ht]
\centerline{
\includegraphics*[trim=0 0 0 0,clip=true,width=\textwidth]{Figurer/badetemp_2014}
}
\caption{\small
The observed and modelled temperature at three beaches in the Inner Oslofjord during the summer 2014}
\label{fig:badetemp_2014}
\end{figure}

\begin{figure}[ht]
\centerline{
\includegraphics*[trim=0 0 0 0,clip=true,width=\textwidth]{Figurer/badetemp_2015}
}
\caption{\small
The observed and modelled temperature at three beaches in the Inner Oslofjord during the summer 2015}
\label{fig:badetemp_2015}
\end{figure}


\subsubsection{Ferrybox - Andr\'{e}?}
To be continued... Andr\'{e}?

\begin{figure}[ht]
\centerline{
\includegraphics*[trim=1cm 0cm 1cm 0cm,clip=true,height=10cm]{Figurer/FjordOs_with_FA_track}}
\caption{\small
The track of Color Fantasy.}
\label{fig:Ferrybox_track}
\end{figure}

\begin{figure}[ht]
\centerline{
\includegraphics*[trim=1cm 0cm 1cm 0cm,clip=true,width=.5\textwidth]{Figurer/FjordOs_vs_Ferrybox_TEMP}
\includegraphics*[trim=1cm 0cm 1cm 0cm,clip=true,width=.5\textwidth]{Figurer/FjordOs_vs_Ferrybox_SALT}}
\caption{\small
Simulated daily mean compared with observations from ferryboxes for temperature (left) and salinity (right).}
\label{fig:Ferrybox_temp_salt}
\end{figure}


\clearpage 

\subsection{Drifting lanes}

De to drifterne i f\o rste tokt

Oppsummer v\ae rforhold og oljedriften etter Godefoss. Sammenlign med lignende tilfelle i 2014/15, samt alternative drivbaner under andre forhold.

Skill-score for drifters

\clearpage 
\subsubsection{Godafoss}
Our model simulation does not cover the time period when Godafoss ran aground, so we have instead released drifters in our model from the position of the oilspill from Godafoss for a period of one year. We think that over such a long period of time, there will be at least one weather and current situation similar to the weather and currents experienced during the Godafoss release. So by showing the models ability to advect oil to the observed locations during a one year run, we at least show that our ocean model is capable of recreating the right drift patterns.

To simulate the drift of the oil, we have applied the open source trajectory-model OpenDrift. This is a trajectory model under development at MET Norway, and is described by its developers as "a software for modeling the trajectories and fate of objects or substances drifting in the ocean, or even in the atmosphere". It is distributed under a GPL v2.0 license, and is available on GitHub\footnote{https://github.com/knutfrode/opendrift}.

The OpenDrift model was forced with currents from the FjordOs model and with wind from the Arome-MetCoOp 2.5km (Arome2.5) atmospheric model (the same atmospheric model was used as forcing when running the FjordOs hindcast). We have also provided daily mean currents from NorKyst-800m outside of the FjordOs-model, to properly treat the particles that are advected out of the FjordOs-model, so they can re-enter at the correct location. We can tune a number of parameters when running OpenDrift, e.g. random walk and the wind drift factor. Random walk was not used in our simulations, and wind drift factor was set to 0.01 (i.e. 1\%). Modelled drifters has been released once per hour from April 1st 2015 to April 1st 2016, a total number og 8760 particles. The lifetime of each particle is set to 15 days, i.e. after 15 days the particle is deactivated. This is done to reduce the computational cost of advecting a large number of particles. 

\begin{figure}[ht]
\centerline{
\includegraphics*[width=.5\textwidth]{Figurer/opendrift/opendrift_godafoss_shortest_time_crop}
\includegraphics*[width=.5\textwidth]{Figurer/opendrift/opendrift_godafoss_consentration_crop}
}
\centerline{
\includegraphics*[width=.5\textwidth]{Figurer/opendrift/opendrift_godafoss_shortest_time_zoom_crop}
\includegraphics*[width=.5\textwidth]{Figurer/opendrift/opendrift_godafoss_consentration_zoom_crop}
}
\caption{\small
Number of hours from particle release, to particle in given area (left panels), and the number of particles that has been inside a given 140x140m area (right panel). Godafoss. Based on one year (April 1st 2015 - April 1st 2016) of simulations, with a maximum lifetime of 15 days og the released particles. This amounts to a total number of 8760 released particles. Please note the different scales of each figure.}
\label{fig:opendrift_godafoss1}
\end{figure}

\begin{figure}[ht]
\centerline{
\includegraphics*[width=.5\textwidth]{Figurer/opendrift/opendrift_godafoss_shortest_time_zoom_endpos_crop}
\includegraphics*[width=.5\textwidth]{Figurer/opendrift/opendrift_godafoss_consentration_zoom_endpos_crop}
}
\caption{\small
For endposition of each trajectory: Number of hours from particle release, to particle in given area (left panels), and the number of particles that has been inside a given 140x140m area (right panel). Godafoss. Based on one year (April 1st 2015 - April 1st 2016) of simulations, with a maximum lifetime of 15 days og the released particles. This amounts to a total number of 8760 released particles. Please note the different scales of each figure.}
\label{fig:opendrift_godafoss2}
\end{figure}

\clearpage 


\clearpage 

\clearpage
\section{Summary and final remarks}
\label{sec:summa}
Considered is the performance of the FjordOs model, a new circulation model covering the Oslofjord, Norway. The FjordOs model is a version of the Regional Ocean Modeling System (ROMS) as documented by \cite{haidv:etal:2008} and \cite{shche:mcwil:2003, shche:mcwil:2005, shche:mcwil:2009}. It is adapted for Oslofjord by utilizing its curvilinear option as detailed in \cite{roed:etal:2016}. The model is developed to improve the ocean input (e.g., currents) to emergency models used to predict pathways of oil and/or other effluents. The utilization of the curvilinear option in ROMS was chosen to increase the resolution without inflating the computer demand. 

The model has earlier been assessed to evaluate its representation of the tidal elevations \citep{hjelm:etal:2017}. They found that the tidal elevation is well represented in the model. This finding is underscored by the present study, in which we focus on an evaluation of the model's rendition of the circulation. To this end we compare model results from a near two-year long simulation to observations. The observations encompass water level, currents and temperature at various time periods at fixed stations, and not least observed trajectories of drifters. 

In essence currents in the Oslofjord are composed of tidal currents, wind forced currents, currents induced by storm surge events, and currents due to differences in density. The last component is commonly caused by differences in temperature and salinity. It is emphasized that the tides are more often than not the most dominant current in the fjord. The evaluation reveals that the model is not perfect. While the tidal currents are well represented, the currents due to difference in density appears to be less well represented. We find that this lack of success is probably associated with the model's failure in representing a realistic stratification (or baroclinicity). One possibility for the latter weakness may be associate with the use of the coarser mesh model NorKyst800 to initialize the model and the fact that the FjordOs model is forced by the NorKyst800 at its southern open boundary bordering on the Skagerrak throughout the simulation. Thus any lack of success in representing the stratification in the NorKyst800 model will also be reflected in the FjordOs model. Another possibility is the vertical mixing in the model. For instance it appears that the vertical mixing inside of the sill in the Drammensfjord is too vigorous, while the vertical mixing in the outer part of the fjord is too weak. 

Nevertheless we argue that its performance is adequate for its purpose. An important justification is that the higher resolution offers a decrease in the number of stranded trajectories compared to models of coarser resolution. Also of importance is that the model provides a realistic representation of the tidal currents and to certain degree the current depth profiles.





\section*{\hspace{17mm}Acknowledgements}
\addcontentsline{toc}{section}{Acknowledgements}
The present study is funded in parts by the Regional Research Fund Oslofjordfondet through Research grant no. 226022, and through smaller contributions from MET Norway, NIVA, HSN, Kystverket, and ExxonMobil. We gratefully acknowledge the support. 

We also like to thank the crew and captain of the R/V Trygve Braarud for excellent support during the cruises and for enduring all our requests. Moreover we would like to thank ExxonMobil for making their acoustic Doppler measurements outside of Slagentangen available to us. Likewise we thank NIVA who provided the CTD data, Norwegian Mapping Authority for available observations of water level, Scanmar AS for temperature measurements near {\AA}sg{\aa}rdstrand, and Finnerud Elektronikk for making the beach temperatures at Stor{\o}yodden, Hvalstrand, and Sj{\o}strand available to us. Finally we thank NIVA and Statnett for providing the data from the four acoustic Doppler profiling measurements across the fjord from Sm{\aa}skj{\ae}r to Evje, and from the two stations at Filtvedt and Brenntangen. 

\clearpage
\section*{\hspace{17mm}Appendix}
\addcontentsline{toc}{section}{Appendix}
\section*{Driftling lanes from the Slagen refinery}
A part of the Fjordos project has been to examine how known, and potential, oilspills would spread out in the Oslofjord. The Slagen refinery at Slagentangen is owned by ExxonMobil, and is described as following on their webpage\footnote{\url{http://www.exxonmobil.no/en-NO/company/operations/operating-locations/slagen-refinery?sc_lang=en-NO}, 26.01.2017}: "Slagen Refinery is situated on the west bank of the Oslofjord about 5 n. miles south of Horten. The marine terminal consists of a pier about 500 m long with loading/discharging berths on both sides. To the south of the long pier there is a small harbour where mooring boats and oil recovery equipment are kept. The terminal and its near surroundings are owned and controlled by Esso Norge AS. It has its own Harbour Office with Marine Supervisors on duty 24 hours a day." and "The Slagen Marine Terminal has approximately 800 tanker calls a year with size variation 100 to 250 000 DWT. The annual import of Crude oil (mainly from the Northsea) and Blendstock is about 6.5 mill. $m^3$ and about 5.7 mill. $m^3$ petroleum products are shipped out."

To model the spread of oil from a potential spill at Slagentangen we have used the same approach as in Section \ref{sect:godafoss_model}. It is very important to point out that the work done in this appendix IS NOT SUFFICIENT to be used for any contingency planning or other work on possible oil spill scenarioes. It should be viewed as a "teaser" on possible future work that could be used for contingency planning of unwanted releases of substances from anywhere within the Oslofjord.

Figure \ref{fig:opendrift_slagen1} show the release position of the particles used in the simulation marked with a black dot.  The left panel show the shortest number of hours from the time of the particle release to a particle is in a given position. The right panel show the corresponding consentrations. When viewing the zoomed in figures (lower figures), it is evident that the coastline of the ocean model does not perfectly match the real coastline, and also the OpenDrift model will at times advect particles onto land. This is linked to the length of the timestep used in OpenDrift. A more thorough work on potential oil spills should adress  these issues.
When looking at the area closest to the release position, it clearly shows that most particles are transported towards the southeast, and this is also the direction of which the particles are transported fastest. This compares well to the observations of currents outside Slagen in Figure \ref{fig:Slagen-rose}.
Figure \ref{fig:opendrift_slagen2} show the end positions of each trajectory, and the corresponding shortes time from release to that position, and consentration.

This sort of maps could be made and categorized by weather pattern or another known factor, and in turn e.g. be used in the unlikely event of a spill in the time between the spill happening, and the forecast of oildrift is recieved. 



\begin{figure}[ht]
\centerline{
\includegraphics*[width=.5\textwidth]{Figurer/opendrift/opendrift_slagen_shortest_time_crop}
\includegraphics*[width=.5\textwidth]{Figurer/opendrift/opendrift_slagen_consentration_crop}
}
\centerline{
\includegraphics*[width=.5\textwidth]{Figurer/opendrift/opendrift_slagen_shortest_time_zoom_crop}
\includegraphics*[width=.5\textwidth]{Figurer/opendrift/opendrift_slagen_consentration_zoom_crop}
}
\caption{\small
Number of hours from particle release, to particle in given area (left panels), and the number of particles that has been inside a given 140x140m area (right panel). Slagentangen. Based on one year (April 1st 2015 - April 1st 2016) of simulations, with a maximum lifetime of 15 days og the released particles. This amounts to a total number of 8760 released particles. Please note the different scales of each figure.}
\label{fig:opendrift_slagen1}
\end{figure}

\begin{figure}[ht]
\centerline{
\includegraphics*[width=.5\textwidth]{Figurer/opendrift/opendrift_slagen_shortest_time_zoom_endpos_crop}
\includegraphics*[width=.5\textwidth]{Figurer/opendrift/opendrift_slagen_consentration_zoom_endpos_crop}
}
\caption{\small
For endposition of each trajectory: Number of hours from particle release, to particle in given area (left panels), and the number of particles that has been inside a given 140x140m area (right panel). Slagentangen. Based on one year (April 1st 2015 - April 1st 2016) of simulations, with a maximum lifetime of 15 days og the released particles. This amounts to a total number of 8760 released particles. Please note the different scales of each figure.}
\label{fig:opendrift_slagen2}
\end{figure}




%\begin{table}[h]
%\vspace{-1.5cm}
%{\bf Table A1: ESD statistics.}\\
%\label{tab:1}
%\begin{tabular}{llll}
%\small OSLO - BLINDERN 18700 & 
%\small $T_m$ $R^2$= 71 -- 86 $T_x$ $R^2$= 69 -- 82 $T_n$ $R^2$= 49 -- 76 \\
%\small $\Delta T_m$ $q_{0.05}$= 3.21 $q_{0.50}$= 1.33 $q_{0.95}$= 1.75 \\  
%\small $\Delta T_x$ $q_{0.05}$= 3.09 $q_{0.50}$= 1.38 $q_{0.95}$= 2.16 \\  
%\small $\Delta T_n$ $q_{0.05}$= 3.15 $q_{0.50}$= 1.12 $q_{0.95}$= 1.01 \\
%\end{tabular}
%\end{table}

\clearpage
\pagebreak
\addcontentsline{toc}{section}{References}

\bibliography{referanse}
%\begin{thebibliography}{1}
%\end{thebibliography}

\clearpage
\pagebreak


 

\end{document}
