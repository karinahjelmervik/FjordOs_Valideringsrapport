\section{Observations}

Model simulations and results are performed for the period April 2014 through December 2015. Relevant observations available to us for this period are shown in Fig.~\ref{fig:kart_obs}. They encompass water level, temperature and salinity profiles from CTD measurements, temperature measurements at fixed stations and profiles of currents from acoustic Doppler instruments. Note that most of these measurements are scattered in time as well as in space. The exceptions are the measurements of water level at the three stations Viker, Oscarsborg, and Oslo, and the the current measurements at Slagentangen which are gathered regularly in time and for much longer periods. The data are made available to us through various sources, and gathered independent of the FjordOs project. Through the FjordOs project we have performed cruises in the Oslofjord to obtain trajectories (or pathways) of drifters for comparison with simulated trajectories using the modelled currents from the simulations with the FjordOs model as input to MET Norway's operational drift model. Since the the aim of the FjordOs project is to provide a circulation model useful as input to drift models of effluents to the fjord, the latter evaluation is perhaps the most important one to perform.

\begin{figure}[htb]
\centerline{
\includegraphics*[trim=0cm 0.8cm 0cm 0cm,clip=true,width=0.8\textwidth]{Figurer/kart_obs}
}
\caption{\small
Displayed are the names of and positions at which observations are available. Dark purple solid circles correspond to temperature stations, green triangles to CTD stations, blue squares to water level stations, while red diamonds indicate positions for which profiles of currents from acoustic Doppler instruments are available.
}
\label{fig:kart_obs}
\end{figure}

\newpage

\subsection{Water level}
The Norwegian Mapping Authority has three permanent stations measuring sea level in the Oslofjord (Fig.~\ref{fig:kart_obs}), namely Viker, Oscarsborg and Oslo. The station at Viker is located close to the open boundary of the model area and is used to adjust the tidal input \cite{hjelm:etal:2017}. The station at Oscarsborg is located halfway in the Inner Oslofjord, while the station at Oslo is situated in the innermost part of the fjord in the Oslo Harbour.

% % % % % % % % % % % % % % % % % % % % % % % % % % % %
%\clearpage
\subsection{Currents}

%\subsubsection{Currents in two cross sections}
In the period from the 15th to the 18th of September 2014 altogether six rigs fitted with Acoustic Doppler Current Profilers (ADCPs) were deployed at seven positions in the Oslofjord (Tab.~\ref{tab:Statnett} and Fig.~\ref{fig:kart_obs}). They were recovered in the period from the 16th through the 25th of November. Two of the rigs were deployed at the stations named Filtvedt and Brenntangen at the entrance to the Dr{\o}bak Sound, while the remaining four were deployed further south at the stations named Sm{\aa}skj{\ae}r, Laksetrappa, Botnegrunnen, and Evje across the fjord eastward and slightly north of {\AA}sg{\aa}rdstrand. Note that the latter forms a cross section located between two deep basins (Fig.~\ref{fig:kart_obs}).  These six rigs were all deployed as a part of a project conducted by Statnett, NIVA, Akvaplan NIVA, and the University of Oslo, but were made available for the FjordOs project. The research vessel R/V Trygve Braarud was used during deployment and recovery of the rigs. For further details about the applied rigs and corresponding instruments the reader is referred to \cite{staalstrom:2015}.

Data from an addtional bottom-mounted ADCP, named Slagen (Fig.~\ref{fig:kart_obs}), and made available to us by Exxonmobil, Slagentangen, is also used for comparison with model results. It has measured currents regularly in two depths since 1997, and is located 50-80 meters northwest of Turning Dolphin at the Slagen Refinery (Fig.~\ref{fig:Slagen-kart}). Note that Bliksekilen nature reserve is located west of Slagen Refinery, which is a shallow water area with rare flora and fauna. 

The strongest currents were found at the two northernmost stations at the entrance to the Dr{\o}bak Sound, namely Filtvedt and Brenntangen. Here the fjord is relatively narrow, and hence the somewhat stronger current may be explained by a stronger tidal signal due to the tapering of the width of the fjord at the entrance to the Dr{\o}bak Sound. We also note that the stations with the smallest depths exhibits the highest current velocities. This may be explained by the fact that the horizontal pressure gradient has a tendency to decrease with depth.

%At one station (Botnegrunnen) a single point current meter was deployed just below the profiling current meter.while the 



\begin{table}[ht] 
\caption{Target positions (WGS84) of the instrument rigs. Depths at the stations are from the Statnett terrain model. Note that these may differ from the model depth at the same location.} 
\label{tab:Statnett} 
\centering 
\begin{tabular}{|llcccl|} 
\hline  
{\bf Station} & {\bf Name} & {\bf Latitude} & {\bf Longitude} & {\bf Depth} & {\bf Instruments} \\ 
&& {\bf [$^o$N]} & {\bf [$^o$E]} & {\bf [m]} & \\ \hline
Kp11.2 & Sm\aa skj\ae r & 59.350124 & 10.497661 & 20 & Aquadopp600 AQP1531 \\
(Ri1) &&&&& Transducer LRT2 \\ \hline
Kp5.7 & Laksetrappa & 59.343452 & 10.581023 & 75 & Aquadopp400 AQP4689 \\
(Rl1) &&&&& Transducer LRT3 \\
      &&&&& Aanderaa Seaguard \\  \hline
Kp2.6 & Botnegrunnen & 59.352375 & 10.626822 & 96 & Continental WAV6117 \\
(Rm1) &&&&& Transducer LRT4 \\ \hline
Kp0.7 & Evje & 59.363182 & 10.653576 & 64 & Aquadopp400 AQP2931 \\
(Rn1) &&&&& Transducer LRT5 \\ \hline
Kn2 & Brenntangen & 59.581803 & 10.646087 & 54 & Aquadopp400 AQP5608 \\
      &&&&& Transducer LRT6 \\ \hline
Km1 & Filtvedt & 59.582064 & 10.627372 & 153 & Continental CNL6037 \\
      & (current) &&&& Transducer 207-2 \\ \hline
Km2 & Filtvedt & 59.580778 & 10.626239 & 125 & 7 TinyTags UIO1-7 \\
      & (temperature) &&&& Transducer 203-2 \\
\hline
\end{tabular}
\end{table}

\clearpage

%\subsubsection{Currents at Slagentangen}
%Using a bottom-mounted Doppler, Exxonmobil has measured the currents in two depths since 1997. The device is placed 50-80 meters northwest of Turning Dolphin at the Slagen Refinery (Fig.~\ref{fig:kart_obs} and \ref{fig:Slagen-kart}). Note that Bliksekilen nature reserve is located west of Slagen Refinery, which is a shallow water area with rare flora and fauna. 

\begin{figure}[htb]
\centerline{
\includegraphics*[trim=0cm 0cm 1cm 0cm,clip=true,width=0.8\textwidth]{Figurer/Slagen_kart}}
\caption{\small
Map of Slagen Refinery. The red dot marks the position corresponding to the extracted simulated data. Souce: Norwegian Coastal Administration}
\label{fig:Slagen-kart}
\end{figure}

% % % % % % % % % % % % % % % % % % % % % % % % % % % %
\clearpage
%\subsection{Hydrography}
\subsection{CTD measurements}

%\subsubsection{CTD measurements}

As part of a program to monitor the eutrophication state of the Outer Oslofjord and the Drammensfjord NIVA, funded by Fagr{\aa}det for Ytre Oslofjord, collects CTD measurements at selected positions. These measurements include profiles of temperature and salinity as well as water quality parameters, and are available through a web portal\footnote{http://www.aquamonitor.no/ytreoslofjord/}. Between April 2014 and December 2015 CTD profiles was measured 12 times. Most of the measurements are from the months January, February, June, July, August, September and November. Thus data are lacking in spring and early summer. We use data from 10 of these CTD stations as listed by Table ~\ref{tab:CTD_pos}, and depicted as green triangles in Fig.~\ref{fig:kart_obs}. 

We note that seven of the  stations are located inside narrow straits and sounds, or inside lesser subfjords or inlets. Of the remaining three stations only two of them are in the open parts of the fjord, namely Torbj{\o}rnskj{\ae}r (OF-1) and Breiangen (OF-5), while the last is located where the western branch of the river Glomma discharges into the fjord ({\O}F-1).  

\begin{table}
\caption{The position of the 10 CTD measurement sites used in this study showing the coordinates (latitude, longitude) and number of measurements at each site.} 
\label{tab:CTD_pos} 
\centering 
\begin{tabular}{|clcccc@{}c|} 
\hline  
{\bf Tag} & {\bf Station} & {\bf Latitude} & {\bf Longitude} & \multicolumn{2}{c}{\bf Number of measurements} &\\ 
&& {\bf [$^o$N]} & {\bf [$^o$E]} & 2014 & 2015 &\\ \hline
D-2 & Inner Drammensfjord & 59.6280 & 10.4210 & 5 & 7 &\\ 
%16.06.2014 & 16.01.2015 \\
%&&&& 05.07.2014 & 05.02.2015 \\
%&&&& 18.08.2014 & 17.06.2015 \\
%&&&& 29.09.2014 & 06.07.2015 \\
%&&&& 15.11.2014 & 14.08.2015\\
%&&&& & 27.09.2015 \\
%&&&& & 12.11.2015 \\ \hline 
D-3 & Solumstrand & 59.7060 & 10.3140 & 5 & 6 &\\ \hline
%16.06.2014 & 16.01.2015 \\
%& Solumstrand &&& 05.07.2014 & 17.06.2015 \\
%&&&& 18.08.2014 & 06.07.2015 \\
%&&&& 29.09.2014 & 14.08.2015 \\ 
%&&&& 15.11.2014 & 27.09.2015 \\
%&&&& & 12.11.2015 \\
LA-1 & Larviksfjord & 59.0190 & 10.0520 & 5 & 7 &\\ 
%14.06.2014 & 17.01.2015 \\
%&&&& 03.07.2014 & 04.02.2015 \\
%&&&& 16.08.2014 & 16.06.2015 \\
%&&&& 24.09.2014 & 07.07.2015 \\
%&&&& 13.11.2014 & 13.08.2015 \\
%&&&& & 23.09.2015 \\
%&&&& & 10.11.2015\\ \hline
MO-2 & Kippenes & 59.4840 & 10.6780 & 5 & 7 &\\ \hline
%16.06.2014 & 15.01.2015 \\
%&&&& 04.07.2014 & 05.02.2015 \\
%&&&& 17.08.2014 & 16.06.2015 \\
%&&&& 28.09.2014 & 05.07.2015 \\
%&&&& 15.11.2014 & 14.08.2015 \\
%&&&& & 26.09.2015 \\
%&&&& & 12.11.2015 \\ \hline
OF-1 & Torbj{\o}rnskj{\ae}r & 59.0410 & 10.7540 & 5 & 7 &\\ 
%OF-2 & Missingene & 59.1870 & 10.6920 & 0 & 14 &\\ 
OF-5 & Breiangen & 59.4870 & 10.4580 & 5 & 7 &\\ \hline
S-9  & Haslau, Singlefjord & 59.1140 & 11.1620 & 7 & 10 &\\ 
SF-1 & Sandefjord & 59.0770 & 10.2460 & 5 & 7 &\\ \hline
T{\O}-1 & Vestfjord & 59.2030 & 10.3550 & 5 & 7 &\\ 
{\O}-1 & Leira. Vesterelva & 59.1370 & 10.8340 & 7 & 10 &\\ \hline
\end{tabular}
\end{table}



\subsection{Temperature}
%\subsubsection{Water temperature near {\AA}sg{\aa}rdstrand}

Hourly temperature measurements at one meter depth over the last 10 years have been measured by Scanmar AS located three kilometres south of {\AA}sg{\aa}rdstrand. The device has an accuracy of $\pm 0.15^o$C in the range from -5 to +30$^o$C. 



%Note that 2014 had a warmer summer followed by a warm winter resulting in higher maximum and lower minimum than 2015 (Fig.~\ref{fig:Scan_temp}). 
% Bør kanskje skrive noe om fluktuasjonene her...

%\begin{figure}[htb]
%\centerline{
%\includegraphics*[trim=2cm 0cm 2cm 0cm,clip=true,width=\textwidth]{Figurer/temp_Scanmar}}
%\caption{\small
%Observed temperature measured by Scanmar AS}
%\label{fig:Scan_temp}
%\end{figure}

%\begin{table}[ht] 
%\caption{Observed water temperature near \AA sg\aa rdstrand} 
%\label{tab:Scan_temp} 
%\centering 
%\begin{tabular}{|cccccc|} 
%\hline  
%{\bf Year} & {\bf Minimum} & {\bf 5 percentile} & {\bf Mean} & {\bf 95 percentile} & {\bf Maximum} \\
%\hline
%\small 2005 & 1.0 & 4.0 & 13.2 & 20.7 & 24.9 \\
%\small 2006 & -6.0 & -0.5 & 10.1 & 20.6 & 23.9 \\
%\small 2007 & -1.2 & 1.1 & 9.7 & 18.3 & 21.4 \\
%\small 2008 & 0.1 & 2.6 & 10.4 & 19.3 & 24.6 \\
%\small 2009 & -2.3 & -0.3 & 9.4 & 19.8 & 24.9 \\
%\small 2010 & -1.7 & -0.2 & 8.7 & 18.7 & 20.5 \\
%\small 2011 & -1.4 & -1.0 & 9.6 & 19.0 & 22.6 \\
%\small 2012 & -1.0 & 0.4 & 9.2 & 18.5 & 21.4 \\
%\small 2013 & -1.3 & -0.4 & 9.3 & 19.4 & 22.2 \\
%\small 2014 & -1.1 & 1.2 & 10.6 & 21.7 & 26.4 \\
%\small 2015 & 0.7 & 3.5 & 10.5 & 18.7 & 20.8 \\
%\hline
%\end{tabular}
%\end{table}

%\newpage
%\subsubsection{Water temperature in the Inner Oslofjord}

In addition temperature measurements at three beaches in the Inner Oslofjord (Fig.~\ref{fig:kart_strand}), namely (from south to north) Sj{\o}strand, Hvalstrand and Stor{\o}yodden, which are performed in a cooperation between Asker and B{\ae}rum kommune, and Finnerud Elektronikk, is available to us. The measurement device are digital thermometers (Maxim Integrated DS18B20) with an accuracy of $\pm 0.5^o$C. They measure water temperature at 40 cm beneath the surface in locations where the water depths are several meters. Temperatures are measured every three hours from 09:00 to 18:00 during the summer months. Again we note that the site Sj{\o}strand is the only one located in a semi-open position, while the two other beaches are well within archipelagoes that are somewhat sheltered from the rest of the fjord.

Finally one rig at Filtvedt, namely rig Km2 listed in Tab.~\ref{tab:Statnett}, had TinyTag temperature loggers deployed at seven different depths between 20 and 120 m.

\begin{comment}
The trends of the observed temperatures at the three beaches are similar, but the temperature is generally lower at the southern beach, Sj{\o}strand, than at the northern beach, Stor{\o}yodden (Fig.~\ref{fig:temp_strand}). In 2014 there were two local maximuma during July, and the maximum observed temperatures in 2014 were higher than in both 2013 and 2015. The temperature increases 1-3 degrees during the day and decreases during the night.
\end{comment}

\begin{figure}[ht]
\centerline{
\begin{minipage}[l]{0.59\textwidth}
\includegraphics*[trim=0 0 0 1cm,clip=true,width=\textwidth]{Figurer/badestrand_kart}
\end{minipage}
\begin{minipage}[r]{0.4\textwidth}
\fbox{\includegraphics*[trim=1 0 0 3cm,clip=true,width=\textwidth]{Figurer/kart_Storoyodden.png}} \\
\fbox{\includegraphics*[trim=0 0 0 3cm,clip=true,width=\textwidth]{Figurer/kart_Hvalstrand.png}} \\
\fbox{\includegraphics*[trim=1 1cm 0 1.5cm,clip=true,width=\textwidth]{Figurer/kart_Sjostrand.png}} \\
\end{minipage}
}
\caption{\small
The positions at three beaches in the Inner Oslofjord where the temperature measurements are performed}
\label{fig:kart_strand}
\end{figure}

\begin{figure}[ht]
\centerline{
\includegraphics*[trim=2cm 0 2cm 0cm,clip=true,width=\textwidth]{Figurer/badetemp}
}
\caption{\small
Observed temperature at three beaches in the Inner Oslofjord}
\label{fig:temp_strand}
\end{figure}

%\begin{table}[ht]
%\caption{Mean observed temperatures at three beaches in the Inner Oslofjord. Only time periods with more than 8 days of observations during the given time period are included.}
%\label{tab:temp_strand}
%\begin{center}
%\begin{tabular}{|l|ccc|ccc|ccc|} \hline
%     & \multicolumn{3}{c|}{Stor\o yodden} & \multicolumn{3}{c|}{Hvalstrand} & \multicolumn{3}{c|}{Sj\o strand} \\ \hline
%Time period & 2013 & 2014 & 2015 & 2013 & 2014 & 2015 & 2013 & 2014 & 2015 \\ \hline
%%01 - 15 May & 0 & 0 & 0 & 0 & 0 & 0 & 0 & 0 & 0 \\ 
%16 - 31 May & 13.7 &  -   & 11.6 &  -   &  -   & 12.2 &  -   &  -   & 12.8 \\ 
%01 - 15 Jun & 15.0 & 19.0 & 13.3 & 15.9 & 19.4 & 13.7 &  -   & 19.9 & 14.6 \\ 
%16 - 30 Jun & 16.2 & 17.3 & 17.1 & 16.7 & 17.7 & 17.5 &  -   & 17.9 & 18.0 \\ 
%01 - 15 Jul & 17.8 & 17.6 & 18.2 & 18.3 & 18.3 & 19.8 & 19.4 & 19.3 &  -   \\ 
%16 - 31 Jul & 20.1 & 22.2 & 18.1 & 19.7 &  -   & 18.9 & 21.2 & 23.9 & 18.4 \\ 
%01 - 15 Aug & 19.5 & 21.3 & 18.0 & 19.8 & 21.3 &  -   & 20.6 & 21.7 & 18.8 \\ 
%16 - 31 Aug & 18.9 & 19.6 & 19.1 & 18.9 & 19.6 & 19.3 & 19.5 & 19.4 & 19.0 \\ 
%01 - 15 Sep & 17.9 & 18.3 & 16.1 & 17.7 & 18.6 & 16.5 & 18.1 & 18.9 & 16.2 \\ 
%16 - 30 Sep &  -   & 15.9 & 14.1 &  -   & 16.2 & 14.1 & 15.1 & 16.3 & 13.7 \\ 
%01 - 15 Oct &  -   &  -   & 12.0 &  -   &  -   & 11.9 &  -   &  -   & 11.5 \\ 
%%16 - 31 Oct & 0 & 0 & 0 & 0 & 0 & 0 & 0 & 0 & 0 \\ 
%\hline
%\end{tabular}
%\end{center}
%\end{table}

%\newpage
%\subsubsection{Ferrybox - Andr\'{e}}

%\newpage
%\subsection{Drifters and oil drift}
%We have observations of drift from release of surface drifters during two cruises in the Oslofjord, and from the oil spill during the grounding of the cargo ship Godafoss.

% % % % % % % % % % % % % % % % % % 
\newpage
\subsubsection{Surface drifters}
As part of the FjordOs project two cruises in the Oslofjord were performed. During these cruises a total number of 15 surface drifters were released (Figure \ref{fig:drifters_design}). Two were released during a cruise in September 2014, and the remaining 13 was released in a cruise in September 2015 (Figure \ref{fig:drifters_tracks}). The second cruise is documented in \cite{hjelm:etal:2016}, which also describes the drifters used in detail. The focus area of the drifter campaigns is the Breiangen area, and the area between Horten and Moss.

\begin{figure}[ht]
\centerline{
\includegraphics*[width=0.5\textwidth]{Figurer/Driftere_ombord}
\includegraphics*[width=0.5\textwidth]{Figurer/Driftere_vann}
}
\caption{\small
The "home-made" drifters used in this study. Left-hand panel depicts them on the deck of the R/V Trygve Braarud, while the right-hand panel displays them in the water after deployment.}
\label{fig:drifters_design}
\end{figure}

\begin{figure}[ht]
\centerline{
\includegraphics*[width=0.5\textwidth]{Figurer/drifters_sept2014}
\includegraphics*[width=0.5\textwidth]{Figurer/drifters_low_crop}
}
\caption{\small
Displayed are the drifter trajectories. Lef-hand panel shows the trajectories of the two drifters released in September 2014, while the right-hand panel displays the trajectories of the 13 drifters released in September 2015.}
\label{fig:drifters_tracks}
\end{figure}

% % % % % % % % % % % % % % % % % % 
\subsection{Godafoss oil spill}
\label{sect:godafoss_obs}
On Thursday the 17th of February 2011 at 19:52 local time, the containership Godafoss ran aground at the Kv{\ae}rnskj{\ae}rgrunnen rock in L{\o}peren, between the islands of Asmal{\o}y and Kirk{\o}y in Hvaler municipality in southeastern Norway (Figure \ref{fig:godafoss_oil}). One of the effects of this grounding was an acute release of oil from the ship, which drifted westward from the accident site. The oil slick has been observed from aeroplane by Kystverket, and the sites where stranded oil has been observed, are also registered (Figure \ref{fig:godafoss_oil}). This accident was on of the motivationg factors for the project FjordOs, and hence, although no simulations are performed for this particular event, we will, nevertheless simulate trajectories from this location using the simulations performed for the period April 2015 through December 2015 (Section \ref{sec:evalu}). 

At the time of the grounding there were clear skies and temperatures around -3$^o$C. Observations of wind from Str{\o}mtangen lighthouse (15 km away from the grounding site) indicate 6-7 m/s winds from the north-east.

\begin{figure}[ht]
\centerline{
\includegraphics*[width=\textwidth]{Figurer/Godafoss}
}
\caption{\small
Observed oil spill from the Godafoss accident. The red arrow on the right-hand side indicates the grounding position (Kv{\ae}rnskj{\ae}rgrunnen). The oil from the ship was released on the 17th of February 2011 at 19:52 local time. The grey areas are oil slicks observed from aircraft and the green areas indicate stranded oil. The text at the green areas in the figure also specify the name of the locations where oil was observed and when.}
\label{fig:godafoss_oil}
\end{figure}


\newpage
\clearpage
