\section{Observations}

The relevant observed data in the area of interest are scattered in both time and space (Fig.~\ref{fig:kart_obs}). Observations over longer time period includes three stations measuring sea level (Viker, Oscarsborg, and Oslo), one bottom-mounted doppler measuring currents in two depth levels (Slagentangen), and one device measuring sea temperature (Scanmar AS). More occasional observations includes CTD measurements, current measurements, ferrybox, some drifter experiments, and water temperature at a few selected beaches during the summer.  

\begin{figure}[htb]
\centerline{
\includegraphics*[trim=0cm 0.8cm 0cm 0cm,clip=true,width=0.8\textwidth]{Figurer/kart_obs}
}
\caption{\small
Positions for observations are marked.
}
\label{fig:kart_obs}
\end{figure}

\newpage

\subsection{Water level}
The Norwegian Mapping Authority has three permanent stations measuring sea level in the area of interest (Fig.~\ref{fig:kart_obs}). The station at Viker is placed close to the open boundary of the model area. The station at Oscarsborg is placed halfway in the Inner Oslofjord, and the station at Oslo is placed in the innermost part of the fjord.

\clearpage

\subsection{Currents}

\subsubsection{Currents in two cross sections}
In the period from Monday 15th to Thursday 18th of September 2014 altogether seven rigs with instruments was deployed at seven positions in Oslofjorden (Tab.~\ref{tab:Statnett} and Fig.~\ref{fig:kart_obs}). In the period from Monday 24th to Wednesday 16th of November the seven rigs with instruments were recovered. All but one rig had profiling current meters, while the last rig had TinyTag temperature loggers deployed at seven different depths between 20 and 120 m. At one station a single point current meter was deployed just below the profiling current meter. 

The rigs were deployed as a part of a project conducted by Statnett, NIVA, Akvaplan NIVA, and the University of Oslo. The research vessel F/F Trygve Braarud was used during deployment and recovery of the rigs. Details on the applied rigs and corresponding instruments can be found in \cite{staalstrom:2015}.

The strongest currents were found where the fjord is relatively narrow, in the Dr{\o}bak Sound at the stations Filtvedt (Km1) and Brenntangen (Kn2). The stronger current in the more narrow part of the fjord can be explained by a stronger tidal signal due to the fjord geometry. In both the two transects across the fjord, it was the shallowest stations that had the highest current velocities. This can be explained by the fact that the horizontal pressure gradient has a tendency to decrease with depth.

\begin{table}[ht] 
\caption{Target positions (WGS84) of the instrument rigs. Depths at the stations are from the Statnett terrain model.} 
\label{tab:Statnett} 
\centering 
\begin{tabular}{|llcccl|} 
\hline  
{\bf Station} & {\bf Name} & {\bf Latitude} & {\bf Longitude} & {\bf Depth} & {\bf Instruments} \\ 
&& {\bf [$^o$N]} & {\bf [$^o$E]} & {\bf [m]} & \\ \hline
Kp11.2 & Sm\aa skj\ae r & 59.350124 & 10.497661 & 20 & Aquadopp600 AQP1531 \\
(Ri1) &&&&& Transducer LRT2 \\ \hline
Kp5.7 & Laksetrappa & 59.343452 & 10.581023 & 75 & Aquadopp400 AQP4689 \\
(Rl1) &&&&& Transducer LRT3 \\
      &&&&& Aanderaa Seaguard \\  \hline
Kp2.6 & Botnegrunnen & 59.352375 & 10.626822 & 96 & Continental WAV6117 \\
(Rm1) &&&&& Transducer LRT4 \\ \hline
Kp0.7 & Evje & 59.363182 & 10.653576 & 64 & Aquadopp400 AQP2931 \\
(Rn1) &&&&& Transducer LRT5 \\ \hline
Kn2 & Brenntangen & 59.581803 & 10.646087 & 54 & Aquadopp400 AQP5608 \\
      &&&&& Transducer LRT6 \\ \hline
Km1 & Filtvedt & 59.582064 & 10.627372 & 153 & Continental CNL6037 \\
      & (current) &&&& Transducer 207-2 \\ \hline
Km2 & Filtvedt & 59.580778 & 10.626239 & 125 & 7 TinyTags UIO1-7 \\
      & (temperature) &&&& Transducer 203-2 \\
\hline
\end{tabular}
\end{table}

\clearpage

\subsubsection{Currents at Slagentangen}
Using a bottom-mounted Doppler, Exxonmobil has measured the currents in two depths since 1997. The device is placed 50-80 meters northwest of Turning Dolphin at the Slagen Refinery (Fig.~\ref{fig:kart_obs} and \ref{fig:Slagen-kart}). Note that Bliksekilen nature reserve is located west of Slagen Refinery, which is a shallow water area with rare flora and fauna. 

\begin{figure}[htb]
\centerline{
\includegraphics*[trim=0cm 0cm 1cm 0cm,clip=true,width=0.8\textwidth]{Figurer/Slagen_kart}}
\caption{\small
Map of Slagen Refinery. The red dot marks the position corresponding to the extracted simulated data. Souce: Norwegian Coastal Administration}
\label{fig:Slagen-kart}
\end{figure}

\clearpage
\subsection{Hydrography}

\subsubsection{CTD measurements}

In order to monitor the eutrophication state of the outer Oslofjord and the Drammensfjord, NIVA are responsible for measurements at selected positions. The measurements includes different types of data, including CTD, and is freely available through a web portal\footnote{http://www.aquamonitor.no/ytreoslofjord/}. Between April 2014 and December 2015 CTD measurements were performed in the model area at ten of the positions (Fig.~\ref{fig:kart_obs}). 

\begin{table}
\caption{The coordinates and number of measurements at the positions of the CTD measurements taken between April 2014 and December 2015 in the model area. Source: NIVA} 
\label{tab:CTD_pos} 
\centering 
\begin{tabular}{|clcccc@{}c|} 
\hline  
{\bf Tag} & {\bf Station} & {\bf Latitude} & {\bf Longitude} & \multicolumn{2}{c}{\bf Number of measurements} &\\ 
&& {\bf [$^o$N]} & {\bf [$^o$E]} & 2014 & 2015 &\\ \hline
D-2 & Inner Drammensfjord & 59.6280 & 10.4210 & 5 & 7 &\\ 
%16.06.2014 & 16.01.2015 \\
%&&&& 05.07.2014 & 05.02.2015 \\
%&&&& 18.08.2014 & 17.06.2015 \\
%&&&& 29.09.2014 & 06.07.2015 \\
%&&&& 15.11.2014 & 14.08.2015\\
%&&&& & 27.09.2015 \\
%&&&& & 12.11.2015 \\ \hline 
D-3 & Inner Drammensfjord & 59.7060 & 10.3140 & 5 & 6 &\\ \hline
%16.06.2014 & 16.01.2015 \\
%& Solumstrand &&& 05.07.2014 & 17.06.2015 \\
%&&&& 18.08.2014 & 06.07.2015 \\
%&&&& 29.09.2014 & 14.08.2015 \\ 
%&&&& 15.11.2014 & 27.09.2015 \\
%&&&& & 12.11.2015 \\
LA-1 & Larviksfjord & 59.0190 & 10.0520 & 5 & 7 &\\ 
%14.06.2014 & 17.01.2015 \\
%&&&& 03.07.2014 & 04.02.2015 \\
%&&&& 16.08.2014 & 16.06.2015 \\
%&&&& 24.09.2014 & 07.07.2015 \\
%&&&& 13.11.2014 & 13.08.2015 \\
%&&&& & 23.09.2015 \\
%&&&& & 10.11.2015\\ \hline
MO-2 & Kippenes & 59.4840 & 10.6780 & 5 & 7 &\\ \hline
%16.06.2014 & 15.01.2015 \\
%&&&& 04.07.2014 & 05.02.2015 \\
%&&&& 17.08.2014 & 16.06.2015 \\
%&&&& 28.09.2014 & 05.07.2015 \\
%&&&& 15.11.2014 & 14.08.2015 \\
%&&&& & 26.09.2015 \\
%&&&& & 12.11.2015 \\ \hline
OF-1 & Torbj{\o}rnskj{\ae}r & 59.0410 & 10.7540 & 5 & 7 &\\ 
%OF-2 & Missingene & 59.1870 & 10.6920 & 0 & 14 &\\ 
OF-5 & Breiangen & 59.4870 & 10.4580 & 5 & 7 &\\ \hline
S-9  & Haslau, Singlefjord & 59.1140 & 11.1620 & 7 & 10 &\\ 
SF-1 & Sandefjord & 59.0770 & 10.2460 & 5 & 7 &\\ \hline
T{\O}-1 & Vestfjord & 59.2030 & 10.3550 & 5 & 7 &\\ 
{\O}-1 & Leira. Vesterelva & 59.1370 & 10.8340 & 7 & 10 &\\ \hline
\end{tabular}
\end{table}




\subsubsection{Water temperature near {\AA}sg{\aa}rdstrand}

Hourly temperature measurements at one meter depth over the last 10 years have been measured by Scanmar AS located three kilometres south of {\AA}sg{\aa}rdstrand. The device has an accuracy of $\pm 0.15^o$C in the range from -5 to +30$^o$C. Note that 2014 had a warmer summer followed by a warm winter resulting in higher maximum and lower minimum than 2015 (Fig.~\ref{fig:Scan_temp}). 
% Bør kanskje skrive noe om fluktuasjonene her...

\begin{figure}[htb]
\centerline{
\includegraphics*[trim=2cm 0cm 2cm 0cm,clip=true,width=\textwidth]{Figurer/temp_Scanmar}}
\caption{\small
Observed temperature measured by Scanmar AS}
\label{fig:Scan_temp}
\end{figure}

%\begin{table}[ht] 
%\caption{Observed water temperature near \AA sg\aa rdstrand} 
%\label{tab:Scan_temp} 
%\centering 
%\begin{tabular}{|cccccc|} 
%\hline  
%{\bf Year} & {\bf Minimum} & {\bf 5 percentile} & {\bf Mean} & {\bf 95 percentile} & {\bf Maximum} \\
%\hline
%\small 2005 & 1.0 & 4.0 & 13.2 & 20.7 & 24.9 \\
%\small 2006 & -6.0 & -0.5 & 10.1 & 20.6 & 23.9 \\
%\small 2007 & -1.2 & 1.1 & 9.7 & 18.3 & 21.4 \\
%\small 2008 & 0.1 & 2.6 & 10.4 & 19.3 & 24.6 \\
%\small 2009 & -2.3 & -0.3 & 9.4 & 19.8 & 24.9 \\
%\small 2010 & -1.7 & -0.2 & 8.7 & 18.7 & 20.5 \\
%\small 2011 & -1.4 & -1.0 & 9.6 & 19.0 & 22.6 \\
%\small 2012 & -1.0 & 0.4 & 9.2 & 18.5 & 21.4 \\
%\small 2013 & -1.3 & -0.4 & 9.3 & 19.4 & 22.2 \\
%\small 2014 & -1.1 & 1.2 & 10.6 & 21.7 & 26.4 \\
%\small 2015 & 0.7 & 3.5 & 10.5 & 18.7 & 20.8 \\
%\hline
%\end{tabular}
%\end{table}

\newpage
\subsubsection{Water temperature in the Inner Oslofjord}

Temperature measurements at three beaches in the Inner Oslofjord (Fig.~\ref{fig:kart_strand}) are performed in a cooperation between Asker and B{\ae}rum kommune, and Finnerud Elektronikk. The digital thermometers (Maxim Integrated DS18B20) have an accuracy of $\pm 0.5^o$C and are placed 40 cm beneath the water surface in positions where the water depths are several meters. Temperatures are measured every three hours from 09:00 to 18:00 during the summer months. 

The trends of the observed temperatures at the three beaches are similar, but the temperature is generally lower at the southern beach, Sj{\o}strand, than at the northern beach, Stor{\o}yodden (Fig.~\ref{fig:temp_strand}). In 2014 there were two local maximums during July, and the maximum observed temperatures in 2014 were higher than in both 2013 and 2015. The temperature increases 1-3 degrees during the day and decreases during the night.

\begin{figure}[ht]
\centerline{
\begin{minipage}[l]{0.59\textwidth}
\includegraphics*[trim=0 0 0 1cm,clip=true,width=\textwidth]{Figurer/badestrand_kart}
\end{minipage}
\begin{minipage}[r]{0.4\textwidth}
\fbox{\includegraphics*[trim=1 0 0 3cm,clip=true,width=\textwidth]{Figurer/kart_Storoyodden.png}} \\
\fbox{\includegraphics*[trim=0 0 0 3cm,clip=true,width=\textwidth]{Figurer/kart_Hvalstrand.png}} \\
\fbox{\includegraphics*[trim=1 1cm 0 1.5cm,clip=true,width=\textwidth]{Figurer/kart_Sjostrand.png}} \\
\end{minipage}
}
\caption{\small
The positions at three beaches in the Inner Oslofjord where the temperature measurements are performed}
\label{fig:kart_strand}
\end{figure}

\begin{figure}[ht]
\centerline{
\includegraphics*[trim=2cm 0 2cm 0cm,clip=true,width=\textwidth]{Figurer/badetemp}
}
\caption{\small
Observed temperature at three beaches in the Inner Oslofjord}
\label{fig:temp_strand}
\end{figure}

%\begin{table}[ht]
%\caption{Mean observed temperatures at three beaches in the Inner Oslofjord. Only time periods with more than 8 days of observations during the given time period are included.}
%\label{tab:temp_strand}
%\begin{center}
%\begin{tabular}{|l|ccc|ccc|ccc|} \hline
%     & \multicolumn{3}{c|}{Stor\o yodden} & \multicolumn{3}{c|}{Hvalstrand} & \multicolumn{3}{c|}{Sj\o strand} \\ \hline
%Time period & 2013 & 2014 & 2015 & 2013 & 2014 & 2015 & 2013 & 2014 & 2015 \\ \hline
%%01 - 15 May & 0 & 0 & 0 & 0 & 0 & 0 & 0 & 0 & 0 \\ 
%16 - 31 May & 13.7 &  -   & 11.6 &  -   &  -   & 12.2 &  -   &  -   & 12.8 \\ 
%01 - 15 Jun & 15.0 & 19.0 & 13.3 & 15.9 & 19.4 & 13.7 &  -   & 19.9 & 14.6 \\ 
%16 - 30 Jun & 16.2 & 17.3 & 17.1 & 16.7 & 17.7 & 17.5 &  -   & 17.9 & 18.0 \\ 
%01 - 15 Jul & 17.8 & 17.6 & 18.2 & 18.3 & 18.3 & 19.8 & 19.4 & 19.3 &  -   \\ 
%16 - 31 Jul & 20.1 & 22.2 & 18.1 & 19.7 &  -   & 18.9 & 21.2 & 23.9 & 18.4 \\ 
%01 - 15 Aug & 19.5 & 21.3 & 18.0 & 19.8 & 21.3 &  -   & 20.6 & 21.7 & 18.8 \\ 
%16 - 31 Aug & 18.9 & 19.6 & 19.1 & 18.9 & 19.6 & 19.3 & 19.5 & 19.4 & 19.0 \\ 
%01 - 15 Sep & 17.9 & 18.3 & 16.1 & 17.7 & 18.6 & 16.5 & 18.1 & 18.9 & 16.2 \\ 
%16 - 30 Sep &  -   & 15.9 & 14.1 &  -   & 16.2 & 14.1 & 15.1 & 16.3 & 13.7 \\ 
%01 - 15 Oct &  -   &  -   & 12.0 &  -   &  -   & 11.9 &  -   &  -   & 11.5 \\ 
%%16 - 31 Oct & 0 & 0 & 0 & 0 & 0 & 0 & 0 & 0 & 0 \\ 
%\hline
%\end{tabular}
%\end{center}
%\end{table}

\newpage
\subsubsection{Ferrybox - Andr\'{e}}

\newpage
\subsection{Drifters and oil drift}
We have observations of drift from release of surface drifters during two cruises in the Oslofjord, and from the oil spill during the grounding of the cargo ship Godafoss.

\subsubsection{Godafoss oil spill}
\label{sect:godafoss_obs}
The containership Godafoss ran aground at the Kv{\ae}rnskj{\ae}rgrunnen rock in L{\o}peren, between the islands of Asmal{\o}y and Kirk{\o}y in Hvaler municipality in southeastern Norway, on Thursday the 17th of February 2011 at 19:52 local time. One of the effects of this grounding was an acute release of oil from the ship, which drifted westward from the accident site. The oil slick has been observed from aeroplane by Kystverket, and the sites where stranded oil has been observed, are also registered (see Figure \ref{fig:godafoss_oil}).

At the time of the grounding, the weather in the area was nice, clear skies and temperatures around -3C. Observations of wind from Str{\o}mtangen lighthouse (15 km away from the grounding site) indicate 6-7 m/s winds from the north-east.

\begin{figure}[ht]
\centerline{
\includegraphics*[width=\textwidth]{Figurer/Godafoss}
}
\caption{\small
Observed oil spill from the Godafoss accident. The red arrow on the right hand side of the figure indicate the position of Kv{\ae}rnskj{\ae}rgrunnen, where the grounding happened, and from where the oil was released on the 17th of February 2011 at 19:52 LT. The grey areas are oil slicks observed from aircraft and the green areas indicate stranded oil. The text at the green areas in the figure also specify the name of the locations where oil has been observed, and when it was observed in those locations.}
\label{fig:godafoss_oil}
\end{figure}

\subsubsection{Surface drifters}
During a two cruises in the Oslofjord, we released a total number of 15 surface drifters (see Figure \ref{fig:drifters_design}). Two was released during a cruise in September 2014, and the remaining 13 was released in a cruise in September 2015 (Figure \ref{fig:drifters_tracks}). The second cruise, and the drifter type, is documented in \cite{hjelm:etal:2016}. The focus area of the drifter campaigns is the Breiangen area, and the area between Horten and Moss.

\begin{figure}[ht]
\centerline{
\includegraphics*[width=0.5\textwidth]{Figurer/Driftere_ombord}
\includegraphics*[width=0.5\textwidth]{Figurer/Driftere_vann}
}
\caption{\small
The "home-made" drifters used in this study. Depicted in the panel to the left when they are on board the research vessel, and in the right hand panel after they have been deployed.}
\label{fig:drifters_design}
\end{figure}

\begin{figure}[ht]
\centerline{
\includegraphics*[width=0.5\textwidth]{Figurer/drifters_sept2014}
\includegraphics*[width=0.5\textwidth]{Figurer/drifters_low_crop}
}
\caption{\small
The tracks of the drifters. In the left panel is the two drifters released in September 2014, and in the right panel the drifters released during the cruise in September 2015.}
\label{fig:drifters_tracks}
\end{figure}

\newpage
\clearpage
