\section{Area of interest}

The area of interest is the Oslofjord including the Drammensfjord and the Inner Oslofjord (Fig.~\ref{fig:kart}). The Oslofjord is located in Southern Norway with the capital of Norway, Oslo, located in the innermost part of the fjord. 
The fjord is about 100 km long, and at 59.56$^o$N the fjord splits into two branches; the Inner Oslofjord (eastern branch) and the Drammensfjord (western branch). The width of the fjord varies from around 25 km at the entrance to about 1-2 km at the Dr{\o}bak Sound in the Inner Oslofjord and 180 meters at Svelvik in the Drammensfjord.

Numerous small and large islands combined with deep and shallow basins (Fig.~\ref{fig:kart}) contributes to a complex circulation pattern. In addition several river discharge fresh water into the fjord, including two of Norway's largest rivers, namely Glomma (near Fredrikstad) and Drammenselva (near Drammen).

\begin{figure}[htb]
\centerline{
\includegraphics*[trim=0cm 0.8cm 0cm 0cm,clip=true,width=0.7\textwidth]{Figurer/kart}
}
\caption{\small
Area of interest; The Oslofjord. The red dots show the location of some major and minor cities and villages along the coast. The blue diamnond indicated the position of the F{\ae}rder Lighthouse.
}
\label{fig:kart}
\end{figure}

\newpage
