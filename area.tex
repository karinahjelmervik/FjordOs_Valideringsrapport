\section{Area of interest}

The area of interest is the Oslofjord including the Drammensfjord and the Inner Oslofjord (Fig.~\ref{fig:kart}). The Oslofjord is located in Southern Norway with the capital of Norway, Oslo, located in the innermost part of the fjord. 
The fjord is about 100 km long, and at 59.56$^o$N the fjord splits into two branches; the Inner Oslofjord (eastern branch) and the Drammensfjord (western branch). The width of the fjord varies from around 50 km at the entrance to about 1-2 km at the Dr{\o}bak Sound in the Inner Oslofjord and 180 meters at Svelvik in the Drammensfjord.

Both branches have a sill. The sill in the eastern branch, the Dr{\o}bak Sill, is located close to the island Oscarsborg. The Dr{\o}ak Sill consists partly of a man made sill, an underwater jetty only 1-2 meters deep extending halfway across the fjord from the western side. Towards the eastern side there is a natural sill of about 20 meters depth. Due its narrowness and shallowness the Dr{\o}bak Sill area is famous for its strong tidal currents, which easily exceeds 1 m/s even though the mean total tidal amplitude is less than 20 cm. 
The western branch is almost cut in two by a narrow, shallow, and long sill which is only 11 meters deep, 180 meters wide, and more than 1 km long. This sill causes a relatively strong tidal current, called the Svelvikstraum.
North of these sills the maximum depth is more than 120 meters in both branches. 
The location of these sills, about two thirds into the fjords, makes the Oslofjord peculiar among Norwegian fjords in that most of them have their sill at the entrance.

Numerous smaller and larger islands combined with deeper and more shallow basins (Fig.~\ref{fig:kart}) contributes to a complex circulation pattern. In addition several river discharge fresh water into the fjord, including two of Norway's largest rivers, namely Glomma (near Fredrikstad) and Drammenselva (near Drammen).

The water level and motion in the Oslofjord is also impacted by events in the Skagerrak, which lies in the north-eastern part of the North Sea outside of the fjord's southern boundary. Storm surge events with amplitudes of one meter and higher are observed in the fjord and are mostly associated with wind and pressure events in the Skagerrak/North Sea area.

\begin{figure}[htb]
\centerline{
\includegraphics*[trim=0cm 0.9cm 0cm 0cm,clip=true,width=0.7\textwidth]{Figurer/kart}
}
\caption{\small
Area of interest; The Oslofjord. The red dots show the location of some major and minor cities and villages along the coast. The blue diamnond indicated the position of the F{\ae}rder Lighthouse.
}
\label{fig:kart}
\end{figure}

\clearpage
