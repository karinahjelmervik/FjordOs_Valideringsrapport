%\newpage
\section{Evaluation}
\label{sec:evalu}
% % % % % % % % % % % % %
\subsection{Water level and tide}
\label{subsec:wlevele}
Time series of water level from the three tide gauge stations mentioned above (Section \ref{subsec:wlevelo}) were extracted for the period April 2014 through December 2015. We also extracted time series for the same period from locations near the three stations from the simulation. To compare the two time series we first analysed them using t\_tide \citep{pavlo:etal:2002} to extract the tidal components. We then subtracted the water level due to these eleven tidal constituents to derive time series of the residual water level.   

%Eleven tidal components are included in the model at the southern open boundary using their corresponding amplitudes and phases for both depth integrated currents and water level (Røed et al., 2016). The time series of tidal components included in the tidal forcing are in fairly good agreement (Fig.~\ref{fig:Waterlevel_tide}, upper, and Tab.\-\ref{tab:Tide}). In addition to the tidal components included in the tidal forcing, more tidal components are present in the time series (Fig. 11). Tidal components with periods of approximately one year (SA) and half a year (SSA) respectively are present in both the observations and the simulations (Tab.3). In addition, the observations have more components with shorter periods which are not included in the tidal forcing and thereby not present in the simulations. This is consistent with the frequency series of the Fourier transformed water level (Fig. 12).

Recall that the tidal forcing at the southern open boundary of the model includes only eleven tidal constitutents, and that we have adjusted the tidal forcing by use of the observed tides at Viken for these eleven constitutents \citep{roed:etal:2016}. Furthermore we used corresponding amplitudes and phases for the depth integrated currents and water level. As depicted by Fig.~\ref{fig:Waterlevel_jan15}, showing the tidal elevation for January 2015 due to these eleven components at Oscarsborg, model and observations are in fairly good agreement. We also note that the residuals, as shown by the lower panel of Fig.~\ref{fig:Waterlevel_jan15}, compare favourably. 

\begin{figure}[hb] 
	\centerline{ \includegraphics*[trim=3cm 0cm 2.5cm 0cm,clip=true,width=\textwidth]{Figurer/Oscarsborg_Tide_selected_jan15} } 
	\centerline{ \includegraphics*[trim=3cm 0cm 2.5cm 0cm,clip=true,width=\textwidth]{Figurer/Oscarsborg_WL_rest_jan15} } 
 	\caption{\small Simulated (black) and observed (red) time series of tides (upper panel) and residual (lower panel) at Oscarsborg (\ref{fig:kart_obs}). The tidal elevation is composed of the eleven components included in the tidal forcing only. The residual is the total water level from which is subtracted the tidal elevation due to these eleven constituents.} 
	\label{fig:Waterlevel_jan15} 
\end{figure} 

This is further corroborated by a comparison of the simulated and observed tidal amplitudes and phases for thirteen individual tidal constitutents at all the three tidal gauge stations (Tab.~\ref{tab:Tide}) of which three of them are not included in the tidal forcing. For instance Tab.~\ref{tab:Tide} include tidal components with periods of approximately one year (SA) and half a year (SSA). Surprisingly they also show up in the model simulations despite not being included in the tidal forcing. This may be explained by the fact that we in addition to the tides as forcing on the southern boundary of the FjordOs model also use daily mean water level, hydrography and currents from the NorKyst800 model when nesting the FjordOs model to the former \citep{roed:etal:2016}. 

\begin{table}[t] 
%\vspace{-1.5cm} 
	\caption{\small Simulated and observed tidal amplitude and phase for selected tidal components.} 
	\label{tab:Tide} 
	\centering 
	\begin{tabular}{|c|c|l|cc|cc|cc|c|} 
\hline  
	  &	&	& \multicolumn{2}{|c|}{\small \bf Viker} & \multicolumn{2}{|c|}{\small \bf Oscarsborg} & \multicolumn{2}{|c|}{\small \bf Oslo} & {\small \bf Included} \\  
{\small \bf Comp.} & {\small \bf Period} & {\small \bf sim/} & {\small \bf amp.} & {\small \bf phase.} & {\small \bf amp.} & {\small \bf phase.} & {\small \bf amp.} & {\small \bf phase.} & {\small \bf in tidal} \\ 
	    & {\small \bf [h]} & {\small \bf obs} & {\small \bf [cm]} & {\small \bf [deg]} & {\small \bf [cm]} & {\small \bf [deg]} & {\small \bf [cm]}   & {\small \bf [deg]} & {\small \bf forcing} \\ \hline 
\small SA   & \small 8764	 	& sim & 15.5 & 284 & 15.6 & 286 & 15.4 & 286 & no   \\
\small      &        	& obs & 10.0 & 319 & 11 & 322 & 11.4 & 324 &    \\
\small SSA  & \small 4382 		& sim & 8.8 & 197 & 9.2 & 200 & 9.4 & 200 & no   \\
\small      &        	& obs & 7.5 & 188 & 8.0 & 189 & 8.2 & 190 &    \\
\small K2   & \small 11.9672 	& sim & 1.6 & 10 & 2.0 & 13 & 2.1 & 15 & yes  \\
\small      &        	& obs & 0.7 & 45 & 0.8 & 66 & 0.9 & 66 &    \\
\small S2   & \small 12.0000 	& sim & 3.3 & 64 & 3.9 & 69 & 4.2 & 70 & yes  \\
\small      &        	& obs & 2.9 & 46 & 3.3 & 65 & 3.5 & 69 &    \\
\small M2   & \small 12.4206 	& sim & 11.5 & 105 & 13.2 & 112 & 13.9 & 114 & yes  \\
\small      &        	& obs & 11.9 & 105 & 13.8 & 121 & 14.4 & 125 &    \\
\small N2   & \small 12.6584 	& sim & 3.0 & 69 & 3.5 & 75 & 3.7 & 76 & yes  \\
\small      &        	& obs & 3.0 & 60 & 3.4 & 76 & 3.6 & 80 &    \\
\small K1   & \small 23.9345 	& sim & 0.2 & 187 & 0.1 & 175 & 0.2 & 157 & yes  \\
\small      &        	& obs & 0.4 & 127 & 0.7 & 130 & 0.8 & 130 &    \\
\small P1   & \small 24.0659 	& sim & 0.6 & 322 & 0.6 & 334 & 0.7 & 342 & yes  \\
\small      &        	& obs & 0.2 & 129 & 0.3 & 102 & 0.4 & 97 &    \\
\small O1   & \small 25.8193 	& sim & 3.5 & 337 & 3.8 & 339 & 3.8 & 339 & yes  \\
\small      &        	& obs & 2.2 & 277 & 2.3 & 281 & 2.4 & 282 &    \\
\small Q1   & \small 26.8684 	& sim & 0.0 & 231 & 0.0 & 216 & 0.1 & 215 & no   \\
\small      &        	& obs & 1.1 & 190 & 1.2 & 198 & 1.3 & 200 &    \\
\small MN4  & \small 6.2692 	& sim & 0.2 & 5 & 0.5 & 32 & 0.6 & 35 & yes  \\
\small      &        	& obs & 0.4 & 249 & 0.6 & 289 & 0.7 & 297 &    \\
\small M4   & \small 6.2103 	& sim & 1.0 & 355 & 1.9 & 18 & 2.5 & 23 & yes  \\
\small      &        	& obs & 1.2 & 281 & 1.8 & 324 & 2.3 & 332 &    \\
\small MS4  & \small 6.1033 	& sim & 0.6 & 80 & 1.2 & 107 & 1.6 & 111 & yes  \\
\small      &        	& obs & 0.3 & 360 & 0.5 & 44 & 0.7 & 56 &    \\
\hline
\end{tabular}
\end{table}

In addition, and as shown by Fig.~\ref{fig:Waterlevel_tide}, the observations have more tidal components with shorter periods which are not included in the tidal forcing, and thereby not present in the simulations. %This is consistent with the frequency series of the Fourier transformed water level (Fig.~\ref{fig:Waterlevel_tide}).

\begin{figure}[tbh] 
	\centerline{ \includegraphics*[trim=3cm 0cm 2.5cm 0cm,clip=true,width=\textwidth]{Figurer/Oscarsborg_Tide_not_included} } 
	\caption{\small Time series at Oscarsborg of the tidal components not included in the tidal forcing.} 
	\label{fig:Waterlevel_tide} 
\end{figure} 

\begin{comment}

\begin{figure}[tbh] 
  \centerline{ \includegraphics*[trim=3cm 0cm 2.5cm 0cm,clip=true,width=\textwidth]{Figurer/Oscarsborg_Tide_Frequency_obs.png} } 
  \centerline{ \includegraphics*[trim=3cm 0cm 2.5cm 0cm,clip=true,width=\textwidth]{Figurer/Oscarsborg_Tide_Frequency_sim.png} } 
  \caption{\small Frequency series of Fourier transformed observed (upperpanel) and simulated (lower panel) water levels at Oscarsborg} 
  \label{fig:Waterlevel_freq} 
\end{figure} 

\end{comment}
%\newpage 
As revealed by Tab.~\ref{tab:Tide} the observed M$_2$ amplitude increases from south to north. This is reflected in the simulations as well as displayed by Fig. Fig.~\ref{fig:M2field}. It is interesting to note that the lowest M$_2$ amplitude is found in the Drammensfjord north of the threshold in Svelvik. In fact the M$_2$ phase has a sudden increase at the thresholds of Svelvik and Dr{\o}bak (Fig.~\ref{fig:M2field}). The same is true for the majority of the other relevant tidal components.

%The amplitude of M$_2$ increases from south to north in the Inner Oslofjord both in the simulations and the observations (Fig.~\ref{fig:M2field} and Tab.~\ref{tab:Tide}). The lowest M$_2$ amplitude in the area of interest, is in the Drammensfjord north of the threshold in Svelvik. The M$_2$ phase has a sudden increase at the thresholds of Svelvik and Dr{\o}bak (Fig.~\ref{fig:M2field}). The same yields for the majority of the other relevant tidal components.

\begin{figure}[hb] 
\centerline{ 
\includegraphics*[trim=1cm 0cm 0cm 0cm,clip=true,width=0.49\textwidth]{Figurer/M2amp_felt}  
\includegraphics*[trim=0.8cm 0cm 0cm 0cm,clip=true,width=0.49\textwidth]{Figurer/M2fase_felt} 
} 
\caption{\small 
Simulated fields of M$_2$ amplitude (left-hand panel) and phase (right-hand panel). Corresponding observed values for M$_2$ amplitude and phase are marked with circles at Viker, Oscarsborg, and Oslo.} 
\label{fig:M2field} 
\end{figure} 



% % % % % % % % % % % %
\clearpage 
\subsection{Currents}
\label{subsec:curree}

%\subsubsection{Currents in two cross sections}

%Current measurements were performed in two cross sections. Here the model results are evaluated using the measurements at the northern cross section, Brenntangen-Filtvedt. The results are similar at the southern cross section.

By analyzing the observed and simulated currents we notice that the strongest currents are found at the two northernmost moorings Filtvedt and Brenntangen at the entrance to the Dr{\o}bak Sound (Fig.~\ref{fig:kart_obs}). Here the fjord is relatively narrow, and hence the somewhat stronger currents may be explained by a stronger tidal signal due to the tapering of the width of the fjord at the entrance to the Dr{\o}bak Sound. We also note that the stations with the shallowest depth exhibits the highest current velocities. This may be explained by the fact that the horizontal pressure gradient has a tendency to decrease with depth.

In the remaining part we restrict the comparison regarding currents to three of the seven rigs equipped with an ADCP. These are the two northern moorings at Filtvedt and Brenntangen at the entrance to the Dr{\o}bak Sound, and the mooring Slagen near {\AA}sg\aa{}rdstrand. The rationale is that the remaining four moorings did not add any significant new insight other than corroborating the evaluation results gathered from comparing the simulated and observed currents at the former three stations.

%The observed and simulated currents at Filtvedt and Brenntangen are of the same magnitude in strength, but the flow pattern differs. Figure~\ref{fig:Filtvedt-cur} shows the observed and simulated currents at Filtvedt as an example. The observed currents at Brenntagen is similar in strength, but the peaks in current strength do not occur at the same time. The observations are dominated by noise in the upper 35-40 meters. Only depths larger than 40 meters are therefore included in the comparison.

Since the observed and simulated currents at Filtvedt and Brenntangen are of the same magnitude in strength, we only show the observed and simulated currents at Filtvedt (Fig.~\ref{fig:Filtvedt-cur}). We emphasize though that the flow pattern at the two moorings differs. Although the observed currents at Brenntagen is similar in strength, the peaks in current speed do not occur at the same time. Moreover, since the observations are dominated by noise in the upper 35-40 meters, we only include observed currents for depths larger than 40 meters in the comparison.

As revealed by Figure \ref{fig:Filtvedt-cur} the 3D current field is complex, and varies horizontally as well as vertically and in time. First we note that the simulated currents are undeniably smoother than the observed currents. This is to be expected since tidal constituents of higher frequencies are present in the observed currents (Section \ref{subsec:wlevele}). Given this we nevertheless observe that there is a lot of similarities between the two. For one both have a three layered structure with an intermediate layer in which the flow is southward (blue, green and black colors), a deep layer dominated by lower frequency tides, and an upper layer with a more variable structure, but mostly northward flows in the modelled currents. Secondly, we note that the depth of the intermediate layer is not constant in time but varies similarly in the observations and the simulations.  

According to the simulations (Fig.~\ref{fig:Filtvedt-simcur}) this three layer structure is not singular for Filtvedt, but is rather the rule in the area at the entrance to the Dr{\o}bak Sound. The figure also reveals that generally the currents in the upper layer are stronger than further down. Furthermore the figure disclose the horizontal variability in the flow pattern telling us that the currents at Filtvedt cannot be taken at representative for the whole area. For instance it should be noted that the tidal currents do not reverse at the same time across the fjord, which explains why the currents at Filtvedt is still southward while in the deeper areas the tides are already toward north at this particular instant. At 100 meters depth the currents at Filtvedt are weak and towards south even though the currents at 100 meters depth is generally stronger and towards north. At 40 meters depth the currents at Brenntangen is weaker than in the rest of the cross section. Note that the depth at Brenntangen is only 58 meters in the observations while in the depth is 46 meters at the corresponding point in the simulations. 

\begin{figure}[ht]
\centerline{
\includegraphics*[trim=0 0 0 0,clip=true,width=\textwidth]{Figurer/Filtvedt_obs_cur}}
\centerline{
\includegraphics*[trim=0 0 0 0,clip=true,width=\textwidth]{Figurer/Filtvedt_sim_cur}}
\caption{\small
Observed (upper panel) and simulated (lower panel) currents at Filtvedt. Since the observations near the surface are dominated by noise, only depths larger than 40 meters are included in the observations. In the lower panel $z$ = 40 meters is marked with a black line. The model depth at this locatios is 155 meters, which is close to the measured depth of 153 meters (Table~\ref{tab:Statnett})).}
\label{fig:Filtvedt-cur}
\end{figure}

\begin{figure}[ht]
\centerline{
\includegraphics*[trim=2cm 3cm 1cm 3.3cm,clip=true,height=5cm]{Figurer/Filtvedt_t4611_z2_current}
\includegraphics*[trim=3.8cm 3cm 6cm 3.3cm,clip=true,height=5cm]{Figurer/Filtvedt_t4611_z40_current}
\includegraphics*[trim=3.8cm 3cm 1cm 3.3cm,clip=true,height=5cm]{Figurer/Filtvedt_t4611_z100_current}}
\caption{\small Snapshots of the simulated currents at three different depths valid 10th of October 2014 at 12:00 UTC. The left-hand panel is at 2 m depth, the middle panel at 40 m depth, while the right-hand panel is at 100 m depth. Note that the range shown by the colorbar in the left-hand panel is 0-60 cm/s, while the colorbar  regarding the middle and right-hand panels has a range 0-20 cm/s.}
\label{fig:Filtvedt-simcur}
\end{figure}

The tides are evident in the observed and simulated currents at all observation points. Both simulated and observed time series of the currents at the seven observation points are analyzed using t\_tide \cite{pavlo:etal:2002} in order to extract the tidal components for each depth. The same period in time is applied for both simulations and observations (mid-September to the end of November 2014). Figure~\ref{fig:Filtvedt-tide} shows the tidal currents at Filtvedt. At Filtvedt the tidal impact in the observations occurs earlier at larger depths, but in the simulations the tidal impact occurs earlier at more shallow depths. This might be due to the flow pattern at the different depths and the fact that the flow pattern depends strongly on the bottom topography which is smoothed in the simulations. Note also that the calculations of the tides are based on only six weeks of data.


\begin{figure}[ht]
\centerline{
\includegraphics*[trim=0 0 0 0,clip=true,width=\textwidth]{Figurer/Filtvedt_obs_tide}}
\centerline{
\includegraphics*[trim=0 0 0 0,clip=true,width=\textwidth]{Figurer/Filtvedt_sim_tide}}
\caption{\small
Observed (upper) and simulated (lower) tidal currents at Filtvedt.}
\label{fig:Filtvedt-tide}
\end{figure}


\clearpage 
\subsubsection{Current at Slagentangen}
The observed currents at Slagentangen are compared with simulated data from 1st October 2014 until 30th November 2015 at approximately the same location and depth (Fig.~\ref{fig:Slagen-kart}).

%\begin{figure}[ht]
%\centerline{
%\includegraphics*[trim=1cm 0cm 1cm 0cm,clip=true,width=0.5\textwidth]{Figurer/Slagen_kart}}
%\caption{\small
%Map retrived from the Norwegian Coastal Administration. The red dot marks the position corresponding to the extracted simulated data.}
%\label{fig:Slagen-kart}
%\end{figure}

\begin{figure}[ht]
\centerline{
\includegraphics*[trim=3cm 0cm 3cm 0cm,clip=true,width=\textwidth]{Figurer/Slagen_tid}}
\caption{\small
Timeseries of observed and simulated velocity magnitudes at Slagen.}
\label{fig:Slagen-tid}
\end{figure}

\begin{figure}[ht]
\centerline{
\includegraphics*[trim=2cm 1cm 1cm 0cm,clip=true,height=4cm]{Figurer/Slagen_Rose_obs} 
\includegraphics*[trim=2cm 1cm 3cm 0cm,clip=true,height=4cm]{Figurer/Slagen_Rose_sim}
}
\caption{\small
Current roses for observed (left) and simulated (right) velocity magnitude at the two depths from 1st of October 2014 to 1st of October 2015.}
\label{fig:Slagen-rose}
\end{figure}

\begin{figure}[t]
\centerline{
\includegraphics*[trim=2cm 0cm 2cm 0cm,clip=true,width=\textwidth]{Figurer/Slagen_pdf} 
}
\caption{\small
Probability density functions of velocities and directions at Slagen for 1st of October 2014 to 1st of October 2015. The bin width is 0.01 knots for velocity and 3 degrees for direction.}
\label{fig:Slagen-pdf}
\end{figure}

\begin{table}[ht]
%\vspace{-1.5cm}
\caption{Yearly maximum observed velocity at Slagen.}
\label{tab:Slagen_max}
\centering
\begin{tabular}{|l|lll|lll|}
\hline 
& \multicolumn{3}{|l|}{\bf Max. velocity at 10m depth} & \multicolumn{3}{|l|}{\bf Max. velocity at 2.5m depth} \\
{\bf Year} & {\bf Date} & {\bf [m/s} & {\bf [deg]} & {\bf Date} & {\bf [m/s]} & {\bf [deg]} \\ \hline 
\small 2006 & 21 Jan 2006 & 0.42 & 139 & 31 Oct 2006 & 0.57 & 140 \\
\small 2007 & 14 Jan 2007 & 0.42 & 172 & 21 Aug 2007 & 1.03 & 359 \\
\small 2008 & 22 Mar 2008 & 0.36 & 149 & 19 Dec 2008 & 0.57 & 160 \\
\small 2009 & 17 Dec 2009 & 0.45 & 142 & 24 Mar 2009 & 0.56 & 139 \\
\small 2010 & 09 Nov 2010 & 0.41 & 138 & 09 Nov 2010 & 0.54 & 138 \\
\small 2011 & 01 Jan 2011 & 0.39 & 146 & 30 Mar 2011 & 0.62 & 185 \\
\small 2012 & 05 Dec 2012 & 0.39 & 138 & 29 May 2012 & 0.57 & 140 \\
\small 2013 & 10 Oct 2013 & 0.42 & 143 & 10 Oct 2013 & 0.49 & 144 \\
\small 2014 & 18 Apr 2014 & 0.44 & 147 & 26 Mar 2014 & 0.55 & 143 \\
\small 2015 & 24 Jan 2015 & 0.33 & 128 & 21 Mar 2015 & 0.55 & 141 \\
\hline
\end{tabular}
\end{table}

\begin{figure}[ht]
\centerline{
\includegraphics*[trim=0cm 0cm 0cm 0cm,clip=true,width=0.8\textwidth]{Figurer/Slagen_QQ}}
\caption{\small
Combined QQ- and scatter plot of observed and simulated current at Slagen from 1st of October 2014 to 1st of October 2015.}
\label{fig:Slagen_QQ}
\end{figure}

Time series reveal that the observed velocities varies and follows no striking pattern (Fig.~\ref{fig:Slagen-tid}).
Current roses show that both the observed and the simulated velocities are stronger in the upper layer (Fig.~\ref{fig:Slagen-rose}). The simulated velocities are stronger than the observed velocities. This is in accordance with the probability density functions (Fig.~\ref{fig:Slagen-pdf}). The yearly maximum observed velocities are approximately 0.4 and 0.6 m/s at 10 and 2.5 meters depth respectively (Tab.~\ref{tab:Slagen_max}). During 2014 and 2015 maximum observed velocity at 2.5 meters depth was 0.55 m/s in southeast direction (143$^o$N) the 26th of March 2014. The velocity at 10 meters depth was 0.08 m/s (153$^o$N) at the time of maximum velocity at 2.5 meters depth indicating that the velocities are different in the two layers.

The mean directions are to the south east. At approximately 2.5 meters depth the mean directions are 146$^o$N and 139$^o$N for observed and simulated directions respectively which is in fairly good agreement. At approximately 10 meters depth the observed mean direction shifts to 170$^o$N while the simulated mean direction is 148$^o$N. 
Testing with popcorn indicate that the preferred direction of the surface currents are towards Bliksekilen located west of the Slagen Refinery. This is not the case neither in the observations nor the simulations.
The probability density functions reveals that the model captures the distribution of directions in the upper layer, but does not capture the change in direction between the two depths (Fig.~\ref{fig:Slagen-pdf}). The standard deviations at 2.5 and 10 meters are 55 and 66 degrees respectively for the observed directions, and 56 and 61 for the simulated directions.

The time series scatter plots reveal that the correlation in time is not satisfying (Fig.~\ref{fig:Slagen-tid} and \ref{fig:Slagen_QQ}). The model seem to have difficulties with capturing the right phenomena influencing the currents to the right time. This is a well known problem when it comes to forecasting currents. The QQ-plots also confirms that the simulated currents are stronger than the observed currents. 


%\clearpage
% % % % % % % % % % % % % % %
\subsection{CTD-measurements}
\label{subsec:CTDe}
For comparison we extracted temperature and salinity profiles from the model simulation at some of the CTD stations listed by Table ~\ref{tab:CTD_pos}, namely OF-1 in the outer part of the fjord, LA-1 in Larvikfjord and D-2 inside the Svelvik Sill in the inner Drammensfjord. Comparison between simulated and observed profiles are shown in Figs.~\ref{fig:CTD_OF-1}-~\ref{fig:CTD_D-3}. 

During the summer, the water in the upper layers are heated. 
The maximum surface temperature is observed towards the end of the summer. 
The profiles in the outer parts of the fjord indicate that the upper layer is too thin in the simulated data (Fig.~\ref{fig:CTD_OF-1}). 
At larger depths, the water is too cold and the salinity is too high. 
This might indicate that the open boundary input and the representation of vertical mixing should be modified. 

Some of the observations are taken from smaller fjord branches, such as in the Larviksfjord close to the open boundary in the outer part of the Oslofjord. 
In such shallow waters, observations reveal that the whole water column is heated during summer and cooled during winter, but the simulated temperature varies only in the upper 20 meters (Fig.~\ref{fig:CTD_OF-1}). 

The water masses below sill depth in the Drammensfjord are known to have very low vertical diffusivity leading to low oxygen conditions in the depth.
Fig.~\ref{fig:CTD_D-3} reveals that the vertical diffusivity in the same basin in the model is two high.
In the model water with low salinity is mixed down all the way to the bottom. 


\begin{figure}[tbh]
\centerline{
\includegraphics*[trim=0cm 0cm 0cm 0cm,clip=true,width=\textwidth]{Figurer/CTD_OF-1}}
\caption{\small
Observed (solid) and simulated (dashed) salinity and temperature profiles at station OF-1 Torbj{\o}rnskj{\ae}r in the outer part of the Oslofjord. All profiles are from 2015.}
\label{fig:CTD_OF-1}
\end{figure}

\begin{figure}[tbh]
\centerline{
\includegraphics*[trim=0cm 0cm 0cm 0cm,clip=true,width=\textwidth]{Figurer/CTD_LA-1}}
\caption{\small
Observed (solid) and simulated (dashed) salinity and temperature profiles at station LA-1 Larviksfjord in a fjord branch in the outer part of the Oslofjord. All profiles are from 2015.}
\label{fig:CTD_LA-1}
\end{figure}

\begin{figure}[tbh]
\centerline{
\includegraphics*[trim=0cm 0cm 0cm 0cm,clip=true,width=\textwidth]{Figurer/CTD_D-3}}
\caption{\small
Observed (solid) and simulated (dashed) salinity and temperature profiles at station D-3 Solumstrand. All profiles are from 2015.}
\label{fig:CTD_D-3}
\end{figure}

To get a better idea of how the properties of the water masses varies with time, contour plots of salinity and temperature are made as a function of depth and time.
Three of the stations in the monitoring program are chosen to describe how the salinity and temperature in different parts of the fjord system evolves through two seasons. Station D-3 outside Solumstrand in the inner Drammensfjord, station OF-5 in Breiangen and station OF-1 near Torbj{\o}rnskj{\ae}r close to the open boundary of the model domain, are chosen.

Observed salinity and temperature at the three chosen stations are shown in 
Figs.~\ref{fig:Salt_YO_2014_2015} and ~\ref{fig:Temp_YO_2014_2015} respectively. 
If the salinity of the stations OF-5 and OF-1 are compared (middle and lower panel in Fig. ~\ref{fig:Salt_YO_2014_2015}) it can be seen that the variations at for instance 40 m depth in Breiangen follow the variations further out in the fjord, but is approximately 1 psu fresher.
This indicate a relatively good water exchange in the outer part of the fjord system.
The water masses below sill depth in the Drammensjord is different. 
The salinity below 40 m depth is lower than at the same depth in Breiangen, and the salinity change very little with time.
These are clear signs of a stagnant water mass, and is expected given the shallow sill 
depth of only 12 m at Svelvik.   

The seasonal temperature changes in the surface layer is slowly diffused down in the water masses in the outer part of the fjord system (middle and lower panel in Fig.~\ref{fig:Temp_YO_2014_2015}).
The temperature at 100 m depth has a seasonal change, but the maximum value is shifted in time, so the highest temperatures are found in the start of January at this depth.
As seen in the upper panel in Fig.~\ref{fig:Temp_YO_2014_2015} the temperature variations 
in the surface layer in the Drammensfjord is prevented from penetrating further down than
about 40 m depth due to the low vertical diffusivity.

Fig.~\ref{fig:Salt_Mod_2014_2015} shows the salinity at the three chosen stations extracted from the model.
The water exchange in the model between the two stations OF-5 and OF-1 is relatively good, and the variations at OF-1 follow the variations further out in the fjord.
This was the same we saw from the observations.
The salinity however is to high at mid depth and this can as mentioned above be due to
wrong open boundary input or vertical diffusivity.
The upper panel in Fig.~\ref{fig:Salt_Mod_2014_2015} shows that the deep water masses in the Drammensfjord have a high salinity at model initialisation, but fresh water from the surface is quickly mixed down.   

Fig.~\ref{fig:Temp_Mod_2014_2015} shows the temperature at the three chosen stations extracted from the model.
When the modelled temperature evolution in the outer part of the fjord system is compared with the observations, the vertical mixing seems to be too low, and the heating of the surface water during summer do not penetrate deep enough during the winter.
While inside the Svelvik Sill the surface waters are mixed down too deep.


\begin{figure}[tbh]
\centerline{
\includegraphics*[trim=0cm 0cm 0cm 0cm,clip=true,width=\textwidth]{Figurer/Salt_YO_2014_2015}}
\caption{\small
Observed salinity at three stations in the Oslofjord. Contour lines mark 20, 30, and 34 psu. The white vertical lines indicate the positions when CTD casts were taken. 
}
\label{fig:Salt_YO_2014_2015}
\end{figure}

\begin{figure}[tbh]
\centerline{
\includegraphics*[trim=0cm 0cm 0cm 0cm,clip=true,width=\textwidth]{Figurer/Temp_YO_2014_2015}}
\caption{\small
Observed temperature at three stations in the Oslofjord. Contour lines mark 5 and 10 $^o$ C. The white vertical lines indicate the positions when CTD casts were taken. 
}
\label{fig:Temp_YO_2014_2015}
\end{figure}

\begin{figure}[tbh]
\centerline{
\includegraphics*[trim=0cm 0cm 0cm 0cm,clip=true,width=\textwidth]{Figurer/Salt_Mod_2014_2015}}
\caption{\small
Modelled salinity at three stations in the Oslofjord. Contour lines mark 20, 30, and 34 psu.
}
\label{fig:Salt_Mod_2014_2015}
\end{figure}

\begin{figure}[tbh]
\centerline{
\includegraphics*[trim=0cm 0cm 0cm 0cm,clip=true,width=\textwidth]{Figurer/Temp_Mod_2014_2015}}
\caption{\small
Modelled temperature at three stations in the Oslofjord. Contour lines mark 5 and 10 $^o$ C.
}
\label{fig:Temp_Mod_2014_2015}
\end{figure}


\clearpage 

\subsubsection{Water temperature near \AA sg\aa rdstrand}

The temperature observations from Scanmar AS are compared with simulated data extracted from 1.15 meters depth at approximately the same location as the observations.

Time series reveal that the simulated and observed temperature are in fairly good agreement (Fig.~\ref{fig:temp_2015}). During winter and spring, but the model underestimate the temperature in the summer and fall with a few degrees  and \ref{fig:temp-QQ_scatter}). The model captures the timing of the daily variations in temperature, but seems to overestimate heating and cooling causing too large daily variations (Fig.~\ref{fig:temp_jun2015}). 

2014 had a warmer summer than 2015. This is evident in both the observations and the simulations (Tab.~\ref{tab:temp}). The summer months of 2014 also had the largest variance in both the observed and the simulated temperature during 2014 and 2015. Generally, the simulated monthly temperature had a larger variance than the observed monthly temperature. The mean of the observed and simulated temperatures are 10.0$^o$C and 8.6$^o$C respectively, while the variances are 27.2$^o$C and 21.1$^o$C respectively.  

\begin{figure}[ht]
%\centerline{
%\includegraphics*[trim=0cm 0cm 0cm 0cm,clip=true,width=\textwidth]{Figurer/Temperatur_2014}}
\centerline{
\includegraphics*[trim=0cm 0cm 0cm 0cm,clip=true,width=\textwidth]{Figurer/Temperatur_2015}}
\caption{\small
Time series of observed and simulated temperature at \AA sg\aa dstrand. The difference is smoothed over 10 days.}
\label{fig:temp_2015}
\end{figure}

\begin{figure}[htb]
\centerline{
\includegraphics*[trim=1cm 0cm 1cm 0cm,clip=true,width=0.7\textwidth]{Figurer/Temperatur_QQ_scatter}}
\caption{\small
Combined QQ- and scatter plot of observed and simulated temperature at \AA sg\aa dstrand.}
\label{fig:temp-QQ_scatter}
\end{figure}

\begin{figure}[htb]
\centerline{
\includegraphics*[trim=0cm 0cm 0cm 0cm,clip=true,width=\textwidth]{Figurer/Temperatur_jun2015}}
\caption{\small
Time series of observed and simulated temperature at \AA sg\aa dstrand in June 2015.}
\label{fig:temp_jun2015}
\end{figure}

\newpage 

\begin{table}
\caption{Monthly statistics for observed and simulated temperature at \AA sg\aa rdstrand.}
\label{tab:temp}
\centering
\begin{tabular}{|ll|rrr|rrr|}
\hline 
&& \multicolumn{3}{|c|}{\bf 2014} & \multicolumn{3}{|c|}{\bf 2015} \\
&& {\bf quantity} & {\bf mean} & {\bf variance}  
& {\bf quantity} & {\bf mean} & {\bf variance} \\ \hline 
\small Jan & obs & 558 & 2.5 & 6.2 & 558 & 4.9 & 2.5 \\
\small     & sim & 745 & - & - & 745 & 4 & 0.7 \\
\small Feb & obs & 504 & 1.8 & 0.5 & 504 & 4 & 1.8 \\
\small     & sim & 673 & - & - & 673 & 3.6 & 0.6 \\
\small Mar & obs & 496 & 3.7 & 0.6 & 496 & 4.2 & 0.4 \\
\small     & sim & 745 & - & - & 745 & 4.7 & 0.3 \\
\small Apr & obs & 515 & 7.5 & 4.1 & 515 & 7 & 2.1 \\
\small     & sim & 721 & 7.6 & 7.1 & 721 & 7.4 & 2.8 \\
\small May & obs & 544 & 12.1 & 9.3 & 544 & 10.3 & 1.3 \\
\small     & sim & 745 & 11.3 & 10.4 & 745 & 9.9 & 3.5 \\
\small Jun & obs & 531 & 16.2 & 4.2 & 531 & 14 & 3.9 \\
\small     & sim & 721 & 15.5 & 8.3 & 721 & 12.8 & 6 \\
\small Jul & obs & 558 & 20.2 & 10.5 & 558 & 17.1 & 2 \\
\small     & sim & 745 & 17 & 24.3 & 745 & 14.6 & 5.1 \\
\small Aug & obs & 512 & 20 & 3.2 & 512 & 18.2 & 1.3 \\
\small     & sim & 745 & 16.4 & 5.6 & 745 & 15.5 & 8.7 \\
\small Sep & obs & 536 & 16.3 & 1.9 & 536 & 15.1 & 1.2 \\
\small     & sim & 721 & 12.9 & 8.9 & 721 & 12.7 & 2.5 \\
\small Oct & obs & 557 & 12.3 & 1.7 & 557 & 10.6 & 0.8 \\
\small     & sim & 745 & 9 & 1.5 & 745 & 8.6 & 2.7 \\
\small Nov & obs & 540 & 8.3 & 2.9 & 540 & 8.8 & 2.4 \\
\small     & sim & 721 & 6.3 & 2 & 721 & 6.1 & 1.3 \\
\small Dec & obs & 490 & 4.4 & 3.6 & 490 & 7.8 & 0.4 \\
\small     & sim & 733 & 3.4 & 1.5 & 733 & 4.8 & 0.8 \\
\hline
\end{tabular}
\end{table}

\clearpage

\subsubsection{Water temperature in the Inner Oslofjord}

The observed and simulated temperature at three beaches in the Inner Oslofjord are in relatively good agreement (Fig.~\ref{fig:badetemp_2014} - \ref{fig:badetemp_2015}). 

Close to the shoreline and only 40 cm under the surface, the temperature is heavily influenced by the weather situation and local circulation patterns. 

The temperature differences during the day are larger in the model than in the observations. The observed temperature increases 1-3 degrees from 09:00 to 18:00 and is not measured during the night, while the modelled temperature increases up to six degrees from 06:00 to 23:00. The fact that temperature is not measured during the night, but only from 09:00 to 18:00, might explain differences i temperature rise during the day, but the difference might indicate too much heating in the model.

%Since the temperature is observed close to the shoreline, some near river outlets, and only 40 cm under the surface, the temperature ise heavily influenced by the weather situation and local circulation patterns. The model is not expected to capture such detailed effects. Still there are similarities between the modelled and the observed temperatures both in temperature level and in fluctuations. 

During the summer 2014 the model predicts higher temperatures at Sj\o strand than was observed. The observations in Hvalstrand have some of the same trends as the modelled temperature with temperatures up to 25 degrees. The air temperatures in 2014 was higher than in 2015 and resulted in higher water temperatures, especially in shallow areas. 

\begin{figure}[ht]
\centerline{
\includegraphics*[trim=0 0 0 0,clip=true,width=\textwidth]{Figurer/badetemp_2014}
}
\caption{\small
The observed and modelled temperature at three beaches in the Inner Oslofjord during the summer 2014}
\label{fig:badetemp_2014}
\end{figure}

\begin{figure}[ht]
\centerline{
\includegraphics*[trim=0 0 0 0,clip=true,width=\textwidth]{Figurer/badetemp_2015}
}
\caption{\small
The observed and modelled temperature at three beaches in the Inner Oslofjord during the summer 2015}
\label{fig:badetemp_2015}
\end{figure}


%\subsubsection{Ferrybox - Andr\'{e}?}
%To be continued... Andr\'{e}?
%
%\begin{figure}[ht]
%\centerline{
%\includegraphics*[trim=1cm 0cm 1cm 0cm,clip=true,height=10cm]{Figurer/FjordOs_with_FA_track}}
%\caption{\small
%The track of Color Fantasy.}
%\label{fig:Ferrybox_track}
%\end{figure}
%
%\begin{figure}[ht]
%\centerline{
%\includegraphics*[trim=1cm 0cm 1cm 0cm,clip=true,width=.5\textwidth]{Figurer/FjordOs_vs_Ferrybox_TEMP}
%\includegraphics*[trim=1cm 0cm 1cm 0cm,clip=true,width=.5\textwidth]{Figurer/FjordOs_vs_Ferrybox_SALT}}
%\caption{\small
%Simulated daily mean compared with observations from ferryboxes for temperature (left) and salinity (right).}
%\label{fig:Ferrybox_temp_salt}
%\end{figure}


\clearpage 

\subsection{Drifting lanes}
\subsubsection{Godafoss}
\label{sect:godafoss_model}
Since the time period of the model does not cover the time period when the container ship Godafoss ran aground, we have instead released drifters in our model from the position of the oil spill from Godafoss for a period of one year. We argue that over such a long period of time, there will be at least one weather and current situation similar to the weather and currents experienced during the Godafoss release. By showing the models ability to advect oil to the observed locations during a one year run, we at least show that our ocean model is capable of recreating the right drift patterns.

To simulate the drift of the oil, we have applied the open source trajectory-model OpenDrift. This is a trajectory model under development at MET Norway, and is described by its developers as "a software for modelling the trajectories and fate of objects or substances drifting in the ocean, or even in the atmosphere". It is distributed under a GPL v2.0 license, and is available on GitHub\footnote{https://github.com/knutfrode/opendrift}.

The OpenDrift model was forced with currents from the FjordOs model and with wind from the Arome-MetCoOp 2.5km (Arome2.5) atmospheric model (the same atmospheric model was used as forcing when running the FjordOs hindcast). We have also provided daily mean currents from NorKyst-800m outside of the FjordOs-model, to properly treat the particles that are advected out of the FjordOs-model, so they can re-enter at the correct location. We can tune a number of parameters when running OpenDrift, e.g. random walk and the wind drift factor. Random walk was not used in our simulations of drifting lanes, and the wind drift factor was set to $0.01$ (i.e. $1\%$). Modelled drifters has been released once per hour from April 1st 2015 to April 1st 2016, a total number of 8760 particles. The lifetime of each particle is set to 15 days, i.e. after 15 days the particle is deactivated. This is done to reduce the computational cost of advecting a large number of particles. 

The results from the drift model can be viewed in Figure \ref{fig:opendrift_godafoss_time} - \ref{fig:opendrift_godafoss_endpos}. In Figure \ref{fig:opendrift_godafoss_time}, we show the number of hours from the particles were released until they reached different areas of the domain. Figure \ref{fig:opendrift_godafoss_conc} show the likelihood of an area experiencing oil. This is done by counting the number of particles that has been inside each grid cell during the simulation, and presenting this as concentrations. Figure \ref{fig:opendrift_godafoss_endpos} depicts the same as the two previous figures, but only showing the values in the final position of each trajectory. This is to better visualise stranding sites. All these calculations are done on a $140m$ by $140m$ grid.

When comparing the results in Figure \ref{fig:opendrift_godafoss_time} - \ref{fig:opendrift_godafoss_endpos} to the observed oil in Figure \ref{fig:godafoss_oil}, we clearly see that there are some similarities between the observations and the simulations. Especially by looking at Figure \ref{fig:opendrift_godafoss_endpos}, right panel, we can see that a substantial number of particles are transported westward from the release position at Hvaler, and strand along the Vestfold coast, the islands around Tj{\o}me, and the F{\ae}rder lighthouse. We would also like to point out the very low number of particles that strand at the peninsula west of Stavern (the area far left in Figure \ref{fig:opendrift_godafoss_endpos}). This corresponds well with where the oil was observed.
In the left panel of Figure \ref{fig:opendrift_godafoss_endpos}, we provide the shortest time from the release of particles, to stranding. We would like to point out two areas: Stavern (western part of Figure 
\ref{fig:opendrift_godafoss_endpos}) and Bustein (northwest of the F{\ae}rder lighthouse). In Figure \ref{fig:godafoss_oil}, the oil at Bustein was discovered two days after the initial spill, while at Stavern (Korntin) the oil was observed after five days. This corresponds well with the timing in Figure \ref{fig:opendrift_godafoss_endpos}.

\begin{figure}[ht]
\centerline{
\includegraphics*[width=.5\textwidth]{Figurer/opendrift/opendrift_godafoss_shortest_time_crop}
\includegraphics*[width=.5\textwidth]{Figurer/opendrift/opendrift_godafoss_shortest_time_zoom_crop}
}
\caption{\small
Number of hours from particle release, to particle is inside a given $140x140m$ area for the Godafoss scenario. Based on one year (April 1st 2015 - April 1st 2016) of simulations, with a maximum lifetime of 15 days of the released particles. This amounts to a total number of $8760$ released particles. Please note the different scales of each figure.}
\label{fig:opendrift_godafoss_time}
\end{figure}

\begin{figure}[ht]
\centerline{
\includegraphics*[width=.5\textwidth]{Figurer/opendrift/opendrift_godafoss_consentration_crop}
\includegraphics*[width=.5\textwidth]{Figurer/opendrift/opendrift_godafoss_consentration_zoom_crop}
}
\caption{\small
Same as Figure \ref{fig:opendrift_godafoss_time}, but for number of particles that has been inside a given $140x140m$ area for the Godafoss scenario. Please note the different scales of each figure.}
\label{fig:opendrift_godafoss_conc}
\end{figure}

\begin{figure}[ht]
\centerline{
\includegraphics*[width=.5\textwidth]{Figurer/opendrift/opendrift_godafoss_shortest_time_zoom_endpos_crop}
\includegraphics*[width=.5\textwidth]{Figurer/opendrift/opendrift_godafoss_consentration_zoom_endpos_crop}
}
\caption{\small
Similar to Figure \ref{fig:opendrift_godafoss_time} and \ref{fig:opendrift_godafoss_conc}, but only showing end position of each trajectory: Number of days from particle release (note different time scale) to particle in given area (left panels), and the number of particles that have been inside a given $140x140m$ area (right panel). Please note the different scales of each figure.}
\label{fig:opendrift_godafoss_endpos}
\end{figure}

\clearpage 
\subsubsection{Surface drifters}
We evaluate the models ability to recreate the drifter trajectories by using a skill-score, as described in \cite{liu:2011}: 

$$s=\displaystyle\sum_{i=1}^{N} d_i / \displaystyle\sum_{i=1}^{N} l_{oi}$$

\[ ss = 
  \begin{cases}
    1 - \frac{s}{n}   & \quad (s \leq n)\\
    1                & \quad (s > n)   \\ 
  \end{cases}
\]
where $n$ is a tolerance threshold, $d_i$ is the separation distance between the modelled and observed endpoints of the Lagrangian trajectories at time step $i$ after the initialisation (virtual particle release), $l_{oi}$ is length of the observed trajectory, and $N$ is the total number of time steps.

A skill-score of 1 means a perfect match, while a score of 0 means no skill. %We have used a tolerance treshold $n=1$.
The skill-score for each drifter compared to trajectories from OpenDrift based on both the FjordOs and NorKyst-800m models is given in Table \ref{tab:skillscore_full}, \ref{tab:skillscore_1hr}, \ref{tab:skillscore2014_1h}, and \ref{tab:skillscore2014_24h}. For simulated trajectories during the September 2015 cruise (Table \ref{tab:skillscore_full} and \ref{tab:skillscore_1hr}), both currents and wind drift has been applied, while for the trajectories of September 2014 (Table \ref{tab:skillscore2014_1h} and \ref{tab:skillscore2014_24h}) only currents have been applied to produce the trajectories.

Examination of the values for skill-score, and visual inspection of the trajectories themselves, do not provide a clear conclusion on which ocean model performs better. During the September 2014 release, the FjordOs model performs better that the NorKyst-800m, but there is a clear weakness here since we do not have hourly data from NorKyst-800m available. Also, there are only two drifters. The FjordOs model has the highest skill-score based on both hourly and daily average data, and was the only model to transport the drifters as far south as the observed drift. Based on this, we think it is safe to conclude that the FjordOs model had an overall current situation that was close to the real situation. The currents in NorKyst-800m is believed to be too weak during this drifter release.

During the September 2015 cruise we released 10 drifters. The NorKyst-800m model has an overall higher skill-score than FjordOs when we take into account the entire trajectory, and  also when we only consider the first hour of the trajectory. When we do the visual comparison between the trajectories in Figure \ref{fig:opendrift_trajectories1} - \ref{fig:opendrift_trajectories3}, it is very tempting to point out that the trajectories based on the FjordOs model more often compares well to the observed trajectory than what the one based on NorKyst-800m does, but we will leave this up to the reader to decide.

\begin{table}
\begin{center}
  \begin{tabular}{ | c | c | c |}
    \hline
    {\bf Drop no.} & {\bf FjordOs} & {\bf NorKyst-800m} \\ \hline
    1 & 0.77 (11) & 0.68 (11) \\ 
    4 & 0.78 (2) & 0.53 (1) \\
    6 & 0.00 (3) & 0.84 (11) \\
    8 & 0.82 (21) & 0.58 (21) \\
    51 & 0.58 (3) & 0.73 (15) \\
    52 & 0.38 (4) & 0.50 (1) \\
    61 & 0.58 (3) & 0.70 (8) \\
    91 & 0.51 (8) & 0.50 (12) \\
    101 & 0.61 (15) & 0.78 (15) \\
    102 & 0.59 (6) & 0.75 (7) \\ \hline
    {\bf Avg.} & {\bf 0.56} & {\bf 0.66} \\
    \hline
  \end{tabular}
\caption{Skill-score of drifter trajectories released during the September 2015 cruise. We have used a tolerance threshold $n=2$ to get positive values for most trajectories. The number in parenthesis indicates how many hours the model trajectory is.}
\label{tab:skillscore_full}
\end{center}
\end{table}

\begin{table}
\begin{center}
  \begin{tabular}{ | c | c | c |}
    \hline
    {\bf Drop no.} & {\bf FjordOs} & {\bf NorKyst-800m} \\ \hline
    1 & 0.17 & 0.14 \\ 
    4 & 0.55 & 0.45 \\
    6 & 0.07 & 0.46 \\
    8 & 0.00 & 0.00 \\
    51 & 0.49 & 0.58 \\
    52 & 0.37 & 0.47 \\
    61 & 0.49 & 0.58 \\
    91 & 0.73 & 0.67 \\
    101 & 0.63 & 0.78 \\
    102 & 0.87 & 0.77 \\ \hline
    {\bf Avg.} & {\bf 0.43} & {\bf 0.49} \\
    \hline
  \end{tabular}
\caption{Same as Table \ref{tab:skillscore_full}, but only considering the first hour of the trajectory.}
\label{tab:skillscore_1hr}
\end{center}
\end{table}

\begin{figure}[ht]
\centerline{
\includegraphics*[width=.5\textwidth]{Figurer/opendrift/skillscore/drop1i0}
\includegraphics*[width=.5\textwidth]{Figurer/opendrift/skillscore/drop4i7}
}
\centerline{
\includegraphics*[width=.5\textwidth]{Figurer/opendrift/skillscore/drop6i12}
\includegraphics*[width=.5\textwidth]{Figurer/opendrift/skillscore/drop8i1}
}
\caption{\small
Figures showing comparisons of modelled and observed drifter trajectories during the September 2015 cruise. Modelled trajectories has been simulated using both NorKyst-800m and the FjordOs model. Values along X- and Y-axis indicate distance in kilometres. Top-left is drop 1, top-right is drop 4, bottom-left is drop 6 and bottom-right is drop 8.}
\label{fig:opendrift_trajectories1}
\end{figure}

\begin{figure}[ht]
\centerline{
\includegraphics*[width=.5\textwidth]{Figurer/opendrift/skillscore/drop51i4}
\includegraphics*[width=.5\textwidth]{Figurer/opendrift/skillscore/drop52i8}
}
\centerline{
\includegraphics*[width=.5\textwidth]{Figurer/opendrift/skillscore/drop61i5}
\includegraphics*[width=.5\textwidth]{Figurer/opendrift/skillscore/drop91i11}
}
\caption{\small
Same as Figure \ref{fig:opendrift_trajectories1}, but for the following drifters: Top-left is drop 51, top-right is drop 52, bottom-left is drop 61 and bottom-right is drop 91.}
\label{fig:opendrift_trajectories2}
\end{figure}

\begin{figure}[ht]
\centerline{
\includegraphics*[width=.5\textwidth]{Figurer/opendrift/skillscore/drop101i6}
\includegraphics*[width=.5\textwidth]{Figurer/opendrift/skillscore/drop102i9}
}
\caption{\small
Same as Figure \ref{fig:opendrift_trajectories1} and \ref{fig:opendrift_trajectories2}, but for the following drifters: Left is drop 101 and right is drop 102.}
\label{fig:opendrift_trajectories3}
\end{figure}


\begin{table}
\begin{center}
  \begin{tabular}{ | c | c | c |}
    \hline
    {\bf Drop no.} & {\bf FjordOs 1h} & {\bf NorKyst-800m 24h} \\ \hline
    1 & 0.79 (15) & 0.73 (15) \\ 
    2 & 0.84 (19) & 0.76 (19) \\ \hline
    {\bf Avg.} & {\bf 0.81} & {\bf 0.75} \\
    \hline
  \end{tabular}
\caption{Same as Table \ref{tab:skillscore_full}, but for the drifter release during the September 2014 cruise. Please note that the FjordOs model has hourly resolution for current data, while the NorKyst-800m model has daily averages only. This is due to the availability of the data from NorKyst-800m.}
\label{tab:skillscore2014_1h}
\end{center}
\end{table}
\begin{table}
\begin{center}
  \begin{tabular}{ | c | c | c |}
    \hline
    {\bf Drop no.} & {\bf FjordOs 24h} & {\bf NorKyst-800m 24h} \\ \hline
    1 & 0.85 (15) & 0.73 (15) \\ 
    2 & 0.91 (19) & 0.76 (19) \\ \hline
    {\bf Avg.} & {\bf 0.88} & {\bf 0.75} \\
    \hline
  \end{tabular}
\caption{Same as Table \ref{tab:skillscore2014_1h}, but modelled trajectories from FjordOs are based on daily average ocean currents to get a more fair comparison between the models.}
\label{tab:skillscore2014_24h}
\end{center}
\end{table}


\begin{figure}[ht]
\centerline{
\includegraphics*[width=.5\textwidth]{Figurer/opendrift/skillscore/2014drop1i0}
\includegraphics*[width=.5\textwidth]{Figurer/opendrift/skillscore/2014drop1i0_24h}
}
\centerline{
\includegraphics*[width=.5\textwidth]{Figurer/opendrift/skillscore/2014drop2i1}
\includegraphics*[width=.5\textwidth]{Figurer/opendrift/skillscore/2014drop2i1_24h}
}
\caption{\small
Same as Figure \ref{fig:opendrift_trajectories1}, but showing the September 2014 release. Top-left is drop 1, top-right is drop 1 based on 24h average data for FjordOs, bottom-left is drop 2 and bottom-right is drop 2 based on 24h average data for FjordOs.}
\label{fig:opendrift_trajectories2014}
\end{figure}


\clearpage 
