%\newpage
%\clearpage
\section{Evaluation}
\label{sec:evalu}
% % % % % % % % % % % % %
\subsection{Water level and tide}
\label{subsec:wlevele}
Time series of water level from the three tide gauge stations (Section \ref{subsec:wlevelo}) were extracted for the period April 2014 through December 2015. To evaluate the model performance we also extracted time series of water level at locations near the three stations from the model simulation for the same time period. We emphasize that the tidal forcing we used at the southern open boundary of the FjordOs model includes eleven tidal constitutents only, and that the tidal forcing was adjusted by use of the observed tide at Viker close to the southern boundary \citep{roed:etal:2016}. 

\begin{figure}[htb]
  \begin{center}
    \begin{tabular}{c}
      \includegraphics*[trim=3cm 0cm 2.5cm 0cm,clip=true,width=15cm]{Figurer/Oscarsborg_Tide_selected_jan15} \\ 
      \includegraphics*[trim=3cm 0cm 2.5cm 0cm,clip=true,width=15cm]{Figurer/Oscarsborg_WL_rest_jan15} \\ 
    \end{tabular}
    \caption{\small Simulated (black) and observed (red) time series of the combined tidal water elevation (upper panel) and residual water level (lower panel) at Oscarsborg (cf. Figure~\ref{fig:kart_obs}) for the month of January 2015.}
    \label{fig:Waterlevel_jan15}
  \end{center}
\end{figure}

To compare model and observation we first analysed the two time series by use of t\_tide \citep{pavlo:etal:2002} to extract the amplitudes and phases of the individual harmonic tidal components. Next we superimposed the water elevation due to each of the eleven tidal constituents included in the tidal forcing to obtain what we term the combined water elevation. We finally subtracted the resulting combined water elevation from the total water elevation of each time series to derive comparabel ``residual'' water elevations. As revealed by Figure~\ref{fig:Waterlevel_jan15} the combined water elevation compares favourably with the observation. This is also true for the residual, but to a lesser degree. The latter is to be expected since the observations include all the tidal water elevations due to tides of longer and shorter periods than the eleven included in the tidal forcing. Note that to be able to discriminate between observations and simulated water levels the time series shown by Figure~\ref{fig:Waterlevel_jan15} is truncated to January 2015.

Of the eleven tidal components included in the tidal forcing M2 is, in terms of amplitude, the dominant constituent. It is therefore noteworthy, as revealed by Table~\ref{tab:Tide} on page \pageref{tab:Tide}, that the model represents this influential constituent to a very satisfactory degree both regarding amplitude and phase. This is true even at the stations Oscarsborg and Oslo that are both far from the models southern boundary. We also note that the constituents S2, N2 and O1 contribute, and that their contributions are in fairly good agreement with the observations as well. 

Table~\ref{tab:Tide} and Figure~\ref{fig:Waterlevel_tide} also show that even the longer period tidal constituents, e.g., SS and SSA, are to some degree picked up by the model despite the fact that they are not incorporated in the tidal forcing. This may be explained by the fact that we in addition to the tides also use daily mean water level, hydrography and currents extracted from the NorKyst800 model as forcing on the southern boundary \citep{roed:etal:2016}. In contrast, and as expected tides of periods shorter than 6 hours are not picked up by the model at all (Figure~\ref{fig:Waterlevel_tide}).

\begin{figure}[tbh] 
	\centerline{ \includegraphics*[trim=3cm 0cm 2.5cm 0cm,clip=true,width=\textwidth]{Figurer/Oscarsborg_Tide_not_included} } 
	\caption{\small Time series at Oscarsborg of the tidal components not included in the tidal forcing.} 
	\label{fig:Waterlevel_tide} 
\end{figure} 

\begin{table}[htb] 
	\caption{\small Simulated and observed tidal amplitudes and phases at three tide gauge stations for selected tidal components sorted by period.} 
	\label{tab:Tide} 
	\centering 
	\begin{tabular}{|c|c|l|cc|cc|cc|c|} 
\hline  
	  &	&	& \multicolumn{2}{|c|}{\small \bf Viker} & \multicolumn{2}{|c|}{\small \bf Oscarsborg} & \multicolumn{2}{|c|}{\small \bf Oslo} & {\small \bf Included} \\  
{\small \bf Comp.} & {\small \bf Period} & {\small \bf sim/} & {\small \bf amp.} & {\small \bf phase.} & {\small \bf amp.} & {\small \bf phase.} & {\small \bf amp.} & {\small \bf phase.} & {\small \bf in tidal} \\ 
	    & {\small \bf [h]} & {\small \bf obs} & {\small \bf [cm]} & {\small \bf [deg]} & {\small \bf [cm]} & {\small \bf [deg]} & {\small \bf [cm]}   & {\small \bf [deg]} & {\small \bf forcing} \\ \hline 
\small SA   & \small 8764    & \small sim & \small 15.5 & \small 284 & \small 15.6 & \small 286 & \small 15.4 & \small 286 & \small no  \\
\small      &        	     & \small obs & \small 10.0 & \small 319 & \small 11   & \small 322 & \small 11.4 & \small 324 &    	\\
\hline
\small SSA  & \small 4382    & \small sim & \small 8.8  & \small 197 & \small 9.2  & \small 200 & \small 9.4  & \small 200 & \small no  \\
\small      &		     & \small obs & \small 7.5  & \small 188 & \small 8.0  & \small 189 & \small 8.2  & \small 190 &    	\\
\hline
\small Q1   & \small 26.8684 & \small sim & \small 0.0  & \small 231 & \small 0.0  & \small 216 & \small 0.1  & \small 215 & \small no	\\
\small      &        	     & \small obs & \small 1.1  & \small 190 & \small 1.2  & \small 198 & \small 1.3  & \small 200 &    	\\
\hline
\small O1   & \small 25.8193 & \small sim & \small 3.5  & \small 337 & \small 3.8  & \small 339 & \small 3.8  & \small 339 & \small yes	\\
\small      &        	     & \small obs & \small 2.2  & \small 277 & \small 2.3  & \small 281 & \small 2.4  & \small 282 &    	\\
\hline
\small P1   & \small 24.0659 & \small sim & \small 0.6  & \small 322 & \small 0.6  & \small 334 & \small 0.7  & \small 342 & \small yes	\\
\small      &        	     & \small obs & \small 0.2  & \small 129 & \small 0.3  & \small 102 & \small 0.4  & \small 97  &    	\\
\hline
\small K1   & \small 23.9345 & \small sim & \small 0.2  & \small 187 & \small 0.1  & \small 175 & \small 0.2  & \small 157 & \small yes	\\
\small      &        	     & \small obs & \small 0.4  & \small 127 & \small 0.7  & \small 130 & \small 0.8  & \small 130 &    	\\
\hline
\small N2   & \small 12.6584 & \small sim & \small 3.0  & \small 69  & \small 3.5  & \small 75  & \small 3.7  & \small 76  & \small yes	\\
\small      &        	     & \small obs & \small 3.0  & \small 60  & \small 3.4  & \small 76  & \small 3.6  & \small 80  &    	\\
\hline
\small M2   & \small 12.4206 & \small sim & \small 11.5 & \small 105 & \small 13.2 & \small 112 & \small 13.9 & \small 114 & \small yes \\
\small      &        	     & \small obs & \small 11.9 & \small 105 & \small 13.8 & \small 121 & \small 14.4 & \small 125 &    	\\
\hline
\small S2   & \small 12.0000 & \small sim & \small 3.3  & \small 64  & \small 3.9  & \small 69  & \small 4.2  & \small 70  & \small yes \\
\small      &        	     & \small obs & \small 2.9  & \small 46  & \small 3.3  & \small 65  & \small 3.5  & \small 69  &    	\\
\hline
\small K2   & \small 11.9672 & \small sim & \small 1.6  & \small 10  & \small 2.0  & \small 13  & \small 2.1  & \small 15  & \small yes \\
\small      &        	     & \small obs & \small 0.7  & \small 45  & \small 0.8  & \small 66  & \small 0.9  & \small 66  &    	\\
\hline
\small MN4  & \small 6.2692  & \small sim & \small 0.2  & \small 5   & \small 0.5  & \small 32  & \small 0.6  & \small 35  & \small yes	\\
\small      &        	     & \small obs & \small 0.4  & \small 249 & \small 0.6  & \small 289 & \small 0.7  & \small 297 &    	\\
\hline
\small M4   & \small 6.2103  & \small sim & \small 1.0  & \small 355 & \small 1.9  & \small 18  & \small 2.5  & \small 23  & \small yes	\\
\small      &        	     & \small obs & \small 1.2  & \small 281 & \small 1.8  & \small 324 & \small 2.3  & \small 332 &    	\\
\hline
\small MS4  & \small 6.1033  & \small sim & \small 0.6  & \small 80  & \small 1.2  & \small 107 & \small 1.6  & \small 111 & \small yes	\\
\small      &        	     & \small obs & \small 0.3  & \small 360 & \small 0.5  & \small 44  & \small 0.7  & \small 56  &    	\\
\hline
\end{tabular}
\end{table}

Finally we note, as revealed by Table~\ref{tab:Tide}, that the observed M2 amplitude increases from south to north. This is reflected in the simulations as well as displayed by Figure~\ref{fig:M2field}. It is interesting to note that the lowest M2 amplitude is found in the Drammensfjord north of the threshold in Svelvik. In fact the M2 phase has a sudden increase at the thresholds of Svelvik and Dr{\o}bak. The same is true for the majority of the other relevant tidal components (not shown).

\begin{figure}[hb] 
	\centerline{ 
		\includegraphics*[trim=1cm 0cm 0cm 0cm,clip=true,width=0.49\textwidth]{Figurer/M2amp_felt}  
		\includegraphics*[trim=0.8cm 0cm 0cm 0cm,clip=true,width=0.49\textwidth]{Figurer/M2fase_felt} 
		} 
	\caption{\small Simulated fields of M2 amplitude (left-hand panel) and phase (right-hand panel). Corresponding observed values for M2 amplitude and phase are marked with circles at Viker, Oscarsborg, and Oslo.} 
	\label{fig:M2field} 
\end{figure} 



% % % % % % % % % % % %
\clearpage 
\subsection{Currents}
\label{subsec:curree}
%================================= New subsection (ANS) ====================
\subsubsection{Currents in the Oslofjord}
\label{subsubsec:driving}
In essence currents in the Oslofjord are caused by tides, storm surges, and differences in temperature and salinity. The tides are due the gravitational pull of the moon and sun. It is more often than not the most dominant current and moves the whole water column. It fluctuates with same period as the tides and transport large amounts of water in and out of the fjord. It is therefore especially strong in narrow parts of the fjord like the Dr{\o}bak Sound and Sill (Oscarsborg) and the Svelvik Sill. The strength of the tidal currents depends on whether the effects of the sun and moon are in phase (spring tides) or out of phase (neap tides). Even though the mean total tidal amplitude is less than 20 cm in the Oslofjord, the tidal currents are up to 1 m/s due to the narrow straits and sill depths. 

The storm surge currents are due water level variations caused by atmospheric wind and pressure forces. It therefore fluctuates with the meteorological conditions. An example is when an atmospheric low pressure passes over the area. As with tides the storm surge currents affect the whole water column. The low air pressure cause water levels to rise and in addition the wind stress on the surface causes the water to a pile up of along the coast. If a storm surge event coincides with the spring tide the water level may become unusually high and lead to unusually strong currents. In connection with storm surge events, amplitudes of 1 m or higher are observed. One such extreme event happened in October 1987 when the water level in Oslo rose to 196 cm above normal water level. 

Differences in temperature and salinity causes water masses of different density to be formed. In turn this causes horizontal pressure differences that forces the water to move. As a result we get currents that vary with depth. This may happen if water masses of different density than the original fjord water enters from the open sea, freshwater is discharged to the fjord via rivers, or dense water is upwelled for instance due to atmospheric winds. We may then measure currents at depth without this being observed at the surface. Vertical variations in depth may also be caused by bathymetry which may prevent the water mass from moving, even if a horizontal pressure gradient is present.

Thus the currents in the Oslofjord varies both in time and in space. Since the tides and storm surges leads to currents of strength and direction that are independent of depth, it makes sense to split up the flow, say $\bu$, into a depth average component, say $\bu_0$ (henceforth referred to as the external or barotropic mode), and a depth varying deviation, say $u_n$ (henceforth refferred to as the internal or baroclinic mode), that is,
\be
	\label{eq:ubar_un}
		\bu = \bu_0 + \bu_n.
\ee
The barotropic mode is simply estimated by calculating the average over all depths, or 
\be
	\label{eq:ubar}
		\bu_0 = \frac{1}{H} \int_{-H}^{0} \bu dz. 
\ee

When rivers discharging freshwater to the Oslofjord the resulting current entrains water with higher salinity from below, causing the volume flow to increase and the salinity to decrease. An example of this is evident in \cite[][Figure 20]{roed:etal:2016} where the water from the Drammen river can be traced far out into the fjord. To conserve the volume, the entrainment of water leads to a counter current below the surface layer transporting new water into the area, to compensate for the loss of volume that the seaward surface flow represents. This compensating volume flux is often defined as the estuarine circulation. The resulting flow pattern is one example of a baroclinic current, where the horizontal pressure varies with depth. 

The Oslofjord has an open, southern boundary towards Skagerrak which lies in the north-eastern part of the North Sea. The circulation in the Skagerrak is on average counterclockwise with brackish outflow from the Baltic Sea \cite[]{rodhe96,svendsen96,albre:roed:2010}. This flow pattern generate horizontal pressure gradients near the mouth of the Oslofjord \cite[]{baals90}, and by mechanisms described by \cite{klinck81} variation in the wind pattern generate mean baroclinic flow events that that may be stronger than the barotropic tidal current, especially in the outer Oslofjord where the fjord width is broader.

Fluctuations in currents due to river discharge normally varies on a much longer time scale than the tidal and storm surge currents. To reveal the estuarine circulation or the mean baroclinic flow it is therefore useful to separate the current in a slowly varying part, say $\overline{u}$, and rapidly varying part, say $u'$, as detailed for instance by \cite{roed:fossu:2004}. Thus
\be
	\label{eq:utime}
		\bu = \overline{\bu} + \bu'
\ee
The mean flow may be estimated by calculating the average over a period much longer than the typical tidal period and wind periods, e.g., several days.
%================================= New subsection ends (ANS) ===============

%=================== New version of "two cross sections" (ANS) =============
\clearpage
\subsubsection{The two cross section: Filtvedt-Brenntangen and Sm{\aa}skj{\ae}r-Evje}
\label{subsubsec:filtve}
The first transect Filtvedt-Brenntangen is located at the entrance to the Dr{\o}bak Sound that forms the eastern branch of the inner Oslofjord (Figure~\ref{fig:kart_obs}). The fjord width in this location is 1.8 km, and the deepest part is found in the middle with a fjord depth of 208 m (Figure~\ref{fig:Two_transect}). The slope of the sea bed is very steep at the sidewalls of the channel. Note that the model bathymetry is a smoothed version of the real bathymetry which is actually steeper. Station Km1 is located on the west side of the channel, and the posistion where model results are extracted is indicated with a white vertical line in the figure.The true position of station Km1 is a little to the west where the real bathymetry has the same depth as in the model. In the same figure the along fjord current is shown with a colorbar where red indicated current into the fjord and blue current out of the fjord. The current is extracted from the model run from the date 24th October, 2014, when the mean flow was directed out of the fjord. At this instance the current is relativly uniform across the fjord, and observations at station Km1 can be said to be representaive for the whole transect.

The second transect Sm{\aa}skj{\ae}r-Evje is located further south, approximatly in the middle of the fjord (cf. Figure~\ref{fig:kart_obs}), where the fjord is about 10 km wide. The deepest part of the transect is found to the east of the middle, with a depth of 204 m. Station Rl1 (see Table~\ref{tab:Statnett} and Figure~\ref{fig:kart_obs}) is located to the west of the deepest part where the bottom slope is steep. The bathymetry of the transect is shown in the lower panel in Figure~\ref{fig:Two_transect}, where the model bathymetry is again plotted as a thick black line and the real batymetry as a thin grey line. The current speed along the fjord is shown in the transect. It is evident from the figure that the current is not uniform across the fjord, and the observations at station Rl1 can not be said to be representative for the whole transect.

\clearpage 

% FIGURE SHOWING THE TRANSECTS
\begin{figure}[ht]
	\centerline{
		\includegraphics*[trim=0 0 0 0,clip=true,width=\textwidth]{Figurer/AndreS/Two_transectas_cur}}
	\caption{\small A snapshot of the observed current speed across the two transect Filtvedt and Brenntangen (upper panel) and Sm{\aa}skj{\ae}r-Evje (bottom panel). The location of the transects are shown in Figure~\ref{fig:kart_obs}. The bathymetry of the model is drawn with a thick black line, while the real bathymetry is drawn with a thin grey line. The current speed is indicated with a colorbar ranging from red (into the fjord) to blue (out of the fjord). A thin black line indicates where the current speed is zero. The location of the stations listed by Table~\ref{tab:Statnett} is indicated with white vertical lines.}
	\label{fig:Two_transect}
\end{figure}

% FIGURE SHOWING CURRENT ARROWS AT 2m, 40m and 100m
\begin{figure}[ht]
	\centerline{
		\includegraphics*[trim=2cm 3cm 1cm 3.3cm,clip=true,height=5cm]{Figurer/Filtvedt_t4611_z2_current}
		\includegraphics*[trim=3.8cm 3cm 6cm 3.3cm,clip=true,height=5cm]{Figurer/Filtvedt_t4611_z40_current}
		\includegraphics*[trim=3.8cm 3cm 1cm 3.3cm,clip=true,height=5cm]{Figurer/Filtvedt_t4611_z100_current}}
	\caption{\small Simulated currents at 2 (left), 40, and 100 (right) meters depth at 10 October 2014 12:00. Note that the two plots to the right have the same colorbar.}
	\label{fig:Filtvedt-simcur}
\end{figure}

Note that the 3D current field is complex, the currents vary horizontally, vertically and with time. Figure~\ref{fig:Filtvedt-simcur} shows the simulated currents at three different depths near Filtvedt at 10 October 2014. In general, the currents in the upper layer are stronger than further down. In the upper layers the currents are towards north, at 40 meters depth the currents towards south, and at 100 meters depth towards north. Because of the complex flow pattern, the currents at a given coordinates cannot be taken at representative for the whole area, but current across the fjord between station Km1 an Kn2 is relativly uniform execept at 100 m depth. At 100 meters depth the currents at Filtvedt are weak and towards south even though the currents at 100 meters depth is generally stronger and towards north. At 40 meters depth the currents at Brenntangen is weaker than in the rest of the cross section. Note that the depth at Brenntangen is only 58 meters in the observations while in the depth is 46 meters at the corresponding point in the simulations. This is due to the smoothing of the model bathymetry. 

\clearpage 

The current variability at the six stations in these two transects are described by \cite{staalstrom:2015}. They found that the tides dominates in the northern transect, where the fjord width is narrow. The tides are less pronounced in the southern transect where the tidal variation is masked by the mean flow. In figure~\ref{fig:Km1} and~\ref{fig:Rl1} the current along the fjord based on observations and model results, for the two stations Km1 and Rl1 are shown. This illustrate that the station in the Filtvedt transect is dominated by a tidal signal, with a typical diurnal period, while the tidal signal is masked by the mean flow in the station in the Bast{\o}y transect. The same pattern is present in the model results.

In figure~\ref{fig:Km1_mean} and~\ref{fig:Rl1_mean} the mean flow, $\overline{u}$ from equation~\ref{eq:utime}, is estimated by applying a running mean with a period of 49 hours, on both the observations and the model results. At the Filtvedt transect the model do not capture the mean flow correctly (Figure~\ref{fig:Km1_mean}), and the mean flow is stronger in the model than in the observations.

The model do capture many of the baroclinic mean flow events observed at station Rl1 in the Bast{\o}y transect, espesially in the last part of the period shown in figure~\ref{fig:Rl1_mean}. It is promising that the model to some extent capture the mean flow events in the part of the fjord where this phenomena dominates. But it must be remarked that the model performance needs to be improved. These kind of events depends on the stratification in the model as well as fresh water input and the influence from the open boundary. It is expected that model performance concerning mean flow will be improved if the stratification in the model is improved. 
  
\clearpage 

\begin{figure}[ht]
	\centerline{
		\includegraphics*[trim=0 0 0 0,clip=true,width=\textwidth]{Figurer/AndreS/Km1_Obs_vs_Mod_v2}}
	\caption{\small Observations (upper panel) and model results (lower panel) from station Km1. The colorbar indicates the current speed along the channel, on a color scale where red is flow into the fjord, and blue out of the fjord. The black contourline shows where the current speed is zero. The shallowest observations is at 16 m depth. This depth is indicated in the lower panel with a horizontal line.}
	\label{fig:Km1}
\end{figure}

%\begin{figure}[ht]
%\centerline{
%\includegraphics*[trim=0 0 0 0,clip=true,width=\textwidth]{Figurer/Filtvedt_obs_cur}}
%\centerline{
%\includegraphics*[trim=0 0 0 0,clip=true,width=\textwidth]{Figurer/Filtvedt_sim_cur}}
%\caption{\small
%Observed (upper) and simulated (lower) currents at Filtvedt. Since the observations near the surface where dominated by noise, only depths larger than 40 meters are included in the upper plot. $z$ = 40 meters is marked with a black line in the lower plot. Note that the model depth is only 155 meters at the position the observations where performed.}
%\label{fig:Filtvedt-cur}
%\end{figure}

\clearpage 

\begin{figure}[ht]
	\centerline{
		\includegraphics*[trim=0 0 0 0,clip=true,width=\textwidth]{Figurer/AndreS/Rl1_Obs_vs_Mod_v2}}
	\caption{\small Observations (upper panel) and model results (lower panel) from station Rl1. The colorbar indicates the current speed along the channel, on a color scale where red is flow into the fjord, and blue out of the fjord. The black contourline shows where the current speed is zero. The shallowest observations is at 10 m depth. This depth is indicated in the lower panel with a horizontal line.}
	\label{fig:Rl1}
\end{figure}


\clearpage 

\begin{figure}[ht]
	\centerline{
		\includegraphics*[trim=0 0 0 0,clip=true,width=\textwidth]{Figurer/AndreS/Km1_Obs_vs_Mod_mean_v2}}
	\caption{\small Observed (upper panel) and simulated (lower panel) mean flow from station Km1. The mean flow is estimated by taking a running mean of 49 hours. The colorbar indicates the current speed along the channel, on a color scale where red is flow into the fjord, and blue out of the fjord. The black contourline shows where the mean flow is zero. The shallowest observations is at 16 m depth.
This depth is indicated in the lower panel with a horizontal line.}
	\label{fig:Km1_mean}
\end{figure}

\clearpage 

\begin{figure}[ht]
	\centerline{
		\includegraphics*[trim=0 0 0 0,clip=true,width=\textwidth]{Figurer/AndreS/Rl1_Obs_vs_Mod_mean_v2}}
	\caption{\small Observed (upper panel) and simulated (lower panel) mean flow from station Rl1. The mean flow is estimated by taking a running mean of 49 hours. The colorbar indicates the current speed along the channel, on a color scale where red is flow into the fjord, and blue out of the fjord. The black contourline shows where the mean flow is zero. The shallowest observations is at 10 m depth. This depth is indicated in the lower panel with a horizontal line.}
	\label{fig:Rl1_mean}
\end{figure}

\clearpage 

Now we will focus on the tidal currents in the cross section between Filtvedt and Brenntangen. In this location the tidal variation is dominating, and the current is relativly uniform across the channel. The mean flow ($\overline{u}$) as it is shown in Figure~\ref{fig:Km1_mean} is subtracted from the flow, leaving the residual parts according to (\ref{eq:utime}). The tidal part ($\widetilde{u}$) is extracted by analysing the current using t\_tide \cite[]{pavlo:etal:2002} that gives the tidal components for each depth. The same period in time is applied for both simulations and observations (mid-September to the end of November 2014). 

Figure~\ref{fig:Filtvedt-tide} shows the tidal currents at Filtvedt. The vertical stratification and current from observations from September 2009 at a station close to Kn2, was reported by \cite{staal:etal:2012}. It was found that the current in this location is influenced by internal tides generated at the Dr{\o}bak Sill propagating southward. The vertical structure of the displacement of the density surfaces had a maximum amplitude around 55 m depth, and there was also observed a phase shift in the current at the same depth. The observations at station Km1 from 2014 at the other side of the channel, have a phase shift in the current speed approximatly at the same depth. The phase shift of the simulated current at Km1 is found deeper in the water column, in the depth range 80-100 m. This is another indication that the stratification in the model do not capture the real stratification correctly.

\cite{hjelm:etal:2017} used (\ref{eq:ubar}) to estimate the barotropic current at station Km1, and t\_tide was used to find the tidal components. The results are shown in table~\ref{tab:UbarTide}. The model performance for the semi diurnal components (S$_2$, M$_2$ and N$_2$) of the barotropic tides are very good. The amplitude of diurnal components (K$_1$ and O$_1$) are to weak and the phase is wrong. The model performance for the components with periods around 6 hours are relativly good.
  
\begin{table}[ht]
%\vspace{-1.5cm}
	\caption{\small Tidal major amplitudes (cm/s) and phases (deg) for the barotropic current at Km1.}
	\label{tab:UbarTide}
	\centering
	\begin{tabular}{crrrrr} \hline
       		& \small Period     & \multicolumn{2}{c}{\small Observed}	& \multicolumn{2}{c}{\small Simulated}  \\
\small Comp.	& \small [h] $\;\;$ & \small [cm/s] 	& \small [deg]	& \small [cm/s] 	& \small [deg]   \\ \hline 
\small SS$_2$  & 12.0000    &  0.5 	& 280	&   0.5 & 334   \\
M$_2$  & 12.4206    &  1.8 	&   9	&   2.0 &  16   \\
N$_2$  & 12.6584    &  0.5 	& 293	&   0.5 & 343   \\
K$_1$  & 23.9345    &  0.5 	&  61	&   0.0 & 261   \\
O$_1$  & 25.8193    &  0.1 	& 120	&   0.2 & 250   \\
MN$_4$ &  6.2692    &  0.3 	& 166	&   0.2 & 305   \\
M$_4$  &  6.2103    &  0.7 	& 220	&   0.7 & 290   \\
MS$_4$ &  6.1033    &  0.0 	& 252	&   0.4 &  19   \\ \hline 
\end{tabular}
\end{table}
  
\begin{figure}[ht]
	\centerline{
		\includegraphics*[trim=0 0 0 0,clip=true,width=\textwidth]{Figurer/Filtvedt_obs_tide}}
	\centerline{
		\includegraphics*[trim=0 0 0 0,clip=true,width=\textwidth]{Figurer/Filtvedt_sim_tide}}
	\caption{\small Extracted tides from the observations (upper panel) and the model results (lower panel) at station Km1.
The colorbar indicates the current speed along the channel, on a color scale where red is flow into the fjord, and blue out of the fjord.
The black contourline shows where the mean flow is zero. The shallowest observations is at 16 m depth. This depth is indicated in the lower panel with a horizontal line. %(Denne figuren vil jeg gjerne plotte selv, Andre.)
}
	\label{fig:Filtvedt-tide}
\end{figure}


%=================== New version of "two cross sections"  ends (ANS) =======

%============================== Currents at Slagentangen =======================
\clearpage 
\subsubsection{Current at Slagentangen}
\label{subsubsec:slagen}
The observed currents at Slagentangen are compared with simulated data from 1st October 2014 until 30th November 2015 at approximately the same location and depth (Figure~\ref{fig:Slagen-kart}).

%\begin{figure}[ht]
%\centerline{
%\includegraphics*[trim=1cm 0cm 1cm 0cm,clip=true,width=0.5\textwidth]{Figurer/Slagen_kart}}
%\caption{\small
%Map retrived from the Norwegian Coastal Administration. The red dot marks the position corresponding to the extracted simulated data.}
%\label{fig:Slagen-kart}
%\end{figure}

\begin{figure}[ht]
	\centerline{
		\includegraphics*[trim=3cm 0cm 3cm 0cm,clip=true,width=\textwidth]{Figurer/Slagen_tid}}
	\caption{\small Timeseries of observed and simulated velocity magnitudes at Slagen.}
	\label{fig:Slagen-tid}
\end{figure}

\begin{figure}[ht]
	\centerline{
		\includegraphics*[trim=2cm 1cm 1cm 0cm,clip=true,height=4cm]{Figurer/Slagen_Rose_obs} 
		\includegraphics*[trim=2cm 1cm 3cm 0cm,clip=true,height=4cm]{Figurer/Slagen_Rose_sim} }
	\caption{\small Current roses for observed (left) and simulated (right) velocity magnitude at the two depths from 1st of October 2014 to 1st of October 2015.}
	\label{fig:Slagen-rose}
\end{figure}

\begin{figure}[t]
	\centerline{
		\includegraphics*[trim=2cm 0cm 2cm 0cm,clip=true,width=\textwidth]{Figurer/Slagen_pdf} }
	\caption{\small Probability density functions of velocities and directions at Slagen for 1st of October 2014 to 1st of October 2015. The bin width is 0.01 knots for velocity and 3 degrees for direction.}
	\label{fig:Slagen-pdf}
\end{figure}

\begin{table}[ht]
%\vspace{-1.5cm}
\caption{Yearly maximum observed velocity at Slagen.}
\label{tab:Slagen_max}
\centering
\begin{tabular}{|l|lll|lll|}
\hline 
& \multicolumn{3}{|l|}{\bf Max. velocity at 10m depth} & \multicolumn{3}{|l|}{\bf Max. velocity at 2.5m depth} \\
{\bf Year} & {\bf Date} & {\bf [m/s} & {\bf [deg]} & {\bf Date} & {\bf [m/s]} & {\bf [deg]} \\ \hline 
\small 2006 & 21 Jan 2006 & 0.42 & 139 & 31 Oct 2006 & 0.57 & 140 \\
\small 2007 & 14 Jan 2007 & 0.42 & 172 & 21 Aug 2007 & 1.03 & 359 \\
\small 2008 & 22 Mar 2008 & 0.36 & 149 & 19 Dec 2008 & 0.57 & 160 \\
\small 2009 & 17 Dec 2009 & 0.45 & 142 & 24 Mar 2009 & 0.56 & 139 \\
\small 2010 & 09 Nov 2010 & 0.41 & 138 & 09 Nov 2010 & 0.54 & 138 \\
\small 2011 & 01 Jan 2011 & 0.39 & 146 & 30 Mar 2011 & 0.62 & 185 \\
\small 2012 & 05 Dec 2012 & 0.39 & 138 & 29 May 2012 & 0.57 & 140 \\
\small 2013 & 10 Oct 2013 & 0.42 & 143 & 10 Oct 2013 & 0.49 & 144 \\
\small 2014 & 18 Apr 2014 & 0.44 & 147 & 26 Mar 2014 & 0.55 & 143 \\
\small 2015 & 24 Jan 2015 & 0.33 & 128 & 21 Mar 2015 & 0.55 & 141 \\
\hline
\end{tabular}
\end{table}

\begin{figure}[ht]
	\centerline{
		\includegraphics*[trim=0cm 0cm 0cm 0cm,clip=true,width=0.8\textwidth]{Figurer/Slagen_QQ} }
	\caption{\small Combined QQ- and scatter plot of observed and simulated current at Slagen from 1st of October 2014 to 1st of October 2015.}
	\label{fig:Slagen_QQ}
\end{figure}

Time series reveal that the observed velocities varies and follows no striking pattern (Figure~\ref{fig:Slagen-tid}). Current roses show that both the observed and the simulated velocities are stronger in the upper layer (Figure~\ref{fig:Slagen-rose}). The simulated velocities are stronger than the observed velocities. This is in accordance with the probability density functions (Figure~\ref{fig:Slagen-pdf}). The yearly maximum observed velocities are approximately 0.4 and 0.6 m/s at 10 and 2.5 meters depth respectively (Tab.~\ref{tab:Slagen_max}). During 2014 and 2015 maximum observed velocity at 2.5 meters depth was 0.55 m/s in southeast direction (143$^o$N) the 26th of March 2014. The velocity at 10 meters depth was 0.08 m/s (153$^o$N) at the time of maximum velocity at 2.5 meters depth indicating that the velocities are different in the two layers.

The mean directions are to the south east. At approximately 2.5 meters depth the mean directions are 146$^o$N and 139$^o$N for observed and simulated directions respectively which is in fairly good agreement. At approximately 10 meters depth the observed mean direction shifts to 170$^o$N while the simulated mean direction is 148$^o$N. Testing with popcorn indicate that the preferred direction of the surface currents are towards Bliksekilen located west of the Slagen Refinery. This is not the case neither in the observations nor the simulations. The probability density functions reveals that the model captures the distribution of directions in the upper layer, but does not capture the change in direction between the two depths (Figure~\ref{fig:Slagen-pdf}). The standard deviations at 2.5 and 10 meters are 55 and 66 degrees respectively for the observed directions, and 56 and 61 for the simulated directions.

The time series scatter plots reveal that the correlation in time is not satisfying (Figure~\ref{fig:Slagen-tid} and \ref{fig:Slagen_QQ}). The model seem to have difficulties with capturing the right phenomena influencing the currents to the right time. This is a well known problem when it comes to forecasting currents. The QQ-plots also confirms that the simulated currents are stronger than the observed currents. 

% % % % % % % % % % % % % % %
\clearpage
\subsection{CTD-measurements}
\label{subsec:CTDe}
For comparison we extracted temperature and salinity profiles from the model simulation at some of the CTD stations listed by Table ~\ref{tab:CTD_pos}, namely OF-1 in the outer part of the fjord, LA-1 in Larvikfjord and D-2 inside the Svelvik Sill in the inner Drammensfjord. Comparison between simulated and observed profiles are shown in Figs.~\ref{fig:CTD_OF-1}-~\ref{fig:CTD_D-3}. 

During the summer, the water in the upper layers are heated. The maximum surface temperature is observed towards the end of the summer. The profiles in the outer parts of the fjord indicate that the upper layer is too thin in the simulated data (Figure~\ref{fig:CTD_OF-1}). At larger depths, the water is too cold and the salinity is too high. This might indicate that the open boundary input and the representation of vertical mixing should be modified. 

Some of the observations are taken from smaller fjord branches, such as in the Larviksfjord close to the open boundary in the outer part of the Oslofjord. In such shallow waters, observations reveal that the whole water column is heated during summer and cooled during winter, but the simulated temperature varies only in the upper 20 meters (Figure~\ref{fig:CTD_OF-1}). 

The water masses below sill depth in the Drammensfjord are known to have very low vertical diffusivity leading to low oxygen conditions in the depth. Figure~\ref{fig:CTD_D-3} reveals that the vertical diffusivity in the same basin in the model is two high. In the model water with low salinity is mixed down all the way to the bottom. 

\begin{figure}[tbh]
	\centerline{
		\includegraphics*[trim=0cm 0cm 0cm 0cm,clip=true,width=\textwidth]{Figurer/CTD_OF-1} }
	\caption{\small Observed (solid) and simulated (dashed) salinity and temperature profiles at station OF-1 Torbj{\o}rnskj{\ae}r in the outer part of the Oslofjord. All profiles are from 2015.}
	\label{fig:CTD_OF-1}
\end{figure}

\begin{figure}[tbh]
	\centerline{
		\includegraphics*[trim=0cm 0cm 0cm 0cm,clip=true,width=\textwidth]{Figurer/CTD_LA-1} }
	\caption{\small Observed (solid) and simulated (dashed) salinity and temperature profiles at station LA-1 Larviksfjord in a fjord branch in the outer part of the Oslofjord. All profiles are from 2015.}
	\label{fig:CTD_LA-1}
\end{figure}

\begin{figure}[tbh]
	\centerline{
		\includegraphics*[trim=0cm 0cm 0cm 0cm,clip=true,width=\textwidth]{Figurer/CTD_D-3}}
	\caption{\small Observed (solid) and simulated (dashed) salinity and temperature profiles at station D-3 Solumstrand. All profiles are from 2015.}
	\label{fig:CTD_D-3}
\end{figure}

To get a better idea of how the properties of the water masses varies with time, contour plots of salinity and temperature are made as a function of depth and time. Three of the stations in the monitoring program are chosen to describe how the salinity and temperature in different parts of the fjord system evolves through two seasons. Station D-3 outside Solumstrand in the inner Drammensfjord, station OF-5 in Breiangen and station OF-1 near Torbj{\o}rnskj{\ae}r close to the open boundary of the model domain, are chosen.

Observed salinity and temperature at the three chosen stations are shown in Figs.~\ref{fig:Salt_YO_2014_2015} and ~\ref{fig:Temp_YO_2014_2015} respectively. If the salinity of the stations OF-5 and OF-1 are compared (middle and lower panel in Figure ~\ref{fig:Salt_YO_2014_2015}) it can be seen that the variations at for instance 40 m depth in Breiangen follow the variations further out in the fjord, but is approximately 1 psu fresher. This indicate a relatively good water exchange in the outer part of the fjord system. The water masses below sill depth in the Drammensjord is different. The salinity below 40 m depth is lower than at the same depth in Breiangen, and the salinity change very little with time. These are clear signs of a stagnant water mass, and is expected given the shallow sill depth of only 12 m at Svelvik.   

The seasonal temperature changes in the surface layer is slowly diffused down in the water masses in the outer part of the fjord system (middle and lower panel in Figure~\ref{fig:Temp_YO_2014_2015}). The temperature at 100 m depth has a seasonal change, but the maximum value is shifted in time, so the highest temperatures are found in the start of January at this depth. As seen in the upper panel in Figure~\ref{fig:Temp_YO_2014_2015} the temperature variations in the surface layer in the Drammensfjord is prevented from penetrating further down than about 40 m depth due to the low vertical diffusivity.

Figure~\ref{fig:Salt_Mod_2014_2015} shows the salinity at the three chosen stations extracted from the model. The water exchange in the model between the two stations OF-5 and OF-1 is relatively good, and the variations at OF-1 follow the variations further out in the fjord. This was the same we saw from the observations. The salinity however is to high at mid depth and this can as mentioned above be due to wrong open boundary input or vertical diffusivity. The upper panel in Figure~\ref{fig:Salt_Mod_2014_2015} shows that the deep water masses in the Drammensfjord have a high salinity at model initialisation, but fresh water from the surface is quickly mixed down.   

Figure~\ref{fig:Temp_Mod_2014_2015} shows the temperature at the three chosen stations extracted from the model. When the modelled temperature evolution in the outer part of the fjord system is compared with the observations, the vertical mixing seems to be too low, and the heating of the surface water during summer do not penetrate deep enough during the winter. While inside the Svelvik Sill the surface waters are mixed down too deep.


\begin{figure}[tbh]
	\centerline{
		\includegraphics*[trim=0cm 0cm 0cm 0cm,clip=true,width=\textwidth]{Figurer/Salt_YO_2014_2015} }
	\caption{\small Observed salinity at three stations in the Oslofjord. Contour lines mark 20, 30, and 34 psu. The white vertical lines indicate the positions when CTD casts were taken.}
	\label{fig:Salt_YO_2014_2015}
\end{figure}

\begin{figure}[tbh]
	\centerline{
		\includegraphics*[trim=0cm 0cm 0cm 0cm,clip=true,width=\textwidth]{Figurer/Temp_YO_2014_2015} }
\caption{\small Observed temperature at three stations in the Oslofjord. Contour lines mark 5 and 10 $^o$ C. The white vertical lines indicate the positions when CTD casts were taken. }
	\label{fig:Temp_YO_2014_2015}
\end{figure}

\begin{figure}[tbh]
	\centerline{
		\includegraphics*[trim=0cm 0cm 0cm 0cm,clip=true,width=\textwidth]{Figurer/Salt_Mod_2014_2015} }
	\caption{\small Modelled salinity at three stations in the Oslofjord. Contour lines mark 20, 30, and 34 psu.}
	\label{fig:Salt_Mod_2014_2015}
\end{figure}

\begin{figure}[tbh]
	\centerline{
		\includegraphics*[trim=0cm 0cm 0cm 0cm,clip=true,width=\textwidth]{Figurer/Temp_Mod_2014_2015}}
	\caption{\small Modelled temperature at three stations in the Oslofjord. Contour lines mark 5 and 10 $^{\textrm{o}}$ C.}
	\label{fig:Temp_Mod_2014_2015}
\end{figure}

% % % % % % % % % % % % % % %
\clearpage
\subsection{Temperature measurements}
\label{subsec:tempee}
\subsubsection{Water temperature near \AA sg\aa rdstrand}

The temperature observations from Scanmar AS are compared with simulated data extracted from 1.15 meters depth at approximately the same location as the observations.

Time series reveal that the simulated and observed temperature are in fairly good agreement (Figure~\ref{fig:temp_2015}). During winter and spring, but the model underestimate the temperature in the summer and fall with a few degrees  and \ref{fig:temp-QQ_scatter}). The model captures the timing of the daily variations in temperature, but seems to overestimate heating and cooling causing too large daily variations (Figure~\ref{fig:temp_jun2015}). 

2014 had a warmer summer than 2015. This is evident in both the observations and the simulations (Tab.~\ref{tab:temp}). The summer months of 2014 also had the largest variance in both the observed and the simulated temperature during 2014 and 2015. Generally, the simulated monthly temperature had a larger variance than the observed monthly temperature. The mean of the observed and simulated temperatures are 10.0$^o$C and 8.6$^o$C respectively, while the variances are 27.2$^o$C and 21.1$^o$C respectively.  

\begin{figure}[ht]
%\centerline{
%\includegraphics*[trim=0cm 0cm 0cm 0cm,clip=true,width=\textwidth]{Figurer/Temperatur_2014}}
\centerline{
\includegraphics*[trim=0cm 0cm 0cm 0cm,clip=true,width=\textwidth]{Figurer/Temperatur_2015}}
\caption{\small
Time series of observed and simulated temperature at \AA sg\aa dstrand. The difference is smoothed over 10 days.}
\label{fig:temp_2015}
\end{figure}

\begin{figure}[htb]
\centerline{
\includegraphics*[trim=1cm 0cm 1cm 0cm,clip=true,width=0.7\textwidth]{Figurer/Temperatur_QQ_scatter}}
\caption{\small
Combined QQ- and scatter plot of observed and simulated temperature at \AA sg\aa dstrand.}
\label{fig:temp-QQ_scatter}
\end{figure}

\begin{figure}[htb]
\centerline{
\includegraphics*[trim=0cm 0cm 0cm 0cm,clip=true,width=\textwidth]{Figurer/Temperatur_jun2015}}
\caption{\small
Time series of observed and simulated temperature at \AA sg\aa dstrand in June 2015.}
\label{fig:temp_jun2015}
\end{figure}

\newpage 

\begin{table}
\caption{Monthly statistics for observed and simulated temperature at \AA sg\aa rdstrand.}
\label{tab:temp}
\centering
\begin{tabular}{|ll|rrr|rrr|}
\hline 
&& \multicolumn{3}{|c|}{\bf 2014} & \multicolumn{3}{|c|}{\bf 2015} \\
&& {\bf quantity} & {\bf mean} & {\bf variance}  
& {\bf quantity} & {\bf mean} & {\bf variance} \\ \hline 
\small Jan & obs & 558 & 2.5 & 6.2 & 558 & 4.9 & 2.5 \\
\small     & sim & 745 & - & - & 745 & 4 & 0.7 \\
\small Feb & obs & 504 & 1.8 & 0.5 & 504 & 4 & 1.8 \\
\small     & sim & 673 & - & - & 673 & 3.6 & 0.6 \\
\small Mar & obs & 496 & 3.7 & 0.6 & 496 & 4.2 & 0.4 \\
\small     & sim & 745 & - & - & 745 & 4.7 & 0.3 \\
\small Apr & obs & 515 & 7.5 & 4.1 & 515 & 7 & 2.1 \\
\small     & sim & 721 & 7.6 & 7.1 & 721 & 7.4 & 2.8 \\
\small May & obs & 544 & 12.1 & 9.3 & 544 & 10.3 & 1.3 \\
\small     & sim & 745 & 11.3 & 10.4 & 745 & 9.9 & 3.5 \\
\small Jun & obs & 531 & 16.2 & 4.2 & 531 & 14 & 3.9 \\
\small     & sim & 721 & 15.5 & 8.3 & 721 & 12.8 & 6 \\
\small Jul & obs & 558 & 20.2 & 10.5 & 558 & 17.1 & 2 \\
\small     & sim & 745 & 17 & 24.3 & 745 & 14.6 & 5.1 \\
\small Aug & obs & 512 & 20 & 3.2 & 512 & 18.2 & 1.3 \\
\small     & sim & 745 & 16.4 & 5.6 & 745 & 15.5 & 8.7 \\
\small Sep & obs & 536 & 16.3 & 1.9 & 536 & 15.1 & 1.2 \\
\small     & sim & 721 & 12.9 & 8.9 & 721 & 12.7 & 2.5 \\
\small Oct & obs & 557 & 12.3 & 1.7 & 557 & 10.6 & 0.8 \\
\small     & sim & 745 & 9 & 1.5 & 745 & 8.6 & 2.7 \\
\small Nov & obs & 540 & 8.3 & 2.9 & 540 & 8.8 & 2.4 \\
\small     & sim & 721 & 6.3 & 2 & 721 & 6.1 & 1.3 \\
\small Dec & obs & 490 & 4.4 & 3.6 & 490 & 7.8 & 0.4 \\
\small     & sim & 733 & 3.4 & 1.5 & 733 & 4.8 & 0.8 \\
\hline
\end{tabular}
\end{table}

% % % % % % % % % % % % % % %
%\clearpage
\subsubsection{Water temperature in the Inner Oslofjord}
As revealed by Figures~\ref{fig:badetemp_2014} - \ref{fig:badetemp_2015} the observed and simulated temperature at the three beaches in the Inner Oslofjord are in relatively good agreement. 

Close to the shoreline and only 40 cm under the surface, the temperature is heavily influenced by the weather situation and local circulation patterns. 

The temperature differences during the day are larger in the model than in the observations. The observed temperature increases 1-3 degrees from 09:00 to 18:00 and is not measured during the night, while the modelled temperature increases up to six degrees from 06:00 to 23:00. The fact that temperature is not measured during the night, but only from 09:00 to 18:00, might explain differences i temperature rise during the day, but the difference might indicate too much heating in the model.

%Since the temperature is observed close to the shoreline, some near river outlets, and only 40 cm under the surface, the temperature ise heavily influenced by the weather situation and local circulation patterns. The model is not expected to capture such detailed effects. Still there are similarities between the modelled and the observed temperatures both in temperature level and in fluctuations. 

During the summer 2014 the model predicts higher temperatures at Sj\o strand than was observed. The observations in Hvalstrand have some of the same trends as the modelled temperature with temperatures up to 25 degrees. The air temperatures in 2014 was higher than in 2015 and resulted in higher water temperatures, especially in shallow areas. 

\begin{figure}[ht]
	\centerline{
		\includegraphics*[trim=0 0 0 0,clip=true,width=\textwidth]{Figurer/badetemp_2014}
		}
	\caption{\small The observed and modelled temperature at three beaches in the Inner Oslofjord during the summer 2014}
\label{fig:badetemp_2014}
\end{figure}

\begin{figure}[ht]
	\centerline{
		\includegraphics*[trim=0 0 0 0,clip=true,width=\textwidth]{Figurer/badetemp_2015}
		}
	\caption{\small The observed and modelled temperature at three beaches in the Inner Oslofjord during the summer 2015}
\label{fig:badetemp_2015}
\end{figure}


% % % % % % % % % % % % % % %
\clearpage 
\subsection{The Godafoss oil spill}
\label{sect:godafoss_model}
Unfortunately there is no overlap between the time period covered by the FjordOs hindcast and the time when the container ship Godafoss ran aground. So no direct comparison between simulated oil drift based on input from the FjordOs model is possible as part of this evaluation. Nevertheless, to investigate whether the FjordOs model provides results that are similar to the data gathered during the Godafoss oil spill (Section \ref{sect:godafoss_obs}), we have opted to map the pathway of a feigned oil spill. To this end we have released Lagrangian particles for a time period of one year (April 1st 2015 to April 1st 2016) at the location where the Godafoss grounded, and simulated particle trajectories using the open source trajectory-model OpenDrift\footnote{OpenDrift is distributed under a GPL v2.0 license, and is available on GitHub (https://github.com/knutfrode/opendrift). This is a trajectory model under development at MET Norway, and is described by its developers as "a software for modelling the trajectories and fate of objects or substances drifting in the ocean, or even in the atmosphere".}. We argue that over such a long period of time, there will be at least one situation similar to the weather and currents experienced during the Godafoss release.

\begin{figure}[hb]
  \begin{center}
    \begin{tabular}{cc}
      \includegraphics*[width=7.2cm]{Figurer/opendrift/opendrift_godafoss_shortest_time_crop}  & \includegraphics*[width=7.2cm]{Figurer/opendrift/opendrift_godafoss_shortest_time_zoom_crop}\\ 
    \end{tabular}
    \caption{\small Displayed is the time it takes a Lagrangian particle to reach inside a given 140 x 140 m area after it is released based on a one year simulation. The location of the release is marked by a black solid circle, and corresponds to the location where Godafoss ran aground. The color bar indicates the time in hours. Left-hand panel shows the time up to and including 47.5 hours, while the right-hand panel shows the time up to and including 12 hrs and for a smaller domain.}
    \label{fig:opendrift_godafoss_time}
  \end{center}
\end{figure}

The OpenDrift model was forced with currents from the FjordOs model and with winds from the Arome-MetCoOp 2.5km (Arome2.5) atmospheric model \citep{mulle:etal:2017}. The latter is the same atmospheric model we used as forcing when running the FjordOs hindcast. To properly treat the particles that are advected out of the FjordOs-model domain, and to enable them to re-enter at the correct location, we provided daily mean currents from NorKyst-800m outside of the FjordOs-model domain. 

\begin{figure}[htb]
  \begin{center}
    \begin{tabular}{cc}
      \includegraphics*[width=7.2cm]{Figurer/opendrift/opendrift_godafoss_consentration_crop}  & \includegraphics*[width=7.2cm]{Figurer/opendrift/opendrift_godafoss_consentration_zoom_crop}\\ 
    \end{tabular}
    \caption{\small As Figure~\ref{fig:opendrift_godafoss_time}, but showing the number of particles that has been inside a given 140x140 m area during the simulation. The colourbar indicates number of particles from 0 (blue) to 50 (red). Right-hand panel is a zoom in of the left-hand panel.}
    \label{fig:opendrift_godafoss_conc}
  \end{center}
\end{figure}

There are a number of parameters that may be tuned when running OpenDrift. One such parameter is the wind drift factor, which we set to $0.01$ (i.e. $1\%$). Otherwise we used the default parameter values. Particles are released once per hour throughout the one year simulation for a total of 8760 particles. The lifetime of each particle is set to 15 days, that is, after 15 days the particle is deactivated. This is done to reduce the computational cost of advecting a large number of particles. We further divided our area into a rectangular cells, and measured the number of hours from the particles were released until they reached the different cells. In addition we counted the number of particles that had been inside each cell during the simulation. The latter gives insight into the probability of experience oil in the individual cells. The size of the cells were chosen to be 140 x 140 meters, a balance between having too small or too large cells. If too small we might end up having too few particles in some cells, or if too large too many particles ends up in the same cell. The results are presented by Figures~\ref{fig:opendrift_godafoss_time} through \ref{fig:opendrift_godafoss_endpos}.

\begin{figure}[htb]
  \begin{center}
    \begin{tabular}{cc}
      \includegraphics*[width=7.2cm]{Figurer/opendrift/opendrift_godafoss_shortest_time_zoom_endpos_crop}  & \includegraphics*[width=7.2cm]{Figurer/opendrift/opendrift_godafoss_consentration_zoom_endpos_crop}\\ 
    \end{tabular}
    \caption{\small As Figures~\ref{fig:opendrift_godafoss_time} (left-hand panel) and \ref{fig:opendrift_godafoss_conc} (right-hand panel), but showing only the end position of each particle trajectory. The colourbar attached to the left-hand panel indicates the number of \emph{days} ranging from 0 (red) to 10 (blue) it takes a particle to reach a given cell of size 140 x 140 meter, while the colourbar attached to the right-hand panel indicates the number of particles ranging from 0 (blue) to 10 (red) that has been inside a given cell of size 140 x 140 meter.}
    \label{fig:opendrift_godafoss_endpos}
  \end{center}
\end{figure}

Comparing the results to the observed oil displayed by Figure~\ref{fig:godafoss_oil}, we observe that there are some obvious similarities between the observations and the simulations. For instance the right-hand panel of Figure~\ref{fig:opendrift_godafoss_endpos} shows, in similarity with the observations, that a substantial number of particles are transported westward from the release position and strand along the Vestfold coast, the islands around Tj{\o}me, and the F{\ae}rder lighthouse. We also note the very low number of particles that strand at the peninsula west of Stavern (the area far left in Figure~\ref{fig:opendrift_godafoss_endpos}). This corresponds well with where the oil was observed or not. Looking at the left-hand panel of Figure~\ref{fig:opendrift_godafoss_endpos} we would like to draw attention to two areas, namely Stavern and Bustein located, respectively, in the far lower left and the middle of Figure~\ref{fig:opendrift_godafoss_endpos}. As displayed by Figure~\ref{fig:opendrift_godafoss_endpos} the feigned oil strands at Bustein after approximately two days, while at Stavern (Korntin) the oil strands after about four to five days. These stranding times corresponds well with the timing shown by Figure~\ref{fig:godafoss_oil}.

% % % % % % % % % % % % % % %
\clearpage 
\subsection{Surface drifters}
\label{subsec:surfdr}
We evaluate the models ability to recreate the drifter trajectories by using a skill-score developed by \cite{liu:2011}. It is defined by 

\be
ss = 
  \begin{cases}
    1 - \frac{s}{n}  & ;\quad s \leq n\\
    1                & ;\quad s > n   \\ 
  \end{cases}, \quad \textrm{where} \quad s=\displaystyle\sum_{i=1}^{N} d_i / \displaystyle\sum_{i=1}^{N} l_{oi}.
\ee
Here $N$ is the total number of time steps, $d_i$ the distance between the modelled and observed endpoints of the Lagrangian trajectories at time step $i$ after the release, $l_{oi}$ the length of the observed trajectory at time step $i$, and $n$ a tolerance threshold. Thus if $ss=1$ it implies a perfect match while $ss=0$ means no skill. To investigate the skill of the FjordOs model we released Lagrangian particles into OpenDrift and computed the skill-score for each of the 15 drifters released as part of the FjordOs project (Section \ref{subsec:drifto}) using the FjordOs hindcast as input. Furthermore, to investigate whether the FjordOs model has a better skill than a coarser resolution model we also computed the skill-score using the NorKyst800 model results as input. In computing the skill-score we used a tolerance threshold $n=2$ to get positive values for most trajectories. 

\begin{table}[htb]
  \begin{center}
    \begin{tabular}{ | c | c | c | c | c |}
    \hline
		& \multicolumn{2}{|c|}{\bf \small{Full trajectory}}  & \multicolumn{2}{|c|}{\bf \small{First hour only}} \\ 
   {\bf \small Drop no.} & {\bf \small FjordOs} & {\bf \small NorKyst-800} & {\bf \small FjordOs} & {\bf \small NorKyst-800}\\ \hline
{\small	1}	& {\small 0.77 (11)}	& {\small 0.68 (11)}	& {\small 0.17}	& {\small 0.14} \\ 
{\small	4}	& {\small 0.78 (2)}	& {\small 0.53 (1)}	& {\small 0.55}	& {\small 0.45} \\
{\small	6}	& {\small 0.00 (3)}	& {\small 0.84 (11)}	& {\small 0.07}	& {\small 0.46} \\
{\small	8}	& {\small 0.82 (21)}	& {\small 0.58 (21)}	& {\small 0.00}	& {\small 0.00} \\
{\small	51}	& {\small 0.58 (3)}	& {\small 0.73 (15)}	& {\small 0.49}	& {\small 0.58} \\
{\small	52}	& {\small 0.38 (4)}	& {\small 0.50 (1)}	& {\small 0.37}	& {\small 0.47} \\
{\small	61}	& {\small 0.58 (3)}	& {\small 0.70 (8)}	& {\small 0.49}	& {\small 0.58} \\
{\small	91}	& {\small 0.51 (8)}	& {\small 0.50 (12)}	& {\small 0.73}	& {\small 0.67} \\
{\small	101}	& {\small 0.61 (15)}	& {\small 0.78 (15)}	& {\small 0.63}	& {\small 0.78} \\
{\small	102}	& {\small 0.59 (6)}	& {\small 0.75 (7)}	& {\small 0.87}	& {\small 0.77} \\ \hline
{\bf\small Avg.}& {\bf\small 0.56}	& {\bf \small 0.66}	& {\bf \small 0.43}	& {\bf \small 0.49}\\
    \hline
    \end{tabular}
    \caption{\small Skill-score of drifter trajectories released during the September 2015 cruise. The numbers in parenthesis in column two and three indicate how many hours we were able to follow each model trajectory. The last two columns reveal the skill-scores following each trajectory for the first hour only.}
    \label{tab:skillscore_full}
  \end{center}
\end{table}

Regarding the 13 drifters released in September 2015 we used both wind and currents as input. Note that the wind input used in OpenDrift is the same as the one we used to force both FjordOs and NorKyst800. The resulting skill-scores for 10 out of the 13 drifters are presented by Table~\ref{tab:skillscore_full}, while the trajectories are displayed by Figures \ref{fig:opendrift_trajectories1} - \ref{fig:opendrift_trajectories3}. The rationale for limiting the number of drifters to 10 is that some of them stranded too soon after the release to compute a meaningful skill-score. 

As revealed by Table~\ref{tab:skillscore_full} we observe that the \emph{average} NorKyst-800 skill-scores are higher compared to FjordOs. This is also true when we consider the \emph{average} skill-score of the first hour of the trajectories. Nevertheless we notice that FjordOs has a higher skill-score for some of the full trajectories, e.g., drop nos. 1, 2, and 8, and also for the 1 hour trajecories, e.g., 1, 2, 91, and 102. Furthermore by performing a visual comparison between the trajectories (Figures \ref{fig:opendrift_trajectories1} - \ref{fig:opendrift_trajectories3}) we are tempted to conclude that the trajectories based on the FjordOs model more often compares well to the observed trajectories than those based on NorKyst-800. We leave this up to the reader to decide. We merely state that the skill-scores alone appear to be insufficient to make a conclusive statement.

\begin{figure}[htb]
	\centerline{
		\includegraphics*[width=.5\textwidth]{Figurer/opendrift/skillscore/drop1i0}
		\includegraphics*[width=.5\textwidth]{Figurer/opendrift/skillscore/drop4i7}
		}
	\caption{\small Displayed are modelled and observed drifter trajectories September 2015. The colours indicate trajectories based on NorKyst-800 (green), FjordOs (blue), and observed (red). Dotted lines form a grid of equal distances with origo at the release point. Drop 1 is to the left (5 km grid), while drop 4 is to the right (0.5 km).}
	\label{fig:opendrift_trajectories1}
\end{figure}

\begin{comment}\begin{figure}[htb]
	\centerline{
		\includegraphics*[width=.5\textwidth]{Figurer/opendrift/skillscore/drop6i12}
		\includegraphics*[width=.5\textwidth]{Figurer/opendrift/skillscore/drop8i1}
		}
	\caption{\small As Figure~\ref{fig:opendrift_trajectories1}, but showing drop 6 (left, 2 km grid), and drop 8 (right, 5 km grid).}
	\label{fig:opendrift_trajectories1a}
\end{figure}\end{comment}

\begin{figure}[htb]
	\centerline{
		\includegraphics*[width=.5\textwidth]{Figurer/opendrift/skillscore/drop6i12}
		\includegraphics*[width=.5\textwidth]{Figurer/opendrift/skillscore/drop8i1}
		}
	\centerline{
		\includegraphics*[width=.5\textwidth]{Figurer/opendrift/skillscore/drop51i4}
		\includegraphics*[width=.5\textwidth]{Figurer/opendrift/skillscore/drop52i8}
		}
	\caption{\small As Figure \ref{fig:opendrift_trajectories1}, but showing drop 6 (top left, 2 km grid), and drop 8 (top right, 5 km grid), drop 51 (bottom left, 2 km grid) and drop 52 (bottom right, 1 km grid).}
	\label{fig:opendrift_trajectories2}
\end{figure}

\begin{figure}[htb]
	\centerline{
		\includegraphics*[width=.5\textwidth]{Figurer/opendrift/skillscore/drop61i5}
		\includegraphics*[width=.5\textwidth]{Figurer/opendrift/skillscore/drop91i11}
		}
	\caption{\small As Figure \ref{fig:opendrift_trajectories1}, but showing drop 61 (lef-hand panel, 1 km grid) and drop 91 (right-hand panel, 1 km grid).}
	\label{fig:opendrift_trajectories2a}
\end{figure}

\begin{figure}[htb]
	\centerline{
		\includegraphics*[width=.5\textwidth]{Figurer/opendrift/skillscore/drop101i6}
		\includegraphics*[width=.5\textwidth]{Figurer/opendrift/skillscore/drop102i9}
		}
	\caption{\small As Figure \ref{fig:opendrift_trajectories1}, but displaying drop 101 (left-hand panel, 5 km grid) and drop 102 (right-hand panel, 1 km grid).}
	\label{fig:opendrift_trajectories3}
\end{figure}

Regarding the two drifters released in September 2014 we only used current as input. The resulting skill-scores are presented by Tables~\ref{tab:skillscore2014_1h} and \ref{tab:skillscore2014_24h}, while Figures~\ref{fig:opendrift_trajectories2014} and \ref{fig:opendrift_trajectories2014a} displays the modelled and observed trajectories. As revealed the FjordOs model performs better than the NorKyst-800, but there is a clear weakness here since hourly data from NorKyst-800 is not available. Also, there are only two drifters. WE observe that the FjordOs model has the highest skill-score based on both hourly and daily average data, and was the only model to transport the drifters as far south as the observed drift. 

\begin{table}[htb]
\begin{center}
  \begin{tabular}{ | c | c | c |}
    \hline
    {\bf Drop no.} & {\bf FjordOs 1h} & {\bf NorKyst-800m 24h} \\ \hline
    1 & 0.79 (15) & 0.73 (15) \\ 
    2 & 0.84 (19) & 0.76 (19) \\ \hline
    {\bf Avg.} & {\bf 0.81} & {\bf 0.75} \\
    \hline
  \end{tabular}
\caption{\small As Table \ref{tab:skillscore_full}, but for the drifter release during the September 2014 cruise. Note that the FjordOs data are based on hourly resolution as input, while the NorKyst-800 data are based on daily averages.}
\label{tab:skillscore2014_1h}
\end{center}
\end{table}

\begin{table}[htb]
\begin{center}
  \begin{tabular}{ | c | c | c |}
    \hline
    {\bf Drop no.} & {\bf FjordOs 24h} & {\bf NorKyst-800m 24h} \\ \hline
    1 & 0.85 (15) & 0.73 (15) \\ 
    2 & 0.91 (19) & 0.76 (19) \\ \hline
    {\bf Avg.} & {\bf 0.88} & {\bf 0.75} \\
    \hline
  \end{tabular}
\caption{\small As Table \ref{tab:skillscore2014_1h}, but modelled trajectories from FjordOs are based on daily average ocean currents to get a more fair comparison between the models.}
\label{tab:skillscore2014_24h}
\end{center}
\end{table}

\begin{figure}[htb]
	\centerline{
		\includegraphics*[width=.5\textwidth]{Figurer/opendrift/skillscore/2014drop1i0}
		\includegraphics*[width=.5\textwidth]{Figurer/opendrift/skillscore/2014drop1i0_24h}
		}
	\caption{\small As Figure \ref{fig:opendrift_trajectories1}, but showing drop 1 of the September 2014 release.}
	\label{fig:opendrift_trajectories2014}
\end{figure}

\begin{figure}[htb]
	\centerline{
		\includegraphics*[width=.5\textwidth]{Figurer/opendrift/skillscore/2014drop2i1}
		\includegraphics*[width=.5\textwidth]{Figurer/opendrift/skillscore/2014drop2i1_24h}
		}
	\caption{\small As Figure \ref{fig:opendrift_trajectories2014}, but for drop 2 2014.}
	\label{fig:opendrift_trajectories2014a}
\end{figure}

We conclude that none of the models gives a perfect match, and that the currents in NorKyst-800 appears to be too weak during these drifter releases. Furthermore we think it is safe to conclude that the FjordOs model provides a better skill overall. It should be kept in mind though that, given the low number of drifters and their limited spatial and temporal distribution, more drifter studies should be performed in which the released drifters should have a better temporal and spatial distribution. In particular it would be interesting to release drifters in some of the narrow straits and sounds, and in the Archipelagoes.
