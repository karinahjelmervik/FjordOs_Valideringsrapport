%\newpage
%\clearpage
\section{Evaluation}
\label{sec:evalu}
% % % % % % % % % % % % %
\subsection{Water level and tide}
\label{subsec:wlevele}
Time series of water level from the three tide gauge stations (Section \ref{subsec:wlevelo}) were extracted for the period April 2014 through December 2015. To evaluate the model performance we also extracted time series of water level at locations near the three stations from the model simulation for the same time period. We emphasize that the tidal forcing we used at the southern open boundary of the FjordOs model includes eleven tidal constitutents only, and that the tidal forcing was adjusted by use of the observed tide at Viker close to the southern boundary \citep{roed:etal:2016}. 

\begin{figure}[htb]
  \begin{center}
    \begin{tabular}{c}
      \includegraphics*[trim=3cm 0cm 2.5cm 0cm,clip=true,width=15cm]{Figurer/Oscarsborg_Tide_selected_jan15} \\ 
      \includegraphics*[trim=3cm 0cm 2.5cm 0cm,clip=true,width=15cm]{Figurer/Oscarsborg_WL_rest_jan15} \\ 
    \end{tabular}
    \caption{\small Simulated (black) and observed (red) time series of the combined tidal water elevation (upper panel) and residual water level (lower panel) at Oscarsborg (cf. Figure~\ref{fig:kart_obs}) for the month of January 2015.}
    \label{fig:Waterlevel_jan15}
  \end{center}
\end{figure}

To compare model and observation we first analysed the two time series by use of t\_tide \citep{pavlo:etal:2002} to extract the amplitudes and phases of the individual harmonic tidal components. Next we superimposed the water elevation due to each of the eleven tidal constituents included in the tidal forcing to obtain what we term the combined water elevation. We finally subtracted the resulting combined water elevation from the total water elevation of each time series to derive comparabel ``residual'' water elevations. As revealed by Figure~\ref{fig:Waterlevel_jan15} the combined water elevation compares favourably with the observation. This is also true for the residual, but to a lesser degree. The latter is to be expected since the observations include all the tidal water elevations due to tides of longer and shorter periods than the eleven included in the tidal forcing. Note that to be able to discriminate between observations and simulated water levels the time series shown by Figure~\ref{fig:Waterlevel_jan15} is truncated to January 2015.

Of the eleven tidal components included in the tidal forcing M2 is, in terms of amplitude, the dominant constituent. It is therefore noteworthy, as revealed by Table~\ref{tab:Tide} on page \pageref{tab:Tide}, that the model represents this influential constituent to a very satisfactory degree both regarding amplitude and phase. This is true even at the stations Oscarsborg and Oslo that are both far from the models southern boundary. We also note that the constituents S2, N2 and O1 contribute, and that their contributions are in fairly good agreement with the observations as well. 

Table~\ref{tab:Tide} and Figure~\ref{fig:Waterlevel_tide} also show that even the longer period tidal constituents, e.g., SS and SSA, are to some degree picked up by the model despite the fact that they are not incorporated in the tidal forcing. This may be explained by the fact that we in addition to the tides also use daily mean water level, hydrography and currents extracted from the NorKyst800 model as forcing on the southern boundary \citep{roed:etal:2016}. In contrast, and as expected tides of periods shorter than 6 hours are not picked up by the model at all (Figure~\ref{fig:Waterlevel_tide}).

\begin{figure}[tbh] 
	\centerline{ \includegraphics*[trim=3cm 0cm 2.5cm 0cm,clip=true,width=\textwidth]{Figurer/Oscarsborg_Tide_not_included} } 
	\caption{\small Time series at Oscarsborg of the tidal components not included in the tidal forcing.} 
	\label{fig:Waterlevel_tide} 
\end{figure} 

\begin{table}[htb] 
	\caption{\small Simulated and observed tidal amplitudes and phases at three tide gauge stations for selected tidal components sorted by period.} 
	\label{tab:Tide} 
	\centering 
	\begin{tabular}{|c|c|l|cc|cc|cc|c|} 
\hline  
	  &	&	& \multicolumn{2}{|c|}{\small \bf Viker} & \multicolumn{2}{|c|}{\small \bf Oscarsborg} & \multicolumn{2}{|c|}{\small \bf Oslo} & {\small \bf Included} \\  
{\small \bf Comp.} & {\small \bf Period} & {\small \bf sim/} & {\small \bf amp.} & {\small \bf phase.} & {\small \bf amp.} & {\small \bf phase.} & {\small \bf amp.} & {\small \bf phase.} & {\small \bf in tidal} \\ 
	    & {\small \bf [h]} & {\small \bf obs} & {\small \bf [cm]} & {\small \bf [deg]} & {\small \bf [cm]} & {\small \bf [deg]} & {\small \bf [cm]}   & {\small \bf [deg]} & {\small \bf forcing} \\ \hline 
\small SA   & \small 8764    & \small sim & \small 15.5 & \small 284 & \small 15.6 & \small 286 & \small 15.4 & \small 286 & \small no  \\
\small      &        	     & \small obs & \small 10.0 & \small 319 & \small 11   & \small 322 & \small 11.4 & \small 324 &    	\\
\hline
\small SSA  & \small 4382    & \small sim & \small 8.8  & \small 197 & \small 9.2  & \small 200 & \small 9.4  & \small 200 & \small no  \\
\small      &		     & \small obs & \small 7.5  & \small 188 & \small 8.0  & \small 189 & \small 8.2  & \small 190 &    	\\
\hline
\small Q1   & \small 26.8684 & \small sim & \small 0.0  & \small 231 & \small 0.0  & \small 216 & \small 0.1  & \small 215 & \small no	\\
\small      &        	     & \small obs & \small 1.1  & \small 190 & \small 1.2  & \small 198 & \small 1.3  & \small 200 &    	\\
\hline
\small O1   & \small 25.8193 & \small sim & \small 3.5  & \small 337 & \small 3.8  & \small 339 & \small 3.8  & \small 339 & \small yes	\\
\small      &        	     & \small obs & \small 2.2  & \small 277 & \small 2.3  & \small 281 & \small 2.4  & \small 282 &    	\\
\hline
\small P1   & \small 24.0659 & \small sim & \small 0.6  & \small 322 & \small 0.6  & \small 334 & \small 0.7  & \small 342 & \small yes	\\
\small      &        	     & \small obs & \small 0.2  & \small 129 & \small 0.3  & \small 102 & \small 0.4  & \small 97  &    	\\
\hline
\small K1   & \small 23.9345 & \small sim & \small 0.2  & \small 187 & \small 0.1  & \small 175 & \small 0.2  & \small 157 & \small yes	\\
\small      &        	     & \small obs & \small 0.4  & \small 127 & \small 0.7  & \small 130 & \small 0.8  & \small 130 &    	\\
\hline
\small N2   & \small 12.6584 & \small sim & \small 3.0  & \small 69  & \small 3.5  & \small 75  & \small 3.7  & \small 76  & \small yes	\\
\small      &        	     & \small obs & \small 3.0  & \small 60  & \small 3.4  & \small 76  & \small 3.6  & \small 80  &    	\\
\hline
\small M2   & \small 12.4206 & \small sim & \small 11.5 & \small 105 & \small 13.2 & \small 112 & \small 13.9 & \small 114 & \small yes \\
\small      &        	     & \small obs & \small 11.9 & \small 105 & \small 13.8 & \small 121 & \small 14.4 & \small 125 &    	\\
\hline
\small S2   & \small 12.0000 & \small sim & \small 3.3  & \small 64  & \small 3.9  & \small 69  & \small 4.2  & \small 70  & \small yes \\
\small      &        	     & \small obs & \small 2.9  & \small 46  & \small 3.3  & \small 65  & \small 3.5  & \small 69  &    	\\
\hline
\small K2   & \small 11.9672 & \small sim & \small 1.6  & \small 10  & \small 2.0  & \small 13  & \small 2.1  & \small 15  & \small yes \\
\small      &        	     & \small obs & \small 0.7  & \small 45  & \small 0.8  & \small 66  & \small 0.9  & \small 66  &    	\\
\hline
\small MN4  & \small 6.2692  & \small sim & \small 0.2  & \small 5   & \small 0.5  & \small 32  & \small 0.6  & \small 35  & \small yes	\\
\small      &        	     & \small obs & \small 0.4  & \small 249 & \small 0.6  & \small 289 & \small 0.7  & \small 297 &    	\\
\hline
\small M4   & \small 6.2103  & \small sim & \small 1.0  & \small 355 & \small 1.9  & \small 18  & \small 2.5  & \small 23  & \small yes	\\
\small      &        	     & \small obs & \small 1.2  & \small 281 & \small 1.8  & \small 324 & \small 2.3  & \small 332 &    	\\
\hline
\small MS4  & \small 6.1033  & \small sim & \small 0.6  & \small 80  & \small 1.2  & \small 107 & \small 1.6  & \small 111 & \small yes	\\
\small      &        	     & \small obs & \small 0.3  & \small 360 & \small 0.5  & \small 44  & \small 0.7  & \small 56  &    	\\
\hline
\end{tabular}
\end{table}

Finally we note, as revealed by Table~\ref{tab:Tide}, that the observed M2 amplitude increases from south to north. This is reflected in the simulations as well as displayed by Figure~\ref{fig:M2field}. It is interesting to note that the lowest M2 amplitude is found in the Drammensfjord north of the threshold in Svelvik. In fact the M2 phase has a sudden increase at the thresholds of Svelvik and Dr{\o}bak. The same is true for the majority of the other relevant tidal components (not shown).

\begin{figure}[hb] 
	\centerline{ 
		\includegraphics*[trim=1cm 0cm 0cm 0cm,clip=true,width=0.49\textwidth]{Figurer/M2amp_felt}  
		\includegraphics*[trim=0.8cm 0cm 0cm 0cm,clip=true,width=0.49\textwidth]{Figurer/M2fase_felt} 
		} 
	\caption{\small Simulated fields of M2 amplitude (left-hand panel) and phase (right-hand panel). Corresponding observed values for M2 amplitude and phase are marked with circles at Viker, Oscarsborg, and Oslo.} 
	\label{fig:M2field} 
\end{figure} 



% % % % % % % % % % % %
\clearpage 
\subsection{Currents}
\label{subsec:curree}

%Current measurements were performed in two cross sections. Here the model results are evaluated using the measurements at the northern cross section, Brenntangen-Filtvedt. The results are similar at the southern cross section.

By analyzing the observed and simulated currents we notice that the strongest currents are found at the two northernmost moorings Filtvedt and Brenntangen at the entrance to the Dr{\o}bak Sound (Fig.~\ref{fig:kart_obs}). Here the fjord is relatively narrow, and hence the somewhat stronger currents may be explained by a stronger tidal signal due to the tapering of the width of the fjord at the entrance to the Dr{\o}bak Sound. We also note that the stations with the shallowest depth exhibits the highest current velocities. This may be explained by the fact that the horizontal pressure gradient has a tendency to decrease with depth.

In the remaining part we restrict the comparison regarding currents to three of the seven rigs equipped with an ADCP. These are the two northern moorings at Filtvedt and Brenntangen at the entrance to the Dr{\o}bak Sound, and the mooring Slagen near {\AA}sg\aa{}rdstrand. The rationale is that the remaining four moorings did not add any significant new insight other than corroborating the evaluation results gathered from comparing the simulated and observed currents at the former three stations.

%The observed and simulated currents at Filtvedt and Brenntangen are of the same magnitude in strength, but the flow pattern differs. Figure~\ref{fig:Filtvedt-cur} shows the observed and simulated currents at Filtvedt as an example. The observed currents at Brenntagen is similar in strength, but the peaks in current strength do not occur at the same time. The observations are dominated by noise in the upper 35-40 meters. Only depths larger than 40 meters are therefore included in the comparison.

Since the observed and simulated currents at Filtvedt and Brenntangen are of the same magnitude in strength, we only show the observed and simulated currents at Filtvedt (Fig.~\ref{fig:Filtvedt-cur}). We emphasize though that the flow pattern at the two moorings differs. Although the observed currents at Brenntagen is similar in strength, the peaks in current speed do not occur at the same time. Moreover, since the observations are dominated by noise in the upper 35-40 meters, we only include observed currents for depths larger than 40 meters in the comparison.

As revealed by Figure \ref{fig:Filtvedt-cur} the 3D current field is complex, and varies horizontally as well as vertically and in time. First we note that the simulated currents are undeniably smoother than the observed currents. This is to be expected since tidal constituents of higher frequencies are present in the observed currents (Section \ref{subsec:wlevele}). Given this we nevertheless observe that there is a lot of similarities between the two. For one both have a three layered structure with an intermediate layer in which the flow is southward (blue, green and black colors), a deep layer dominated by lower frequency tides, and an upper layer with a more variable structure, but mostly northward flows in the modelled currents. Secondly, we note that the depth of the intermediate layer is not constant in time but varies similarly in the observations and the simulations.  

According to the simulations (Fig.~\ref{fig:Filtvedt-simcur}) this three layer structure is not singular for Filtvedt, but is rather the rule in the area at the entrance to the Dr{\o}bak Sound. The figure also reveals that generally the currents in the upper layer are stronger than further down. Furthermore the figure disclose the horizontal variability in the flow pattern telling us that the currents at Filtvedt cannot be taken at representative for the whole area. For instance it should be noted that the tidal currents do not reverse at the same time across the fjord, which explains why the currents at Filtvedt is still southward while in the deeper areas the tides are already toward north at this particular instant. At 100 meters depth the currents at Filtvedt are weak and towards south even though the currents at 100 meters depth is generally stronger and towards north. At 40 meters depth the currents at Brenntangen is weaker than in the rest of the cross section. Note that the depth at Brenntangen is only 58 meters in the observations while in the depth is 46 meters at the corresponding point in the simulations. 

\begin{figure}[ht]
\centerline{
\includegraphics*[trim=0 0 0 0,clip=true,width=\textwidth]{Figurer/Filtvedt_obs_cur}}
\centerline{
\includegraphics*[trim=0 0 0 0,clip=true,width=\textwidth]{Figurer/Filtvedt_sim_cur}}
\caption{\small
Observed (upper panel) and simulated (lower panel) currents at Filtvedt. Since the observations near the surface are dominated by noise, only depths larger than 40 meters are included in the observations. In the lower panel $z$ = 40 meters is marked with a black line. The model depth at this locatios is 155 meters, which is close to the measured depth of 153 meters (Table~\ref{tab:Statnett})).}
\label{fig:Filtvedt-cur}
\end{figure}

\begin{figure}[ht]
\centerline{
\includegraphics*[trim=2cm 3cm 1cm 3.3cm,clip=true,height=5cm]{Figurer/Filtvedt_t4611_z2_current}
\includegraphics*[trim=3.8cm 3cm 6cm 3.3cm,clip=true,height=5cm]{Figurer/Filtvedt_t4611_z40_current}
\includegraphics*[trim=3.8cm 3cm 1cm 3.3cm,clip=true,height=5cm]{Figurer/Filtvedt_t4611_z100_current}}
\caption{\small Snapshots of the simulated currents at three different depths valid 10th of October 2014 at 12:00 UTC. The left-hand panel is at 2 m depth, the middle panel at 40 m depth, while the right-hand panel is at 100 m depth. Note that the range shown by the colorbar in the left-hand panel is 0-60 cm/s, while the colorbar  regarding the middle and right-hand panels has a range 0-20 cm/s.}
\label{fig:Filtvedt-simcur}
\end{figure}

The tides are evident in the observed and simulated currents at all observation points. Both simulated and observed time series of the currents at the seven observation points are analyzed using t\_tide \cite{pavlo:etal:2002} in order to extract the tidal components for each depth. The same period in time is applied for both simulations and observations (mid-September to the end of November 2014). Figure~\ref{fig:Filtvedt-tide} shows the tidal currents at Filtvedt. At Filtvedt the tidal impact in the observations occurs earlier at larger depths, but in the simulations the tidal impact occurs earlier at more shallow depths. This might be due to the flow pattern at the different depths and the fact that the flow pattern depends strongly on the bottom topography which is smoothed in the simulations. Note also that the calculations of the tides are based on only six weeks of data.


\begin{figure}[ht]
\centerline{
\includegraphics*[trim=0 0 0 0,clip=true,width=\textwidth]{Figurer/Filtvedt_obs_tide}}
\centerline{
\includegraphics*[trim=0 0 0 0,clip=true,width=\textwidth]{Figurer/Filtvedt_sim_tide}}
\caption{\small
Observed (upper) and simulated (lower) tidal currents at Filtvedt.}
\label{fig:Filtvedt-tide}
\end{figure}


\clearpage 
\subsubsection{Current at Slagentangen}
The observed currents at Slagentangen are compared with simulated data from 1st October 2014 until 30th November 2015 at approximately the same location and depth (Fig.~\ref{fig:Slagen-kart}).

%\begin{figure}[ht]
%\centerline{
%\includegraphics*[trim=1cm 0cm 1cm 0cm,clip=true,width=0.5\textwidth]{Figurer/Slagen_kart}}
%\caption{\small
%Map retrived from the Norwegian Coastal Administration. The red dot marks the position corresponding to the extracted simulated data.}
%\label{fig:Slagen-kart}
%\end{figure}

\begin{figure}[ht]
\centerline{
\includegraphics*[trim=3cm 0cm 3cm 0cm,clip=true,width=\textwidth]{Figurer/Slagen_tid}}
\caption{\small
Timeseries of observed and simulated velocity magnitudes at Slagen.}
\label{fig:Slagen-tid}
\end{figure}

\begin{figure}[ht]
\centerline{
\includegraphics*[trim=2cm 1cm 1cm 0cm,clip=true,height=4cm]{Figurer/Slagen_Rose_obs} 
\includegraphics*[trim=2cm 1cm 3cm 0cm,clip=true,height=4cm]{Figurer/Slagen_Rose_sim}
}
\caption{\small
Current roses for observed (left) and simulated (right) velocity magnitude at the two depths from 1st of October 2014 to 1st of October 2015.}
\label{fig:Slagen-rose}
\end{figure}

\begin{figure}[t]
\centerline{
\includegraphics*[trim=2cm 0cm 2cm 0cm,clip=true,width=\textwidth]{Figurer/Slagen_pdf} 
}
\caption{\small
Probability density functions of velocities and directions at Slagen for 1st of October 2014 to 1st of October 2015. The bin width is 0.01 knots for velocity and 3 degrees for direction.}
\label{fig:Slagen-pdf}
\end{figure}

\begin{table}[ht]
%\vspace{-1.5cm}
\caption{Yearly maximum observed velocity at Slagen.}
\label{tab:Slagen_max}
\centering
\begin{tabular}{|l|lll|lll|}
\hline 
& \multicolumn{3}{|l|}{\bf Max. velocity at 10m depth} & \multicolumn{3}{|l|}{\bf Max. velocity at 2.5m depth} \\
{\bf Year} & {\bf Date} & {\bf [m/s} & {\bf [deg]} & {\bf Date} & {\bf [m/s]} & {\bf [deg]} \\ \hline 
\small 2006 & 21 Jan 2006 & 0.42 & 139 & 31 Oct 2006 & 0.57 & 140 \\
\small 2007 & 14 Jan 2007 & 0.42 & 172 & 21 Aug 2007 & 1.03 & 359 \\
\small 2008 & 22 Mar 2008 & 0.36 & 149 & 19 Dec 2008 & 0.57 & 160 \\
\small 2009 & 17 Dec 2009 & 0.45 & 142 & 24 Mar 2009 & 0.56 & 139 \\
\small 2010 & 09 Nov 2010 & 0.41 & 138 & 09 Nov 2010 & 0.54 & 138 \\
\small 2011 & 01 Jan 2011 & 0.39 & 146 & 30 Mar 2011 & 0.62 & 185 \\
\small 2012 & 05 Dec 2012 & 0.39 & 138 & 29 May 2012 & 0.57 & 140 \\
\small 2013 & 10 Oct 2013 & 0.42 & 143 & 10 Oct 2013 & 0.49 & 144 \\
\small 2014 & 18 Apr 2014 & 0.44 & 147 & 26 Mar 2014 & 0.55 & 143 \\
\small 2015 & 24 Jan 2015 & 0.33 & 128 & 21 Mar 2015 & 0.55 & 141 \\
\hline
\end{tabular}
\end{table}

\begin{figure}[ht]
\centerline{
\includegraphics*[trim=0cm 0cm 0cm 0cm,clip=true,width=0.8\textwidth]{Figurer/Slagen_QQ}}
\caption{\small
Combined QQ- and scatter plot of observed and simulated current at Slagen from 1st of October 2014 to 1st of October 2015.}
\label{fig:Slagen_QQ}
\end{figure}

Time series reveal that the observed velocities varies and follows no striking pattern (Fig.~\ref{fig:Slagen-tid}).
Current roses show that both the observed and the simulated velocities are stronger in the upper layer (Fig.~\ref{fig:Slagen-rose}). The simulated velocities are stronger than the observed velocities. This is in accordance with the probability density functions (Fig.~\ref{fig:Slagen-pdf}). The yearly maximum observed velocities are approximately 0.4 and 0.6 m/s at 10 and 2.5 meters depth respectively (Tab.~\ref{tab:Slagen_max}). During 2014 and 2015 maximum observed velocity at 2.5 meters depth was 0.55 m/s in southeast direction (143$^o$N) the 26th of March 2014. The velocity at 10 meters depth was 0.08 m/s (153$^o$N) at the time of maximum velocity at 2.5 meters depth indicating that the velocities are different in the two layers.

The mean directions are to the south east. At approximately 2.5 meters depth the mean directions are 146$^o$N and 139$^o$N for observed and simulated directions respectively which is in fairly good agreement. At approximately 10 meters depth the observed mean direction shifts to 170$^o$N while the simulated mean direction is 148$^o$N. 
Testing with popcorn indicate that the preferred direction of the surface currents are towards Bliksekilen located west of the Slagen Refinery. This is not the case neither in the observations nor the simulations.
The probability density functions reveals that the model captures the distribution of directions in the upper layer, but does not capture the change in direction between the two depths (Fig.~\ref{fig:Slagen-pdf}). The standard deviations at 2.5 and 10 meters are 55 and 66 degrees respectively for the observed directions, and 56 and 61 for the simulated directions.

The time series scatter plots reveal that the correlation in time is not satisfying (Fig.~\ref{fig:Slagen-tid} and \ref{fig:Slagen_QQ}). The model seem to have difficulties with capturing the right phenomena influencing the currents to the right time. This is a well known problem when it comes to forecasting currents. The QQ-plots also confirms that the simulated currents are stronger than the observed currents. 

% % % % % % % % % % % % % % %
\clearpage
\subsection{CTD-measurements}
\label{subsec:CTDe}
For comparison we extracted temperature and salinity profiles from the model simulation at some of the CTD stations listed by Table ~\ref{tab:CTD_pos}, namely OF-1 in the outer part of the fjord, LA-1 in Larvikfjord and D-2 inside the Svelvik Sill in the inner Drammensfjord. Comparison between simulated and observed profiles are shown in Figs.~\ref{fig:CTD_OF-1}-~\ref{fig:CTD_D-3}. 

During the summer, the water in the upper layers are heated. 
The maximum surface temperature is observed towards the end of the summer. 
The profiles in the outer parts of the fjord indicate that the upper layer is too thin in the simulated data (Fig.~\ref{fig:CTD_OF-1}). 
At larger depths, the water is too cold and the salinity is too high. 
This might indicate that the open boundary input and the representation of vertical mixing should be modified. 

Some of the observations are taken from smaller fjord branches, such as in the Larviksfjord close to the open boundary in the outer part of the Oslofjord. 
In such shallow waters, observations reveal that the whole water column is heated during summer and cooled during winter, but the simulated temperature varies only in the upper 20 meters (Fig.~\ref{fig:CTD_OF-1}). 

The water masses below sill depth in the Drammensfjord are known to have very low vertical diffusivity leading to low oxygen conditions in the depth.
Fig.~\ref{fig:CTD_D-3} reveals that the vertical diffusivity in the same basin in the model is two high.
In the model water with low salinity is mixed down all the way to the bottom. 

\begin{figure}[tbh]
\centerline{
\includegraphics*[trim=0cm 0cm 0cm 0cm,clip=true,width=\textwidth]{Figurer/CTD_OF-1}}
\caption{\small
Observed (solid) and simulated (dashed) salinity and temperature profiles at station OF-1 Torbj{\o}rnskj{\ae}r in the outer part of the Oslofjord. All profiles are from 2015.}
\label{fig:CTD_OF-1}
\end{figure}

\begin{figure}[tbh]
\centerline{
\includegraphics*[trim=0cm 0cm 0cm 0cm,clip=true,width=\textwidth]{Figurer/CTD_LA-1}}
\caption{\small
Observed (solid) and simulated (dashed) salinity and temperature profiles at station LA-1 Larviksfjord in a fjord branch in the outer part of the Oslofjord. All profiles are from 2015.}
\label{fig:CTD_LA-1}
\end{figure}

\begin{figure}[tbh]
\centerline{
\includegraphics*[trim=0cm 0cm 0cm 0cm,clip=true,width=\textwidth]{Figurer/CTD_D-3}}
\caption{\small
Observed (solid) and simulated (dashed) salinity and temperature profiles at station D-3 Solumstrand. All profiles are from 2015.}
\label{fig:CTD_D-3}
\end{figure}

To get a better idea of how the properties of the water masses varies with time, contour plots of salinity and temperature are made as a function of depth and time.
Three of the stations in the monitoring program are chosen to describe how the salinity and temperature in different parts of the fjord system evolves through two seasons. Station D-3 outside Solumstrand in the inner Drammensfjord, station OF-5 in Breiangen and station OF-1 near Torbj{\o}rnskj{\ae}r close to the open boundary of the model domain, are chosen.

Observed salinity and temperature at the three chosen stations are shown in 
Figs.~\ref{fig:Salt_YO_2014_2015} and ~\ref{fig:Temp_YO_2014_2015} respectively. 
If the salinity of the stations OF-5 and OF-1 are compared (middle and lower panel in Fig. ~\ref{fig:Salt_YO_2014_2015}) it can be seen that the variations at for instance 40 m depth in Breiangen follow the variations further out in the fjord, but is approximately 1 psu fresher.
This indicate a relatively good water exchange in the outer part of the fjord system.
The water masses below sill depth in the Drammensjord is different. 
The salinity below 40 m depth is lower than at the same depth in Breiangen, and the salinity change very little with time.
These are clear signs of a stagnant water mass, and is expected given the shallow sill 
depth of only 12 m at Svelvik.   

The seasonal temperature changes in the surface layer is slowly diffused down in the water masses in the outer part of the fjord system (middle and lower panel in Fig.~\ref{fig:Temp_YO_2014_2015}).
The temperature at 100 m depth has a seasonal change, but the maximum value is shifted in time, so the highest temperatures are found in the start of January at this depth.
As seen in the upper panel in Fig.~\ref{fig:Temp_YO_2014_2015} the temperature variations 
in the surface layer in the Drammensfjord is prevented from penetrating further down than
about 40 m depth due to the low vertical diffusivity.

Fig.~\ref{fig:Salt_Mod_2014_2015} shows the salinity at the three chosen stations extracted from the model.
The water exchange in the model between the two stations OF-5 and OF-1 is relatively good, and the variations at OF-1 follow the variations further out in the fjord.
This was the same we saw from the observations.
The salinity however is to high at mid depth and this can as mentioned above be due to
wrong open boundary input or vertical diffusivity.
The upper panel in Fig.~\ref{fig:Salt_Mod_2014_2015} shows that the deep water masses in the Drammensfjord have a high salinity at model initialisation, but fresh water from the surface is quickly mixed down.   

Fig.~\ref{fig:Temp_Mod_2014_2015} shows the temperature at the three chosen stations extracted from the model.
When the modelled temperature evolution in the outer part of the fjord system is compared with the observations, the vertical mixing seems to be too low, and the heating of the surface water during summer do not penetrate deep enough during the winter.
While inside the Svelvik Sill the surface waters are mixed down too deep.


\begin{figure}[tbh]
\centerline{
\includegraphics*[trim=0cm 0cm 0cm 0cm,clip=true,width=\textwidth]{Figurer/Salt_YO_2014_2015}}
\caption{\small
Observed salinity at three stations in the Oslofjord. Contour lines mark 20, 30, and 34 psu. The white vertical lines indicate the positions when CTD casts were taken. 
}
\label{fig:Salt_YO_2014_2015}
\end{figure}

\begin{figure}[tbh]
\centerline{
\includegraphics*[trim=0cm 0cm 0cm 0cm,clip=true,width=\textwidth]{Figurer/Temp_YO_2014_2015}}
\caption{\small
Observed temperature at three stations in the Oslofjord. Contour lines mark 5 and 10 $^o$ C. The white vertical lines indicate the positions when CTD casts were taken. 
}
\label{fig:Temp_YO_2014_2015}
\end{figure}

\begin{figure}[tbh]
\centerline{
\includegraphics*[trim=0cm 0cm 0cm 0cm,clip=true,width=\textwidth]{Figurer/Salt_Mod_2014_2015}}
\caption{\small
Modelled salinity at three stations in the Oslofjord. Contour lines mark 20, 30, and 34 psu.
}
\label{fig:Salt_Mod_2014_2015}
\end{figure}

\begin{figure}[tbh]
\centerline{
\includegraphics*[trim=0cm 0cm 0cm 0cm,clip=true,width=\textwidth]{Figurer/Temp_Mod_2014_2015}}
\caption{\small
Modelled temperature at three stations in the Oslofjord. Contour lines mark 5 and 10 $^o$ C.
}
\label{fig:Temp_Mod_2014_2015}
\end{figure}

% % % % % % % % % % % % % % %
\clearpage
\subsection{Temperature measurements}
\label{subsec:tempee}
\subsubsection{Water temperature near \AA sg\aa rdstrand}

The temperature observations from Scanmar AS are compared with simulated data extracted from 1.15 meters depth at approximately the same location as the observations.

Time series reveal that the simulated and observed temperature are in fairly good agreement (Fig.~\ref{fig:temp_2015}). During winter and spring, but the model underestimate the temperature in the summer and fall with a few degrees  and \ref{fig:temp-QQ_scatter}). The model captures the timing of the daily variations in temperature, but seems to overestimate heating and cooling causing too large daily variations (Fig.~\ref{fig:temp_jun2015}). 

2014 had a warmer summer than 2015. This is evident in both the observations and the simulations (Tab.~\ref{tab:temp}). The summer months of 2014 also had the largest variance in both the observed and the simulated temperature during 2014 and 2015. Generally, the simulated monthly temperature had a larger variance than the observed monthly temperature. The mean of the observed and simulated temperatures are 10.0$^o$C and 8.6$^o$C respectively, while the variances are 27.2$^o$C and 21.1$^o$C respectively.  

\begin{figure}[ht]
%\centerline{
%\includegraphics*[trim=0cm 0cm 0cm 0cm,clip=true,width=\textwidth]{Figurer/Temperatur_2014}}
\centerline{
\includegraphics*[trim=0cm 0cm 0cm 0cm,clip=true,width=\textwidth]{Figurer/Temperatur_2015}}
\caption{\small
Time series of observed and simulated temperature at \AA sg\aa dstrand. The difference is smoothed over 10 days.}
\label{fig:temp_2015}
\end{figure}

\begin{figure}[htb]
\centerline{
\includegraphics*[trim=1cm 0cm 1cm 0cm,clip=true,width=0.7\textwidth]{Figurer/Temperatur_QQ_scatter}}
\caption{\small
Combined QQ- and scatter plot of observed and simulated temperature at \AA sg\aa dstrand.}
\label{fig:temp-QQ_scatter}
\end{figure}

\begin{figure}[htb]
\centerline{
\includegraphics*[trim=0cm 0cm 0cm 0cm,clip=true,width=\textwidth]{Figurer/Temperatur_jun2015}}
\caption{\small
Time series of observed and simulated temperature at \AA sg\aa dstrand in June 2015.}
\label{fig:temp_jun2015}
\end{figure}

\newpage 

\begin{table}
\caption{Monthly statistics for observed and simulated temperature at \AA sg\aa rdstrand.}
\label{tab:temp}
\centering
\begin{tabular}{|ll|rrr|rrr|}
\hline 
&& \multicolumn{3}{|c|}{\bf 2014} & \multicolumn{3}{|c|}{\bf 2015} \\
&& {\bf quantity} & {\bf mean} & {\bf variance}  
& {\bf quantity} & {\bf mean} & {\bf variance} \\ \hline 
\small Jan & obs & 558 & 2.5 & 6.2 & 558 & 4.9 & 2.5 \\
\small     & sim & 745 & - & - & 745 & 4 & 0.7 \\
\small Feb & obs & 504 & 1.8 & 0.5 & 504 & 4 & 1.8 \\
\small     & sim & 673 & - & - & 673 & 3.6 & 0.6 \\
\small Mar & obs & 496 & 3.7 & 0.6 & 496 & 4.2 & 0.4 \\
\small     & sim & 745 & - & - & 745 & 4.7 & 0.3 \\
\small Apr & obs & 515 & 7.5 & 4.1 & 515 & 7 & 2.1 \\
\small     & sim & 721 & 7.6 & 7.1 & 721 & 7.4 & 2.8 \\
\small May & obs & 544 & 12.1 & 9.3 & 544 & 10.3 & 1.3 \\
\small     & sim & 745 & 11.3 & 10.4 & 745 & 9.9 & 3.5 \\
\small Jun & obs & 531 & 16.2 & 4.2 & 531 & 14 & 3.9 \\
\small     & sim & 721 & 15.5 & 8.3 & 721 & 12.8 & 6 \\
\small Jul & obs & 558 & 20.2 & 10.5 & 558 & 17.1 & 2 \\
\small     & sim & 745 & 17 & 24.3 & 745 & 14.6 & 5.1 \\
\small Aug & obs & 512 & 20 & 3.2 & 512 & 18.2 & 1.3 \\
\small     & sim & 745 & 16.4 & 5.6 & 745 & 15.5 & 8.7 \\
\small Sep & obs & 536 & 16.3 & 1.9 & 536 & 15.1 & 1.2 \\
\small     & sim & 721 & 12.9 & 8.9 & 721 & 12.7 & 2.5 \\
\small Oct & obs & 557 & 12.3 & 1.7 & 557 & 10.6 & 0.8 \\
\small     & sim & 745 & 9 & 1.5 & 745 & 8.6 & 2.7 \\
\small Nov & obs & 540 & 8.3 & 2.9 & 540 & 8.8 & 2.4 \\
\small     & sim & 721 & 6.3 & 2 & 721 & 6.1 & 1.3 \\
\small Dec & obs & 490 & 4.4 & 3.6 & 490 & 7.8 & 0.4 \\
\small     & sim & 733 & 3.4 & 1.5 & 733 & 4.8 & 0.8 \\
\hline
\end{tabular}
\end{table}

% % % % % % % % % % % % % % %
%\clearpage
\subsubsection{Water temperature in the Inner Oslofjord}
As revealed by Figures~\ref{fig:badetemp_2014} - \ref{fig:badetemp_2015} the observed and simulated temperature at the three beaches in the Inner Oslofjord are in relatively good agreement. 

Close to the shoreline and only 40 cm under the surface, the temperature is heavily influenced by the weather situation and local circulation patterns. 

The temperature differences during the day are larger in the model than in the observations. The observed temperature increases 1-3 degrees from 09:00 to 18:00 and is not measured during the night, while the modelled temperature increases up to six degrees from 06:00 to 23:00. The fact that temperature is not measured during the night, but only from 09:00 to 18:00, might explain differences i temperature rise during the day, but the difference might indicate too much heating in the model.

%Since the temperature is observed close to the shoreline, some near river outlets, and only 40 cm under the surface, the temperature ise heavily influenced by the weather situation and local circulation patterns. The model is not expected to capture such detailed effects. Still there are similarities between the modelled and the observed temperatures both in temperature level and in fluctuations. 

During the summer 2014 the model predicts higher temperatures at Sj\o strand than was observed. The observations in Hvalstrand have some of the same trends as the modelled temperature with temperatures up to 25 degrees. The air temperatures in 2014 was higher than in 2015 and resulted in higher water temperatures, especially in shallow areas. 

\begin{figure}[ht]
	\centerline{
		\includegraphics*[trim=0 0 0 0,clip=true,width=\textwidth]{Figurer/badetemp_2014}
		}
	\caption{\small The observed and modelled temperature at three beaches in the Inner Oslofjord during the summer 2014}
\label{fig:badetemp_2014}
\end{figure}

\begin{figure}[ht]
	\centerline{
		\includegraphics*[trim=0 0 0 0,clip=true,width=\textwidth]{Figurer/badetemp_2015}
		}
	\caption{\small The observed and modelled temperature at three beaches in the Inner Oslofjord during the summer 2015}
\label{fig:badetemp_2015}
\end{figure}


% % % % % % % % % % % % % % %
\clearpage 
\subsection{The Godafoss oil spill}
\label{sect:godafoss_model}
Unfortunately there is no overlap between the time period covered by the FjordOs hindcast and the time when the container ship Godafoss ran aground. So no direct comparison between simulated oil drift based on input from the FjordOs model is possible as part of this evaluation. Nevertheless, to investigate whether the FjordOs model provides results that are similar to the data gathered during the Godafoss oil spill (Section \ref{sect:godafoss_obs}), we have opted to map the pathway of a feigned oil spill. To this end we have released Lagrangian particles for a time period of one year (April 1st 2015 to April 1st 2016) at the location where the Godafoss grounded, and simulated particle trajectories using the open source trajectory-model OpenDrift\footnote{OpenDrift is distributed under a GPL v2.0 license, and is available on GitHub (https://github.com/knutfrode/opendrift). This is a trajectory model under development at MET Norway, and is described by its developers as "a software for modelling the trajectories and fate of objects or substances drifting in the ocean, or even in the atmosphere".}. We argue that over such a long period of time, there will be at least one situation similar to the weather and currents experienced during the Godafoss release.

\begin{figure}[hb]
  \begin{center}
    \begin{tabular}{cc}
      \includegraphics*[width=7.2cm]{Figurer/opendrift/opendrift_godafoss_shortest_time_crop}  & \includegraphics*[width=7.2cm]{Figurer/opendrift/opendrift_godafoss_shortest_time_zoom_crop}\\ 
    \end{tabular}
    \caption{\small Displayed is the time it takes a Lagrangian particle to reach inside a given 140 x 140 m area after it is released based on a one year simulation. The location of the release is marked by a black solid circle, and corresponds to the location where Godafoss ran aground. The color bar indicates the time in hours. Left-hand panel shows the time up to and including 47.5 hours, while the right-hand panel shows the time up to and including 12 hrs and for a smaller domain.}
    \label{fig:opendrift_godafoss_time}
  \end{center}
\end{figure}

The OpenDrift model was forced with currents from the FjordOs model and with winds from the Arome-MetCoOp 2.5km (Arome2.5) atmospheric model \citep{mulle:etal:2017}. The latter is the same atmospheric model we used as forcing when running the FjordOs hindcast. To properly treat the particles that are advected out of the FjordOs-model domain, and to enable them to re-enter at the correct location, we provided daily mean currents from NorKyst-800m outside of the FjordOs-model domain. 

\begin{figure}[htb]
  \begin{center}
    \begin{tabular}{cc}
      \includegraphics*[width=7.2cm]{Figurer/opendrift/opendrift_godafoss_consentration_crop}  & \includegraphics*[width=7.2cm]{Figurer/opendrift/opendrift_godafoss_consentration_zoom_crop}\\ 
    \end{tabular}
    \caption{\small As Figure~\ref{fig:opendrift_godafoss_time}, but showing the number of particles that has been inside a given 140x140 m area during the simulation. The colourbar indicates number of particles from 0 (blue) to 50 (red). Right-hand panel is a zoom in of the left-hand panel.}
    \label{fig:opendrift_godafoss_conc}
  \end{center}
\end{figure}

There are a number of parameters that may be tuned when running OpenDrift. One such parameter is the wind drift factor, which we set to $0.01$ (i.e. $1\%$). Otherwise we used the default parameter values. Particles are released once per hour throughout the one year simulation for a total of 8760 particles. The lifetime of each particle is set to 15 days, that is, after 15 days the particle is deactivated. This is done to reduce the computational cost of advecting a large number of particles. We further divided our area into a rectangular cells, and measured the number of hours from the particles were released until they reached the different cells. In addition we counted the number of particles that had been inside each cell during the simulation. The latter gives insight into the probability of experience oil in the individual cells. The size of the cells were chosen to be 140 x 140 meters, a balance between having too small or too large cells. If too small we might end up having too few particles in some cells, or if too large too many particles ends up in the same cell. The results are presented by Figures~\ref{fig:opendrift_godafoss_time} through \ref{fig:opendrift_godafoss_endpos}.

\begin{figure}[htb]
  \begin{center}
    \begin{tabular}{cc}
      \includegraphics*[width=7.2cm]{Figurer/opendrift/opendrift_godafoss_shortest_time_zoom_endpos_crop}  & \includegraphics*[width=7.2cm]{Figurer/opendrift/opendrift_godafoss_consentration_zoom_endpos_crop}\\ 
    \end{tabular}
    \caption{\small As Figures~\ref{fig:opendrift_godafoss_time} (left-hand panel) and \ref{fig:opendrift_godafoss_conc} (right-hand panel), but showing only the end position of each particle trajectory. The colourbar attached to the left-hand panel indicates the number of \emph{days} ranging from 0 (red) to 10 (blue) it takes a particle to reach a given cell of size 140 x 140 meter, while the colourbar attached to the right-hand panel indicates the number of particles ranging from 0 (blue) to 10 (red) that has been inside a given cell of size 140 x 140 meter.}
    \label{fig:opendrift_godafoss_endpos}
  \end{center}
\end{figure}

Comparing the results to the observed oil displayed by Figure~\ref{fig:godafoss_oil}, we observe that there are some obvious similarities between the observations and the simulations. For instance the right-hand panel of Figure~\ref{fig:opendrift_godafoss_endpos} shows, in similarity with the observations, that a substantial number of particles are transported westward from the release position and strand along the Vestfold coast, the islands around Tj{\o}me, and the F{\ae}rder lighthouse. We also note the very low number of particles that strand at the peninsula west of Stavern (the area far left in Figure~\ref{fig:opendrift_godafoss_endpos}). This corresponds well with where the oil was observed or not. Looking at the left-hand panel of Figure~\ref{fig:opendrift_godafoss_endpos} we would like to draw attention to two areas, namely Stavern and Bustein located, respectively, in the far lower left and the middle of Figure~\ref{fig:opendrift_godafoss_endpos}. As displayed by Figure~\ref{fig:opendrift_godafoss_endpos} the feigned oil strands at Bustein after approximately two days, while at Stavern (Korntin) the oil strands after about four to five days. These stranding times corresponds well with the timing shown by Figure~\ref{fig:godafoss_oil}.

% % % % % % % % % % % % % % %
\clearpage 
\subsection{Surface drifters}
\label{subsec:surfdr}
We evaluate the models ability to recreate the drifter trajectories by using a skill-score developed by \cite{liu:2011}. It is defined by 

\be
ss = 
  \begin{cases}
    1 - \frac{s}{n}  & ;\quad s \leq n\\
    1                & ;\quad s > n   \\ 
  \end{cases}, \quad \textrm{where} \quad s=\displaystyle\sum_{i=1}^{N} d_i / \displaystyle\sum_{i=1}^{N} l_{oi}.
\ee
Here $N$ is the total number of time steps, $d_i$ the distance between the modelled and observed endpoints of the Lagrangian trajectories at time step $i$ after the release, $l_{oi}$ the length of the observed trajectory at time step $i$, and $n$ a tolerance threshold. Thus if $ss=1$ it implies a perfect match while $ss=0$ means no skill. To investigate the skill of the FjordOs model we released Lagrangian particles into OpenDrift and computed the skill-score for each of the 15 drifters released as part of the FjordOs project (Section \ref{subsec:drifto}) using the FjordOs hindcast as input. Furthermore, to investigate whether the FjordOs model has a better skill than a coarser resolution model we also computed the skill-score using the NorKyst800 model results as input. In computing the skill-score we used a tolerance threshold $n=2$ to get positive values for most trajectories. 

\begin{table}[htb]
  \begin{center}
    \begin{tabular}{ | c | c | c | c | c |}
    \hline
		& \multicolumn{2}{|c|}{\bf \small{Full trajectory}}  & \multicolumn{2}{|c|}{\bf \small{First hour only}} \\ 
   {\bf \small Drop no.} & {\bf \small FjordOs} & {\bf \small NorKyst-800} & {\bf \small FjordOs} & {\bf \small NorKyst-800}\\ \hline
{\small	1}	& {\small 0.77 (11)}	& {\small 0.68 (11)}	& {\small 0.17}	& {\small 0.14} \\ 
{\small	4}	& {\small 0.78 (2)}	& {\small 0.53 (1)}	& {\small 0.55}	& {\small 0.45} \\
{\small	6}	& {\small 0.00 (3)}	& {\small 0.84 (11)}	& {\small 0.07}	& {\small 0.46} \\
{\small	8}	& {\small 0.82 (21)}	& {\small 0.58 (21)}	& {\small 0.00}	& {\small 0.00} \\
{\small	51}	& {\small 0.58 (3)}	& {\small 0.73 (15)}	& {\small 0.49}	& {\small 0.58} \\
{\small	52}	& {\small 0.38 (4)}	& {\small 0.50 (1)}	& {\small 0.37}	& {\small 0.47} \\
{\small	61}	& {\small 0.58 (3)}	& {\small 0.70 (8)}	& {\small 0.49}	& {\small 0.58} \\
{\small	91}	& {\small 0.51 (8)}	& {\small 0.50 (12)}	& {\small 0.73}	& {\small 0.67} \\
{\small	101}	& {\small 0.61 (15)}	& {\small 0.78 (15)}	& {\small 0.63}	& {\small 0.78} \\
{\small	102}	& {\small 0.59 (6)}	& {\small 0.75 (7)}	& {\small 0.87}	& {\small 0.77} \\ \hline
{\bf\small Avg.}& {\bf\small 0.56}	& {\bf \small 0.66}	& {\bf \small 0.43}	& {\bf \small 0.49}\\
    \hline
    \end{tabular}
    \caption{\small Skill-score of drifter trajectories released during the September 2015 cruise. The numbers in parenthesis in column two and three indicate how many hours we were able to follow each model trajectory. The last two columns reveal the skill-scores following each trajectory for the first hour only.}
    \label{tab:skillscore_full}
  \end{center}
\end{table}

Regarding the 13 drifters released in September 2015 we used both wind and currents as input. Note that the wind input used in OpenDrift is the same as the one we used to force both FjordOs and NorKyst800. The resulting skill-scores for 10 out of the 13 drifters are presented by Table~\ref{tab:skillscore_full}, while the trajectories are displayed by Figures \ref{fig:opendrift_trajectories1} - \ref{fig:opendrift_trajectories3}. The rationale for limiting the number of drifters to 10 is that some of them stranded too soon after the release to compute a meaningful skill-score. 

As revealed by Table~\ref{tab:skillscore_full} we observe that the \emph{average} NorKyst-800 skill-scores are higher compared to FjordOs. This is also true when we consider the \emph{average} skill-score of the first hour of the trajectories. Nevertheless we notice that FjordOs has a higher skill-score for some of the full trajectories, e.g., drop nos. 1, 2, and 8, and also for the 1 hour trajecories, e.g., 1, 2, 91, and 102. Furthermore by performing a visual comparison between the trajectories (Figures \ref{fig:opendrift_trajectories1} - \ref{fig:opendrift_trajectories3}) we are tempted to conclude that the trajectories based on the FjordOs model more often compares well to the observed trajectories than those based on NorKyst-800. We leave this up to the reader to decide. We merely state that the skill-scores alone appear to be insufficient to make a conclusive statement.

\begin{figure}[htb]
	\centerline{
		\includegraphics*[width=.5\textwidth]{Figurer/opendrift/skillscore/drop1i0}
		\includegraphics*[width=.5\textwidth]{Figurer/opendrift/skillscore/drop4i7}
		}
	\caption{\small Displayed are modelled and observed drifter trajectories September 2015. The colours indicate trajectories based on NorKyst-800 (green), FjordOs (blue), and observed (red). Dotted lines form a grid of equal distances with origo at the release point. Drop 1 is to the left (5 km grid), while drop 4 is to the right (0.5 km).}
	\label{fig:opendrift_trajectories1}
\end{figure}

\begin{comment}\begin{figure}[htb]
	\centerline{
		\includegraphics*[width=.5\textwidth]{Figurer/opendrift/skillscore/drop6i12}
		\includegraphics*[width=.5\textwidth]{Figurer/opendrift/skillscore/drop8i1}
		}
	\caption{\small As Figure~\ref{fig:opendrift_trajectories1}, but showing drop 6 (left, 2 km grid), and drop 8 (right, 5 km grid).}
	\label{fig:opendrift_trajectories1a}
\end{figure}\end{comment}

\begin{figure}[htb]
	\centerline{
		\includegraphics*[width=.5\textwidth]{Figurer/opendrift/skillscore/drop6i12}
		\includegraphics*[width=.5\textwidth]{Figurer/opendrift/skillscore/drop8i1}
		}
	\centerline{
		\includegraphics*[width=.5\textwidth]{Figurer/opendrift/skillscore/drop51i4}
		\includegraphics*[width=.5\textwidth]{Figurer/opendrift/skillscore/drop52i8}
		}
	\caption{\small As Figure \ref{fig:opendrift_trajectories1}, but showing drop 6 (top left, 2 km grid), and drop 8 (top right, 5 km grid), drop 51 (bottom left, 2 km grid) and drop 52 (bottom right, 1 km grid).}
	\label{fig:opendrift_trajectories2}
\end{figure}

\begin{figure}[htb]
	\centerline{
		\includegraphics*[width=.5\textwidth]{Figurer/opendrift/skillscore/drop61i5}
		\includegraphics*[width=.5\textwidth]{Figurer/opendrift/skillscore/drop91i11}
		}
	\caption{\small As Figure \ref{fig:opendrift_trajectories1}, but showing drop 61 (lef-hand panel, 1 km grid) and drop 91 (right-hand panel, 1 km grid).}
	\label{fig:opendrift_trajectories2a}
\end{figure}

\begin{figure}[htb]
	\centerline{
		\includegraphics*[width=.5\textwidth]{Figurer/opendrift/skillscore/drop101i6}
		\includegraphics*[width=.5\textwidth]{Figurer/opendrift/skillscore/drop102i9}
		}
	\caption{\small As Figure \ref{fig:opendrift_trajectories1}, but displaying drop 101 (left-hand panel, 5 km grid) and drop 102 (right-hand panel, 1 km grid).}
	\label{fig:opendrift_trajectories3}
\end{figure}

Regarding the two drifters released in September 2014 we only used current as input. The resulting skill-scores are presented by Tables~\ref{tab:skillscore2014_1h} and \ref{tab:skillscore2014_24h}, while Figures~\ref{fig:opendrift_trajectories2014} and \ref{fig:opendrift_trajectories2014a} displays the modelled and observed trajectories. As revealed the FjordOs model performs better than the NorKyst-800, but there is a clear weakness here since hourly data from NorKyst-800 is not available. Also, there are only two drifters. WE observe that the FjordOs model has the highest skill-score based on both hourly and daily average data, and was the only model to transport the drifters as far south as the observed drift. 

\begin{table}[htb]
\begin{center}
  \begin{tabular}{ | c | c | c |}
    \hline
    {\bf Drop no.} & {\bf FjordOs 1h} & {\bf NorKyst-800m 24h} \\ \hline
    1 & 0.79 (15) & 0.73 (15) \\ 
    2 & 0.84 (19) & 0.76 (19) \\ \hline
    {\bf Avg.} & {\bf 0.81} & {\bf 0.75} \\
    \hline
  \end{tabular}
\caption{\small As Table \ref{tab:skillscore_full}, but for the drifter release during the September 2014 cruise. Note that the FjordOs data are based on hourly resolution as input, while the NorKyst-800 data are based on daily averages.}
\label{tab:skillscore2014_1h}
\end{center}
\end{table}

\begin{table}[htb]
\begin{center}
  \begin{tabular}{ | c | c | c |}
    \hline
    {\bf Drop no.} & {\bf FjordOs 24h} & {\bf NorKyst-800m 24h} \\ \hline
    1 & 0.85 (15) & 0.73 (15) \\ 
    2 & 0.91 (19) & 0.76 (19) \\ \hline
    {\bf Avg.} & {\bf 0.88} & {\bf 0.75} \\
    \hline
  \end{tabular}
\caption{\small As Table \ref{tab:skillscore2014_1h}, but modelled trajectories from FjordOs are based on daily average ocean currents to get a more fair comparison between the models.}
\label{tab:skillscore2014_24h}
\end{center}
\end{table}

\begin{figure}[htb]
	\centerline{
		\includegraphics*[width=.5\textwidth]{Figurer/opendrift/skillscore/2014drop1i0}
		\includegraphics*[width=.5\textwidth]{Figurer/opendrift/skillscore/2014drop1i0_24h}
		}
	\caption{\small As Figure \ref{fig:opendrift_trajectories1}, but showing drop 1 of the September 2014 release.}
	\label{fig:opendrift_trajectories2014}
\end{figure}

\begin{figure}[htb]
	\centerline{
		\includegraphics*[width=.5\textwidth]{Figurer/opendrift/skillscore/2014drop2i1}
		\includegraphics*[width=.5\textwidth]{Figurer/opendrift/skillscore/2014drop2i1_24h}
		}
	\caption{\small As Figure \ref{fig:opendrift_trajectories2014}, but for drop 2 2014.}
	\label{fig:opendrift_trajectories2014a}
\end{figure}

We conclude that none of the models gives a perfect match, and that the currents in NorKyst-800 appears to be too weak during these drifter releases. Furthermore we think it is safe to conclude that the FjordOs model provides a better skill overall. It should be kept in mind though that, given the low number of drifters and their limited spatial and temporal distribution, more drifter studies should be performed in which the released drifters should have a better temporal and spatial distribution. In particular it would be interesting to release drifters in some of the narrow straits and sounds, and in the Archipelagoes.
