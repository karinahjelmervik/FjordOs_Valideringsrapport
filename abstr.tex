{\bf \sffamily Abstract}                                          \\
\small{Provided is an evaluation on the performance of the FjordOs model, a new circulation model covering the Oslofjord, Norway. The model is developed to improve the ocean input (e.g., currents) to emergency models used to predict pathways of oil and/or other effluents. The FjordOs model is a regional adaption of the Regional Ocean Modeling System (ROMS), and makes use of the model's curvilinear option to increase the resolution without inflating the computer demand significantly. To assess the model's rendition of the circulation we compare results from a simulation near two years long to available observations. The observations encompass water level, currents and temperature at various time periods at fixed stations, and observed trajectories of drifters. The evaluation reveals that the model is not perfect, but nevertheless we argue that its performance is adequate for its purpose. An important justification is that the higher resolution offers a decrease in the number of stranded trajectories compared to models of coarser resolution. Also of importance is that the model provides a realistic depth profile of the currents, and the tidal elevation.}



