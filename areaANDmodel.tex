\section{The Oslofjord and the FjordOs model}
The area of interest is the Oslofjord including the Drammensfjord and the Inner Oslofjord as shown by Fig.~\ref{fig:kart}. The map also conveniently displays the domain covered by the FjordOs model. The fjord is located in southeastern Norway and is well described in the literature, e.g., \cite{baalsrud:2002}, \cite{roed:etal:2016} and \cite{hjelm:etal:2017}. Here we only point out some facts that should be kept in mind when establishing a circulation model aimed at providing pathways of various effluents to the fjord.

As revealed by Fig.~\ref{fig:kart}, the fjord is rather long and narrow with occasional wider parts. In the north-south direction it is $\sim$100 km long. Where it borders on the Skagerrak it is $\sim$50 km wide. Proceeding northwards it tapers to $\sim$10 km at Horten-Moss before opening up into Breiangen which is $\sim$20 km wide. At about 59.5$^o$N the fjord splits into two branches. An eastern branch narrows into the the $\sim$10 km long and narrow Dr{\o}bak Sound ($\sim$1-2 km wide), before it opens up to form the somewhat wider Inner Oslofjord with its characteristic "swan head". A western branch tapers into a very narrow strait about 180 m wide at Svelvik before it opens up a bit to form the $\sim$10 km long Drammensfjord.


\begin{figure}[htb]
\centerline{
\includegraphics*[trim=0cm 0.9cm 0cm 0cm,clip=true,width=0.7\textwidth]{Figurer/kart}
}
\caption{\small
Displayed is the Oslofjord conveniently also showing the area covered by the FjordOs model. The red dots show the location of some major and minor cities and villages along the coast mentioned in the text. The blue diamnond indicates the position of the F{\ae}rder Lighthouse, which is close to the model's southern boundary.
}
\label{fig:kart}
\end{figure}

Both branches have a sill. The sill in the eastern branch, the Dr{\o}bak Sill, is located close to the island Kaholmen which holds the citadel Oscarsborg. It consists partly of a man made underwater jetty about 1-2 m deep and extending halfway across the fjord from the western side. East of the jetty there is a natural sill $\sim$20 m deep. Due its narrowness and shallowness the Dr{\o}bak Sill area is famous for its strong tidal currents, which easily exceeds 1 m/s even though the mean total tidal amplitude is less than 0.2 m. The sill in the western branch is in contrast rather long and narrow, about 1 km long and 180 m wide, and has a minimum depth as shallow as 11 m. This sill also causes a strong tidal current called the Svelvikstraum. In both branches the maximum depth north of the sills is more than 120 m. 

In addition to the existence of many small and large islands giving rise to many narrow sounds and straits, the fjord also have several deep basins ranging from 190 to 400 m depths. Moreover, the fjord also exhibit a rather irregular coastline, and several rivers discharges their freshwater into the fjord. Among the latter are two of Norway's largest rivers, namely Glomma (near Fredrikstad) and Drammenselva (near Drammen). An important contributer to the water level variations and thereby the circulation pattern in the fjord is the impact of events in the Skagerrak/North Sea through the fjord's southern perimeter. For instance are storm surge events with amplitudes of one meter and higher observed in the fjord, which are associated with wind and pressure events in the Skagerrak/North Sea.

These complexities all contributes to a compounded circulation pattern. Furthermore since many of the effluents are inside of archipelagoes, e.g. Glomma, it is of some importance to resolve the many straits and narrow sounds when designing a circulation model for the fjord aimed at providing as realistic as possible pathways of effluents. To conceivably account for all these complexities and at the same time not exceeding the available computer capacities we opted to adapt the Rutgers Regional Ocean Modeling System (ROMS) when constructing a new model for the Oslofjord. In this we decided to exploit its curvilinear option. ROMS is a publicly available ocean model featuring a terrain-following vertical coordinate and a free-surface, and is well documented by \cite{haidv:etal:2008} and by \cite{shche:mcwil:2003,shche:mcwil:2005,shche:mcwil:2009}. The particular version adapted to the Oslofjord is called FjordOs. For details on the FjordOs model and the simulations performed for the period April 2014 through December 2015, which is used in the evaluation presented in Section \ref{sec:evalu}, the reader is referred to \cite{roed:etal:2016}. The latter reference also describes the setup including the applied external inputs, such as atmospheric input, river input, tidal input, and the input of sea level, currents and hydrography at the model's open lateral southern boundary.

Finally we emphasize that since ROMS is a terrain-following model it is afflicted by the ubiquitous pressure gradient errors \citep[][and references therein]{bernt:thiem:2007}. To minimize its effect we have, as is common, smoothed the topography to avoid excessive pressure gradient errors to appear. Thus the real Oslofjord topography differs from the model topography. The effect is to lessen the gradient of steep slopes, for instance close to the coastline and at shelf breaks at the rim of deeper basins. This should be kept in mind when comparing model results and observations.   

%The FjordOs model covers the area of interest (Fig.~\ref{fig:kart}). The model period in this study is April 2014 to December 2015.

%\clearpage
