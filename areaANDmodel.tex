\section{The Oslofjord and the FjordOs model}
\label{sec:area}
% % % % 
\subsection{The Oslofjord}
The area of interest, and the domain covered by the model FjordOs, is the Oslofjord including the Drammensfjord and the Inner Oslofjord (Fig.~\ref{fig:kart}). The fjord is located in southeastern Norway and is well described in the literature \citep[e.g., ][]{baalsrud:2002, roed:etal:2016, hjelm:etal:2017}. Here we merely point out some salient facts that should be kept in mind when establishing a circulation model aimed at providing pathways of various effluents to the fjord.

As revealed by Figure \ref{fig:kart}, the fjord is rather long and narrow with occasional wider parts. At about 59.5$^o$N the fjord splits in two branches. A western branch tapers into a narrow strait at Svelvik before it opens up somewhat to form the Drammensfjord. An eastern branch forms the long and narrow Dr{\o}bak Sound, before it also opens up to form the Inner Oslofjord with its characteristic "swan head". In the north south direction it is about 100 km long. At the entrance it is about 50 km wide, in the Dr{\o}bak Sound about 1-2 km wide, and as narrow as 180 meters at Svelvik.
\begin{figure}[htb]
	\centerline{\includegraphics*[trim=0cm 0.9cm 0cm 0cm,clip=true,width=0.7\textwidth]{Figurer/kart} }
	\caption{\small Displayed is the area covered by the Oslofjord and the FjordOs model. The red dots show the locations of some major and minor cities and villages along the coast, and which are mentioned in the text. The blue diamond indicates the position of the F{\ae}rder Lighthouse close to the model's southern boundary.}
	\label{fig:kart}
\end{figure}






Both branches have a sill. The sill in the eastern branch, the Dr{\o}bak Sill, is located close to the island Kaholmen which holds the citadel Oscarsborg. It consists partly of a man made underwater jetty only 1-2 meters deep extending halfway across the fjord from the western side. East of the jetty there is a natural sill of about 20 meters depth. Due its narrowness and shallowness the Dr{\o}bak Sill area is famous for its strong tidal currents, which easily exceeds 1 m/s even though the mean total tidal amplitude is less than 20 cm. The sill in the western branch is rather long and narrow, about 1 km long and 180 meters wide. The minimum depth is as shallow as 11 meters. This sill also causes a strong tidal current called the Svelvikstraum. North of the sills the maximum depth is more than 120 meters in both branches. 

In addition to the existence of many small and large islands giving rise to many narrow sounds and straits, the fjord also have several deep basins ranging from 190 to 400 meters depths. Moreover, the fjord also exhibit a rather irregular coastline, and several rivers discharging fresh water into the fjord. Among the latter are two of Norway's largest rivers, namely Glomma (near Fredrikstad) and Drammenselva (near Drammen). An important contributer to the water level variations and thereby the circulation pattern in the fjord is the impact of events in the Skagerrak/North Sea through the Oslofjord's southern perimeter. For instance are storm surge events with amplitudes of one meter and higher observed in the fjord, which are associated with wind and pressure events in the Skagerrak/North

% % % % 
\subsection{The FjordOs model}
These complexities all contributes to a compounded circulation pattern, a pattern that is important to resolve when designing a circulation model for the fjord aimed at providing as realistic as possible pathways of effluents. To conceivably account for all these complexities and at the same time not exceeding the available computer capacities we opted to adapt the Rutgers Regional Ocean Modeling System (ROMS) when constructing the model for the Oslofjord, and to exploit its curvilinear option. ROMS is a publicly available ocean model featuring a terrain-following vertical coordinate and a free-surface. It is well documented by \cite{haidv:etal:2008} and by \cite{shche:mcwil:2003,shche:mcwil:2005,shche:mcwil:2009}. The particular version adapted to the Oslofjord is called FjordOs. For details on the FjordOs model and the simulations performed for the two years 2014 and 2015 the reader is referred to \cite{roed:etal:2016}. The latter also describes the setup including the applied external inputs, such as atmospheric input, river input, tidal input, and the input of sea level, currents and hydrography at the model's open lateral southern boundary.

Finally we emphasize that since ROMS is a terrain-following model it is plagued by currents created by the inescapable pressure gradient error \citep{haney:1991, bernt:thiem:2007}. To minimize its effect we have, as is common, smoothed the topography to avoid excessive pressure gradient errors to appear. Thus the real Oslofjord topography differs from the model topography. The effect is to lessen the gradient of steep slopes, for instance close to the coastline and at shelf breaks of the deeper basins. This should be kept in mind when comparing model results and observations.   

%The FjordOs model covers the area of interest (Fig.~\ref{fig:kart}). The model period in this study is April 2014 to December 2015.

%\clearpage
